\subsection{Experimental Results}

\subsubsection{Experiment 1}

\begin{table}[ht]
	\centering
	{\renewcommand{\arraystretch}{1.75}
	\begin{tabular}{|p{3cm}|p{3cm}|p{3cm}|p{3cm}|}
		\hline
		Temperature ($\degree C$) & $1^{st}$ & $2^{nd}$ & $3^{rd}$ \\ \hline
		$t_1$ & & & \\ \hline
		$t_2$ & & & \\ \hline
		$t_3$ & & & \\ \hline
		$m_0c_0$ & & & \\ \hline
	\end{tabular}}
\end{table}
$\displaystyle \overline{m_0c_0}=$ \hspace{2cm} ($cal/K$)\\

Detailed calculations using the $1^{st}$ test: \[m_0c_{0-1} = mc \frac{(t_3-t_1)-(t_2-t_3)}{t_2-t_3}= \hspace{6cm}\]	

\subsubsection{Experiment 2}

\begin{table}[ht]
	\centering
	{\renewcommand{\arraystretch}{1.75}
		\begin{tabular}{|p{3.5cm}|p{3cm}|p{3cm}|p{3cm}|}
			\hline
			Temperature ($\degree C$) & $1^{st}$ & $2^{nd}$ & $3^{rd}$ \\ \hline
			$t_1$ & & & \\ \hline
			$t_2$ & & & \\ \hline
			$t_3$ & & & \\ \hline
			$Q$ & & & \\ \hline
			$\overline{Q}$ & \multicolumn{3}{l|}{} \\ \hline
			$\Delta H = \frac{\overline{Q}}{n}$ ($cal/mol$) & \multicolumn{3}{l|}{} \\ \hline
	\end{tabular}}
\end{table}
Detailed calculations using the $1^{st}$ test:
\[Q_1 = (m_0c_0 + m_{\ch{HCl}}c_{\ch{HCl}}+m_{\ch{NaCl}}c_{\ch{NaCl}})(t_3-\frac{t_1+t_2}{2})\]
\[= \hspace{7.4cm}\]
\newpage
\subsubsection{Experiment 3}

\begin{table}[ht]
	\centering
	{\renewcommand{\arraystretch}{1.75}
		\begin{tabular}{|p{3cm}|p{3cm}|p{3cm}|p{3cm}|}
			\hline
			Temperature ($\degree C$) & $1^{st}$ & $2^{nd}$ & $3^{rd}$ \\ \hline
			$t_1$ & & & \\ \hline
			$t_2$ & & & \\ \hline
			$Q$ ($cal$) & & & \\ \hline
			$\Delta H$ ($cal/mol$) & & & \\ \hline
			$\Delta \overline{H}$ ($cal/mol$) & \multicolumn{3}{l|}{} \\ \hline
	\end{tabular}}
\end{table}
Detailed calculations using the $1^{st}$ test:
\[Q_1 = (m_0c_0 + m_{\ch{H2O}}c_{\ch{H2O}}+m_{\ch{CuSO4}}c_{\ch{CuSO4}})(t_2-t_1)\]
\[= \hspace{7cm}\]\vspace{0.1cm}
\[\Delta H_1 = \frac{Q_1}{n} = \hspace{2cm} = \hspace{4.5cm}\]
Since this is 

\subsubsection{Experiment 4}
\begin{table}[ht]
	\centering
	{\renewcommand{\arraystretch}{1.75}
		\begin{tabular}{|p{3cm}|p{3cm}|p{3cm}|p{3cm}|}
			\hline
			Temperature ($\degree C$) & $1^{st}$ & $2^{nd}$ & $3^{rd}$ \\ \hline
			$t_1$ & & & \\ \hline
			$t_2$ & & & \\ \hline
			$Q$ ($cal$) & & & \\ \hline
			$\Delta H$ ($cal/mol$) & & & \\ \hline
			$\Delta \overline{H}$ ($cal/mol$) & \multicolumn{3}{l|}{} \\ \hline
	\end{tabular}}
\end{table}
Detailed calculations using the $1^{st}$ test:
\[Q_1 = (m_0c_0 + m_{\ch{H2O}}c_{\ch{H2O}}+m_{\ch{NH4Cl}}c_{\ch{NH4Cl}})(t_2-t_1)\]
\[= \hspace{6.9cm}\]\vspace{0.1cm}
\[\Delta H_1 = \frac{Q_1}{n} = \hspace{2cm} = \hspace{4.4cm}\]
This is 

\subsection{Answer the questions}
\begin{enumerate}
	\item Is $\Delta \overline{H}$ of reaction \ch{HCl + NaOH -> NaCl + H2O} calculated based on the molar of \ch{HCl} or \ch{NaOH} when 25ml \ch{Hcl} 2M solution reacts with 25ml \ch{NaOH} 1M solution? Explain.
	
	- We have the following relation:
	
	\hspace{2cm}\ch{HCl + NaOH -> NaCl + H2O}
	
	\hspace{2cm}0.05 \hspace{0.4cm} 0.025
	
	\hspace{2cm}0.025 \hspace{0.2cm} 0.025
	
	\hspace{2cm}0.025 \hspace{0.2cm} 0
	
	- NaOH was fully reacted and HCl still remained 0.025 mol. Thus, $\Delta H$ was calculated according to $n_{\ch{NaOH}}$. Since the residue HCl did not react, the total reaction did not produce heat.
	
	\item If we replaced HCl 1M with \ch{HNO3} 1M, would the result in experiment 2 change? Explain.
	
	- It would be the same , since $\Delta H$ is a specific quantity for each separated reaction. Replacing HCl 1M with \ch{HNO3} 1M resulted in neutralization of the solution due to having the same number of moles (or concentration of \ch{H+}).
	
	\item Calculate $\Delta H_3$ based on Hess's law. Compare the result with its experimental one. Considering 6 reasons for errors.
	\begin{itemize}[label=-]
		\item Heat loss from calorimeter.
		\item Thermometer.
		\item Volumetric glassware.
		\item Balance.
		\item Water absorption from \ch{CuSO4}.
		\item Assuming that specific heat of \ch{CuSO4} is $1$ ($cal/mol \cdot K$).
	\end{itemize}
	Which could be the most significant factor causing errors? Explain.
	
	- Hess's law: $\Delta H_3 = \Delta H_1 + \Delta H_2 = \hspace{3cm} = $ \hspace{2cm} ($kCal/mol$)
	
	- The experimental results are smaller than the theoretical $\Delta H_3$ calculated by Hess's law. The most important reason for this comes from \ch{CuSO4} absorbing a small amount of water at anhydrous form:
	
	\hspace{2cm}\ch{CuSO4_{(anhydrous)} + 5 H2O -> CuSO4. 5 H2O}
	
	- The equation above produces $\Delta H_1$. Once \ch{CuSO4} is in hydrated form, it releases less heat than that in theory.
	
	- In addition, hydrated \ch{CuSO4} causes difference in $n_{\ch{CuSO4}}$ since $n_{\ch{CuSO4_{(anhydrous)}}} \neq n_{\ch{CuSO4_{(hydrated)}}}$.
\end{enumerate}

