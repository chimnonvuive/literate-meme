\chapter{Motor Design}
\section{Nomenclature}
\begin{tabular}[t]{lp{7cm}}
	$ \eta_c $ & coupling efficiency\\
	$ \eta_b $ & bearing efficiency\\
	$ \eta_{hg} $ & helical gear efficiency\\
	$ \eta_{ch} $ & chain drive efficiency\\
	$ \eta_{sys} $ & efficiency of the system\\
	$ P_m $ & maximum operating power of belt conveyor, $ \unit{kW} $\\
	$ P_w $ & opearting power of belt conveyor given a workload, $ \unit{kW} $\\
	$ P_{motor} $ & calculated motor power to drive the system, $ \unit{kW} $\\
	$ P_{sh} $ & operating power of shaft, $ \unit{kW} $\\
\end{tabular}
\begin{tabular}[t]{lp{7cm}}
	$ n_{bc} $ & rotational speed of belt conveyor, $ \unit{rpm} $\\
	$ n_{sh} $ & rotational speed of shaft, $ \unit{rpm} $\\
	$ u_{hg} $ & transmission ratio of helical gear\\
	$ u_{ch} $ & transmission ratio of chain drive\\
	$ u_{sys} $ & transmission ratio of the system\\
	$ T_{motor} $ & motor torque, $ \unit{N\cdot mm} $\\
	$ T_{sh} $ & shaft torque, $ \unit{N\cdot mm} $
\end{tabular}

\section{Calculate $ \eta_{sys} $}
From table 2.3 :\\
$ \eta_c = 1 $, $ \eta_b = 0.99 $, $ \eta_{hg} = 0.96 $, $ \eta_{ch} = 0.95 $\\
$ \eta_{sys} = \eta_c\eta_b^3\eta_{hg}\eta_{ch} \approx 0.88 $

\section{Calculate $ P_{motor} $}
$ P_m = \dfrac{F_tv}{1000} \approx 13.73 \unit{(kW)}$\\\\
From equation (2.13) :\\
$ P_w = P_m\sqrt{\dfrac{\left(\dfrac{T_1}{T}\right)^2t_1 + \left(\dfrac{T_2}{T}\right)^2t_2}{t_1+t_2}} \approx 10.41 \unit{(kW)} $\\
$ P_{motor} = \dfrac{P_w}{\eta_{sys}} \approx 11.76\unit{(kW)}$

\section{Calculate $ n_{motor} $}
$ n_{bc} = \dfrac{6\times10^4v}{\pi D} \approx 116.5 \unit{(rpm)}$\\
$ u_{ch} = 5$, $ u_{hg} = 5$ (table 2.4)\\
$ u_{sys} = u_{ch}u_{hg} = 25 $\\
$ n_{motor} = u_{sys}n_{bc} \approx 2912.54  \unit{(rpm)} $

\section{Choose motor}
To work normally, the maximum operating power of the chosen motor must be no smaller than both estimated $ P_{motor} $ and $ P_m $. Since $ P_{motor} < P_m $ for our case, the minimum operating power of choice is $ P_m $. In similar fashion, its rotational speed must also be no smaller than estimated $ n_{motor} $.\\
Thus, from table P1.3, we choose motor 4A160M2Y3 which operates at $ 18.5\unit{kW} $ and $ 2930\unit{rpm} $\\
$\Rightarrow P_{motor} = 18.5\unit{kW}, n_{motor} = 2930\unit{(rpm)}$\\
Recalculating $ u_{sys} $ with the new $ P_{motor} $ and $ n_{motor} $, we obtain:\\
$ u_{sys} = \dfrac{n_{motor}}{n_{bc}} \approx 25.15	$\\
Assuming $ u_{hg} = const $:\\
$ u_{ch} = \dfrac{u_{sys}}{u_{hg}} \approx 5.03$

\section{Calculate power, rotational speed and torque of the motor and 2 shafts}
Let us denote $ P_{sh1} $, $ n_{sh1} $ and $ T_{sh1} $ be the transmitted power, rotational speed and torque onto shaft 1, respectively. Similarly, $ P_{sh2} $, $ n_{sh2} $ and $ T_{sh2} $ will be the transmitted parameters onto shaft 2. These notations will be used throughout the next chapters.
\subsection{Power}
$ P_{ch} = P_m \approx 13.72 \unit{(kW)}$\\
$ P_{sh2} = \dfrac{P_{ch}}{\eta_{ch}} \approx 14.45 \unit{(kW)}$\\
$ P_{sh1} = \dfrac{P_{sh2}}{\eta_b\eta_{hg}} \approx 15.2 \unit{(kW)}$\\
$ P_{motor} = \dfrac{P_{sh1}}{\eta_b\eta_c} \approx 15.35 \unit{(kW)}$
\subsection{Rotational speed}
$ n_{sh1} = n_{motor} = 2930\unit{(rpm)}$\\
$ n_{sh2} = \dfrac{n_{sh1}}{u_{hg}} = 586 \unit{(rpm)}$
\subsection{Torque}
$ T_{motor} = 9.55\times10^6 \dfrac{P_{motor}}{n_{motor}} \approx 50047.36 \unit{(N\cdot mm)}$\\
$ T_{sh1} = 9.55\times10^6 \dfrac{P_{sh1}}{n_{sh1}} \approx 49547.08 \unit{(N\cdot mm)}$\\
$ T_{sh2} = 9.55\times10^6 \dfrac{P_{sh2}}{n_{sh2}} \approx 235447.73 \unit{(N\cdot mm)}$\\\\
In summary, we obtain the following table:\\\\\\\\\\
\begin{table}[ht]
	\centering
	\begin{tabular}{|
			>{\columncolor[HTML]{C0C0C0}}l |l|l|l|l|}
		\hline
		& \multicolumn{1}{c|}{\cellcolor[HTML]{C0C0C0}Motor} & \multicolumn{2}{c|}{\cellcolor[HTML]{C0C0C0}Shaft 1} & \multicolumn{1}{c|}{\cellcolor[HTML]{C0C0C0}Shaft 2} \\ \hline
		$ P \unit{(kW)}$ & 15.35                                            & \multicolumn{2}{l|}{15.2}                          & 14.45                                              \\ \hline
		$ u $ & \multicolumn{2}{p{3.cm}|}{5}                                                         & \multicolumn{2}{l|}{5.03}                                                       \\ \hline
		$ n \unit{(rpm)}$ & 2930                                               & \multicolumn{2}{l|}{2930}                            & 586                                                  \\ \hline
		$ T \unit{(N\cdot mm)}$ & 50047.36                                         & \multicolumn{2}{l|}{49547.08}                      & 235447.73                                          \\ \hline
	\end{tabular}
	\caption{System properties}
\end{table}