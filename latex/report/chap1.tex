\chapter{Overview of the Company}
\section{Tabular Re-arrangement}
Example:

The sterilization temperature for sterilizing the tank should be at least $ 135\degc $ for $ 30 $ min. The sterilizing temperature at the condensomat should not fall below $ 125\degc $.

You can "tabularize" the body of paragraph into a table, which is more concise and readable.\\\\
\textbf{Minimum temperatures for sterilizing}
\begin{table}[ht]
	\begin{tabular}{p{0.4\textwidth}p{0.4\textwidth}}
		\textbf{location} & \textbf{temperature}\\
		at the tank & temp. $ =135\degc $, 30 min\\
		at the condensomat & temp. $ =125\degc $
	\end{tabular}
\end{table}

\section{Choosing between diagram, table and text descriptions}
In general:
\begin{itemize}
	\item Text description is exact but hard to read.
	\item Table is exact but still not easy to read.
	\item Diagram is not exact but easy to visualize and memorable.
	\item Placing the right thing on a page is essential for a great report.
\end{itemize}
The next important key point to remember is called text-figure-relationship, which is how and where you place texts and figures in a page:
\begin{itemize}
	\item Placing figures (diagrams and tables) in an appendix is good for positioning but leads to turning pages multiple times (bad text-figure-relationship).
	\item Placing figures near/on the same page as the explaining text is a good text-figure-relationship. If the report is double-sided, the figure should be on the same double-page as the text (the diagram and explaining text should either be on odd or even pages).
	\item Cross referencing to a figure with a statement and add the phrase "Figure xx." at the end after a comma.
	\item Drawing/scanning the figures as early as possible while writing the explaining text.
\end{itemize}

Graphics simplify reality (e.g. principle drawing, map), explain abstract ideas by means of spatial arrangement (e.g. bar chart, pie chart, tree chart), create associations (e.g. logo and pictogram).

There are 13 basic rules for information-effective design of figures:
\begin{enumerate}
	\item Accentuate important items.
	\item Delete/leave out unimportant items (use max. 4-7 graphic elements in 1 picture or it is overloaded).
	\item Line thickness and font size must be sufficient. Figures should be readable at a distance of 30-40 cm.
	\item Relationship of graphic elements should be emphasized (lines, arrows, columns, rows, common color). These should be specified in detail (character of the relationship, meaning behind it) using labels and explanations in the legend.
	\item Objects near each other belong together.
	\item An element placed above or below other elements is regarded as hierarchically super-ordinate or sub-ordinate.
	\item Logical/chronological sequence has its elements placed beside each other.
	\item Circular arrangement is thought as a cycle or repeated sequence.
	\item An element surrounding another element is understood as the outer includes the inner element.
	\item Boxes, bars, lines, columns, etc. must be clearly marked (text labels or graphical explanations).
	\item 1 type of element may have only 1 function within one/series of figures (e.g. arrows for vector direction).
	\item $ x- $axis is horizontal, $ y- $axis is vertical
	\item Look up for standardized symbols (e.g. DIN 66001 for flow charts, DIN 32520, DIN 66261).
\end{enumerate}
