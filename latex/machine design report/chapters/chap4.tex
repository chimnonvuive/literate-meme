\chapter{Gearbox Design (Helix gears)}
\section{Nomenclature}
\begin{tabular}[t]{p{0.1\linewidth}p{0.35\linewidth}}
	$ [\sigma_H] $ & permissible contact stress, $ \unit{MPa} $\\
	$ [\sigma_H]_{max} $ & permissible contact stress due to overload, $ \unit{MPa} $\\
	$ [\sigma_F] $ & permissible bending stress, $ \unit{MPa} $\\
	$ [\sigma_F]_{max} $ & permissible bending stress due to overload, $ \unit{MPa} $\\
	$ \text{AG} $ & accuracy grade of gear\\
	$ a $ & center distance, $ \unit{mm} $\\
	$ b $ & face width, $ \unit{mm} $\\
	$ c $ & gear meshing rate\\
	$ d $ & pitch circle, $ \unit{mm} $\\
	$ d_a $ & addendum diameter, $ \unit{mm} $\\
	$ d_b $ & base diameter, $ \unit{mm} $\\
	$ d_f $ & deddendum diameter, $ \unit{mm} $\\
	$ F_a $ & axial force, $ \unit{N} $\\
	$ F_r $ & radial force, $ \unit{N} $\\
	$ F_t $ & tangential force, $ \unit{N} $\\
	$ H $ & surface roughness, HB\\
\end{tabular}
\begin{tabular}[t]{p{0.05\linewidth}p{0.4\linewidth}}
	$ K_d $ & coefficient of gear material\\	
	$ K_F $ & load factor from bending stress\\
	$ K_{FC} $ & load placement factor\\
	$ K_{FL} $ & aging factor due to bending stress\\
	$ K_{Fv} $ & factor of dynamic load from bending stress at meshing area\\
	$ K_{F\alpha} $ & factor of load distribution from bending stress on gear teeth\\
	$ K_{F\beta} $ & factor of load distribution from bending stress on top land\\
	$ K_H $ & load factor of contact stress\\
	$ K_{HL} $ & aging factor due to contact stress\\
	$ K_{Hv} $ & factor of dynamic load from contact stress at meshing area\\
	$ K_{H\alpha} $ & factor of load distribution from contact stress on gear teeth\\
	$ K_{H\beta} $ & factor of load distribution from contact stress on top land\\
	$ k_x $ & a coefficient\\
\end{tabular}\newpage
\noindent\begin{tabular}[t]{p{0.05\linewidth}p{0.4\linewidth}}
	$ k_y $ & a coefficient\\
	$ m $ & traverse module, $ \unit{mm} $\\
	$ m_F $ & root of fatigue curve in bending stress test\\
	$ m_H $ & root of fatigue curve in contact stress test\\
	$ m_n $ & normal module, $ \unit{mm} $\\
	$ N_{FE} $ & working cycle of equivalent tensile stress corresponding to $ [\sigma_F] $\\
	$ N_{FO} $ & working cycle of bearing stress corresponding to $ [\sigma_F] $\\
	$ N_{HE} $ & working cycle of equivalent tensile stress corresponding to $ [\sigma_H] $\\
	$ N_{HO} $ & working cycle of bearing stress corresponding to $ [\sigma_H] $\\
	$ S $ & specific length, $ \unit{mm} $\\
	$ S_F $ & safety factor of bending stress\\
	$ S_H $ & safety factor of contact stress\\
	$ T $ & input torque, $ \unit{N\cdot mm} $\\
	$ v $ & rotational velocity, $ \unit{m/s} $\\
	$ x $ & gear correction factor\\
	$ Y_F $ & tooth shape factor\\
	$ Y_R $ & surface roughness factor of the gear's face\\
	$ Y_s $ & sensitivity to stress concentration factor\\
	$ Y_\beta $ & helix angle factor\\
	$ Y_\varepsilon $ & contact ratio factor\\
	$ y $ & center displacement factor\\
	$ Z_R $ & surface roughness factor of the working's area\\
\end{tabular}
\begin{tabular}[t]{p{0.05\linewidth}p{0.4\linewidth}}
	$ Z_v $ & speed factor\\
	$ z_H $ & contact surface's shape factor\\
	$ z_M $ & material's mechanical properties factor \\
	$ z_{min} $ & minimum number of teeth corresponding to $ \beta $\\
	$ z_v $ & virtual number of teeth\\
	$ z_\varepsilon $ & meshing condition factor\\
	$ \alpha $ & normal pressure angle, following Vietnam standard (TCVN 1065-71), i.e. $ \alpha = 20^\circ $\\
	$ \alpha_t $ & traverse pressure angle, $ ^\circ $\\
	$ \varepsilon_\alpha $ & traverse contact ratio\\
	$ \varepsilon_\beta $ & face contact ratio\\
	$ \beta $ & helix angle, $ ^\circ $\\
	$ \beta_b $ & base circle helix angle, $ ^\circ $\\
	$ \psi_{ba} $ & width to shaft distance ratio\\
	$ \psi_{bd} $ & face width factor \\
	$ \sigma_b $ & ultimate strength, $ \unit{MPa} $\\
	$ \sigma_{ch} $ & yield limit, $ \unit{MPa} $\\
	$ \sigma_{Flim}^o $ & permissible bending stress corresponding to working cycle, $ \unit{MPa} $\\
	$ \sigma_{Hlim}^o $ & permissible contact stress corresponding to working cycle, $ \unit{MPa} $\\
	$ _{1} $ & subscript for the pinion \\
	$ _{2} $ & subscript for the driven gear\\
	$ _q $ & subscript for the quick transmission stage\\
	$ _s $ & subscript for the slow transmission stage\\
	$ _w $ & subscript for the value after correction\\
\end{tabular}

\paragraph{Known parameters} From Chapter 1 and 2, we know that:\\
$ L=8\unit{years} $, $ K_{ng}=260\unit{days} $, $ Ca=1\unit{shift} $\\
$ T_1=T$, $T_2=0.7T$, $ t_1=15\unit{s}$, $t_2=11\unit{s} $\\
$ n_{sh1}=2922\unit{rpm} $, $ n_{sh2}=905.02\unit{rpm} $, $ n_{sh3}=292.20\unit{rpm} $\\
$ u_h=10 $, $ u_1 =u_q=3.23$, $ u_2=u_s=3.1 $

This chapter will increase readability by calculating both stages at the same time with the first one being quick stage and the latter is slow stage.
\section{Choose material}
Because all gears are the same in material and working hours except for their angular rotational speed, this section applies for both pairs.

The material of choice for the 2 pair of gears is steel 40X. The specifications are $ S\leq100\unit{mm} $, $ \text{HB} 250 $, $ \sigma_b = 850\unit{MPa} $, $ \sigma_{ch} = 550 \unit{MPa}$, see Table 6.1  \cite{tk1}.

From Table 6.2 \cite{tk1}, $ \sigma_{Hlim}^o = 2\text{HB} + 70$, $ S_H = 1.1 $, $ \sigma_{Flim}^o = 1.8\text{HB} $, $ S_F = 1.75 $.

Therefore, they have the same properties except for their surface roughness $ H $, since $ H_2 = H_1 - 10 \div 15$.

For the pinion, $ H_1=\text{HB}250 \Rightarrow \sigma_{Hlim1}^o = 570\unit{MPa}$, $ \sigma_{Flim1}^o = 450\unit{MPa}$.

For the driven gear, $ H_2=\text{HB}240 \Rightarrow \sigma_{Hlim2}^o = 550\unit{MPa}$, $ \sigma_{Flim2}^o = 432\unit{MPa}$.

In this part, distinguishing between 2 stages is unnecessary since the variables are material-dependent, which in this case all the gears have identical choice of material. Therefore, unless otherwise specified, a single subscript $ 1 $ or $ 2 $ indicates the variable applies for both stages

\section{Calculate $ [\sigma_H] $ and $ [\sigma_F] $}
The permissible stresses are calculated as follows, see Equation 6.1 and 6.2 \cite{tk1}:
\[[\sigma_{H}]=\dfrac{\sigma_{Hlim}^o}{S_H}Z_RZ_vK_{xH}K_{HL}\]
\[[\sigma_{F}]=\dfrac{\sigma_{Flim}^o}{S_F}Y_RY_sK_{xF}K_{FL}\]

\textbf{Calculate working cycle of bearing stress} The stress is found using the formula below, see Equation 6.5 \cite{tk1}:\\
$ N_{HO1} = 30H_1^{2.4} = 17067789.40\unit{(cycles)}$.\\
$ N_{HO2} = 30H_2^{2.4} = 15474913.67\unit{(cycles)}$.\\

\textbf{Calculate working cycle of equivalent tensile stress} Since $ H_1,H_2\leq\text{HB}350 $, $ m_H=6 $, $ m_F=6 $. Also, both pairs of gears are meshed indefinitely, which makes $ c=1 $. From working condition, we calculate:\\
$ L_h = 8\unit{\left( \dfrac{hours}{shift}\right)}\times Ca  K_{ng} L=16640\unit{(hours)}$\\
Find $ N_{HE} $ and $ N_{FE} $  using equation 6.7 and 6.8 \cite{tk1}:\\
$ N_{HE1q} = 60n_{sh1}cL_h\left[ \left( \dfrac{T_1}{T}\right)^3\dfrac{t_1}{t_1+t_2} + \left( \dfrac{T_2}{T}\right)^3\dfrac{t_2}{t_1+t_2}\right] = 2.11\times10^9\unit{(cycles)}$\\
$ N_{HE2q} = 60n_{sh2}cL_h\left[ \left( \dfrac{T_1}{T}\right)^3\dfrac{t_1}{t_1+t_2} + \left( \dfrac{T_2}{T}\right)^3\dfrac{t_2}{t_1+t_2}\right] = 0.65\times10^9\unit{(cycles)}$\\
$ N_{FE1q} = 60n_{sh1}cL_h\left[ \left( \dfrac{T_1}{T}\right)^{m_F}\dfrac{t_1}{t_1+t_2} + \left( \dfrac{T_2}{T}\right)^{m_F}\dfrac{t_2}{t_1+t_2}\right] = 1.83\times10^9\unit{(cycles)}$\\
$ N_{FE2q} = 60n_{sh2}cL_h\left[ \left( \dfrac{T_1}{T}\right)^{m_F}\dfrac{t_1}{t_1+t_2} + \left( \dfrac{T_2}{T}\right)^{m_F}\dfrac{t_2}{t_1+t_2}\right] = 0.57\times10^9\unit{(cycles)}$\\
$ N_{HE1s} = 60n_{sh2}cL_h\left[ \left( \dfrac{T_1}{T}\right)^3\dfrac{t_1}{t_1+t_2} + \left( \dfrac{T_2}{T}\right)^3\dfrac{t_2}{t_1+t_2}\right] = 0.65\times10^9\unit{(cycles)}$\\
$ N_{HE2s} = 60n_{sh3}cL_h\left[ \left( \dfrac{T_1}{T}\right)^3\dfrac{t_1}{t_1+t_2} + \left( \dfrac{T_2}{T}\right)^3\dfrac{t_2}{t_1+t_2}\right] = 0.21\times10^9\unit{(cycles)}$\\
$ N_{FE1s} = 60n_{sh2}cL_h\left[ \left( \dfrac{T_1}{T}\right)^{m_F}\dfrac{t_1}{t_1+t_2} + \left( \dfrac{T_2}{T}\right)^{m_F}\dfrac{t_2}{t_1+t_2}\right] = 0.57\times10^9\unit{(cycles)}$\\
$ N_{FE2s} = 60n_{sh3}cL_h\left[ \left( \dfrac{T_1}{T}\right)^{m_F}\dfrac{t_1}{t_1+t_2} + \left( \dfrac{T_2}{T}\right)^{m_F}\dfrac{t_2}{t_1+t_2}\right] = 0.18\times10^9\unit{(cycles)}$\\

\textbf{Calculate aging factor} 
For steel, $ N_{FO1}= N_{FO2}= 4\times10^6\unit{(MPa)}$. Equation 6.3 and 6.4 \cite{tk1}  gives (if the factors are smaller than 1, round them up to 1, see p.94 \cite{tk1}):\\
$ K_{HL1q} = \sqrt[m_H]{N_{HO1}/N_{HE1q}} = 0.45 < 1 \Rightarrow K_{HL1q} = 1$\\
$ K_{HL2q} = \sqrt[m_H]{N_{HO2}/N_{HE2q}} = 0.54 < 1 \Rightarrow K_{HL2q} = 1$\\
$ K_{FL1q} = \sqrt[m_F]{N_{FO1}/N_{FE1q}} = 0.36 < 1 \Rightarrow K_{FL1q} = 1$\\
$ K_{FL2q} = \sqrt[m_F]{N_{FO2}/N_{FE2q}} = 0.44 < 1 \Rightarrow K_{FL2q} = 1$\\
$ K_{HL1s} = \sqrt[m_H]{N_{HO1}/N_{HE1s}} = 0.54 < 1 \Rightarrow K_{HL1s} = 1$\\
$ K_{HL2s} = \sqrt[m_H]{N_{HO2}/N_{HE2s}} = 0.65 < 1 \Rightarrow K_{HL2s} = 1$\\
$ K_{FL1s} = \sqrt[m_F]{N_{FO1}/N_{FE1s}} = 0.44 < 1 \Rightarrow K_{FL1s} = 1$\\
$ K_{FL2s} = \sqrt[m_F]{N_{FO2}/N_{FE2s}} = 0.53 < 1 \Rightarrow K_{FL2s} = 1$\\

\textbf{Calculate $ [{\sigma}_H] $, $ [{\sigma}_{F1}] $, $ [{\sigma}_{F2}] $}
Since the motor works in one direction, $ K_{FC} = 1$, which means all $ K $ factors are equal to $ 1 $ and we can safely skip them.

For initial estimation, assume $ Z_RZ_vK_{xH} = 1 $ and $ Y_RY_sK_{xF} = 1 $, we obtain the permissible stresses (notice they are equal in both quick and slow transmission stages):\\
$ [{\sigma}_{H1}] = \sigma_{Hlim1}^o/S_{H} = 518.18 \unit{(MPa)}$\\
$ [{\sigma}_{H2}] = \sigma_{Hlim2}^o/S_{H} = 500 \unit{(MPa)}$\\
$ [{\sigma}_{F1}] = \sigma_{Flim1}^o/S_{F} = 257.14\unit{(MPa)}$\\
$ [{\sigma}_{F2}] = \sigma_{Flim2}^o/S_{F} = 246.86 \unit{(MPa)}$

The mean permissible contact stress must be lower than 1.25 times of either $ [{\sigma}_{H1}] $ or $ [{\sigma}_{H2}] $, whichever is smaller. In this case, it is $ 1.25[{\sigma}_{H2}]$ or $ 625\unit{MPa} $:
\[[{\sigma}_{H}] =\dfrac{1}{2}\left( [{\sigma}_{H1}]+[{\sigma}_{H2}]\right)  = 509.09 \unit{(MPa)}\leq 625\]
which satisfy the condition.

In case of overloading, the permissible contact and bending stresses are calculated as follows:\\
$ [{\sigma}_H]_{max} = 2.8\sigma_{ch} = 1540\unit{(MPa)} $\\
$ [{\sigma}_F]_{max} = 0.8\sigma_{ch} = 440\unit{(MPa)} $



\section{Determine basic specifications of the Transmission system}

\subsection{Determine basic parameters}
	Figure \ref{fig:problem} shows both pairs are helical, which  gives $ K_a = 43 $, Table 6.5 \cite{tk1}. Also, the entire speed reducer has asymmetrical design, resulting in $ \psi_{ba} = 0.3$, Table 6.6 \cite{tk1}. This value is then used in Equation 6.16 \cite{tk1} to find $ \psi_{bd} $:\\
$ \psi_{bdq} = 0.53\psi_{ba}(u_q+1) = 0.67 $\\
$ \psi_{bds} = 0.53\psi_{ba}(u_s+1) = 0.65 $

Using interpolation, we approximate the factors, Table 6.7 \cite{tk1}:\\
$ K_{H\beta q} = 1.04 $, $ K_{F\beta q} = 1.09 $\\
$ K_{H\beta s} = 1.06 $, $ K_{F\beta s} = 1.13 $\\
Since the gear system only consists of involute gears and it is also a speed reducer gearbox, we estimate $ a $, Equation 6.15a \cite{tk1}:\\
$ a_q = K_a(u_{q}+1)\sqrt[3]{\dfrac{T_{sh1}K_{H\beta q}}{[{\sigma}_H]^2u_{q}\psi_{ba}}} = 83.12 \unitp{mm}$\\
$ a_s = K_a(u_{s}+1)\sqrt[3]{\dfrac{T_{sh2}K_{H\beta s}}{[{\sigma}_H]^2u_{s}\psi_{ba}}} = 119.85 \unitp{mm}$\\

It is recommended to round up the center distances to the nearest multiple of 5 for small production, p.99 \cite{tk1}. Thus, $ a_{wq}=85\unit{mm} $ and $ a_{ws}=120\unit{mm} $.

\subsection{Determine gear meshing parameters}

\textbf{Find $ m $} Using Equation 6.17 \cite{tk1} and Table 6.8 \cite{tk1}, we determine $ m $ for each pair of gears:\\
$ m_q = (0.01\div0.02)a_{wq} = 0.85\div1.7 \unitp{mm} \Rightarrow m_q=1.5\unit{(mm)}$\\
$ m_s = (0.01\div0.02)a_{ws} = 1.2\div2.4 \unitp{mm} \Rightarrow m_s=2\unit{(mm)}$
	
\paragraph{Find $ z_1 $, $ z_2 $, $ b_w $} Let $ \beta = 14^\circ $. Combining equation (6.18) and (6.20), we come up with the formula to calculate $ z_1 $. Then, find $ z_2 $ and $ b $.\\
$ z_{1q} = \dfrac{2a_{wq}\cos\beta}{m_q(u_q+1)} = 26.01 \Rightarrow z_{1q} = 27$\\
$ z_{2q} = u_{q}z_{1q} = 87.17 \Rightarrow z_{2q} = 88 $\\
$ b_q = \psi_{ba}a_{wq} = 25.50\unit{(mm)}$\\
$ z_{1s} = \dfrac{2a_{ws}\cos\beta}{m_s(u_s+1)} = 28.42 \Rightarrow z_{1s} = 29$\\
$ z_{2s} = u_{s}z_{1s} = 89.82 \Rightarrow z_{2s} = 90 $\\
$ b_s = \psi_{ba}a_{ws} = 36.00\unit{(mm)}$ 

\paragraph{Correct $ \beta $} There are 2 approaches for correction involving the change of either $ \alpha $ or $ \beta $. Because altering $ \alpha $ leads to many other corrections ($ d_1 $, $ d_2 $ and $ a_w $), $ \beta $ will be used instead.\\
Since $ z_1 $ is rounded, we must find $ \beta $ to obtain the correct angle, ensuring that $ \beta \in (8^\circ, 20^\circ) $. Using equation (6.32):\\
$ \beta_w = \arccos\dfrac{m(z_1+z_2)}{2a_w} \approx 13.59^\circ$

\paragraph{Find $ x_1 $, $ x_2 $} To find $ x_1 $ and $ x_2 $, we will follow the calculation scheme provided in p.103. Since $ \beta_w \approx 13.59^\circ \in (10,15]$, $ z_{min} = 11$, which leads to $ z_1 $ satisfying condition $ z_1 \geq z_{min} + 2 > 10 $, according to table (6.9). Combined with $ u_{hg} = 5 \geq 3.5 $, we obtain $ x_1 = 0.3 $, $ x_2 = -0.3 $, disregarding the calculation of $ y $.

\subsection{Basic parameters}

\begin{tabular}[t]{p{8cm}}
	$ d_1 = d_{w1} = \dfrac{mz_1}{\cos\beta} \approx 41.67\unit{(mm)} $\\
	$ d_2 = d_{w2} = \dfrac{mz_2}{\cos\beta} \approx 208.33 \unit{(mm)} $\\
	$ d_{a1} = d_1 + 2(1+x_1)m \approx 45.57\unit{(mm)}$\\
	$ d_{a2} = d_2 + 2(1+x_2)m \approx 210.43\unit{(mm)}$\\
	$ d_{f1} = d_1 - (2.5-2x_1)m \approx 38.82\unit{(mm)}$\\
	$ d_{f2} = d_2 - (2.5-2x_2)m \approx 203.68\unit{(mm)}$\\
\end{tabular}
\begin{tabular}[t]{p{8cm}}
	$ d_{b1} = d_1\cos\alpha \approx 39.15\unit{(mm)}$\\
	$ d_{b2} = d_2\cos\alpha \approx 195.77 \unit{(mm)}$\\
	$ \alpha_t = \alpha_{tw} = \arctan\dfrac{\tan\alpha}{\cos\beta_w} \approx 20.53^\circ $\\
	$ v = \dfrac{\pi d_1n_{sh1}}{6\times10^4} \approx 6.39\unit{(m/s)}$
\end{tabular}

\subsection{Find $ [{\sigma}_{Hw}] $, $ [\sigma_{Fw1}] $ and $ [\sigma_{Fw2}] $}
In this section, we will try to approximate these parameters based on the factors $ Z_R$, $Z_V$, $K_{xH} $ and $ Y_R$, $Y_s$, $K_{xF} $ to substitute to equation (6.1) and (6.2):
\[
\begin{array}{l@{{} = {}}l}
[\sigma_{Hw}] & [{\sigma}_H]Z_RZ_VK_{xH}\\

[\sigma_{Fw}] & [{\sigma}_F]Y_RY_SK_{xF}\\
\end{array}
\]
Assuming smooth surface condition, $ Z_R = 1 $.\\
$ Z_V = 0.85v^{0.1} \approx 1.02$ with $ H\leq350 $.\\
In case of $ v>5\unit{(m/s)} $, $ K_{xH} = 1$.\\
The pair of gears are properly polished, which makes $ Y_R=1.1 $\\
$ Y_s = 1.08-0.0695\ln(m) \approx 1.05 $\\
Since $ d_{a1},d_{a2}\leq400\unit{(mm)} $, $ K_{xF}=1 $, which leads to:\\
$ [{\sigma}_{Hw}] = 520.93\unit{(MPa)}$\\
$ [{\sigma}_{Fw1}] = 297.51\unit{(MPa)}$\\
$ [{\sigma}_{Fw2}] = 285.61\unit{(MPa)}$

\subsection{Contact stress analysis}
From section 6.3.3. in the text, contact stress applied on a gear surface must satisfy the condition below:
\[
\sigma_H = z_Mz_Hz_\varepsilon\sqrt{2T_{sh1}K_H\dfrac{u_{hg}+1}{bu_{hg}d_{w1}^2}} \leq [{\sigma}_{Hw}]
\]

\paragraph{Find $ z_M $}
$ z_M = 274 $, according to table (6.5) 
\paragraph{Find $ z_H $}
$ \beta_b = \arctan\left( \cos\alpha_t\tan\beta_w\right) \approx 12.76^\circ \Rightarrow z_H = \sqrt{2\dfrac{\cos\beta_{b}}{\sin(2\alpha_{tw})}} \approx 1.72$
\paragraph{Find $ z_\varepsilon $} Obtaining $ z_\varepsilon $ through calculations:\\
$ \varepsilon_\alpha = \dfrac{\sqrt{d_{a1}^2-d_{b1}^2}+\sqrt{d_{a2}^2-d_{b2}^2}-2a_w\sin\alpha_{tw}}{2\pi m\dfrac{\cos\alpha_t}{\cos\beta_w}} \approx 1.41$\\
$ \varepsilon_\beta = b\dfrac{\sin\beta_w}{m\pi} \approx 3.12>1 \Rightarrow z_\varepsilon = \varepsilon_\alpha^{-0.5} \approx 0.86 $
\paragraph{Find $ K_H $} We find $ K_H $ using equation $ K_H = K_{H\beta}K_{H\alpha}K_{Hv} $\\
From table (6.13), $ v\leq 10 \unit{(m/s)}\Rightarrow \text{AG} = 8 $ \\
From table (P2.3), using interpolation, we approximate:\\ $ K_{Hv} \approx1.07$, $ K_{Fv} \approx1.18$\\
From table (6.14), using interpolation, we approximate:\\ $ K_{H\alpha} \approx1.1$, $ K_{F\alpha} \approx1.29 $ \\	
$ \Rightarrow K_H \approx 1.3 $
\paragraph{Find $ \sigma_H $} After calculating $ z_M $, $ z_H $, $ z_\varepsilon $, $ K_H $, we get the following result:
\[\sigma_H \approx 477.51 \unit{(MPa)}\leq [{\sigma}_{Hw}] \approx 509.09 \unit{(MPa)}\] 

\subsection{Bending stress analysis}
For safety reasons, the following conditions must be met:
\[
\begin{array}{l@{{} \leq {}}l}
\sigma_{F1} = 2\dfrac{T_{sh1}K_FY_\varepsilon Y_\beta Y_{F1}}{bd_{w1}m_n} & [\sigma_{Fw1}]\\ 
\sigma_{F2} = \dfrac{\sigma_{F1}Y_{F2}}{Y_{F1}} & [\sigma_{Fw2}]
\end{array}
\]

\paragraph{Find $ Y_\varepsilon $} Knowing that $ \varepsilon_\alpha \approx 1.41 $, we can calculate $ Y_\varepsilon = \varepsilon_\alpha^{-1} \approx 0.71 $
\paragraph{Find $ Y_\beta $} $ Y_\beta = 1-\dfrac{\beta_w}{140}\approx0.9$
\paragraph{Find $ Y_F $} Using formula $ z_v = z\cos^{-3}(\beta_w) $ and table (6.18):\\
$ z_{v1} = z_1\cos^{-3}(\beta_w) \approx 29.4 \Rightarrow Y_{F1} \approx 3.54$\\
$ z_{v2} = z_2\cos^{-3}(\beta_w) \approx 147.01 \Rightarrow Y_{F2} \approx 3.63 $
\paragraph{Find $ K_F $}
Using $ K_{F\beta} $, $ K_{F\alpha} $, $ K_{Fv} $ calculated from the sections above, we derive:\\ $ K_F = K_{F\beta}K_{F\alpha}K_{Fv} \approx 1.91 $

\paragraph{Find $ \sigma_F $} Since $ m_n = m\cos\beta_w \approx 1.46$, substituting all the values, we find out that:
\begin{align*}
\sigma_{F1} \approx 114.11 \unit{(MPa)} & \leq [\sigma_{Fw1}]\approx 297.51 \unit{(MPa)}\\
\sigma_{F2} \approx 117.01 \unit{(MPa)} & \leq [\sigma_{Fw2}]\approx 285.61 \unit{(MPa)}
\end{align*}

\subsection{Force on shafts}
$ F_t = \dfrac{2T_{sh1}}{d_{w1}} \approx 2402.28 \unit{(N)}$\\
$ F_r = F_t\tan\alpha_{tw} \approx 899.55 \unit{(N)}$\\
$ F_a = F_t\tan\beta_w\approx 580.75 \unit{(N)}$\\
In summary, we have the following table:
\begin{table}[ht]
	\centering
	\begin{tabular}[t]{|
			>{\columncolor[HTML]{C0C0C0}}l |p{2.5cm}|p{2.5cm}|}
		\hline
		& \multicolumn{1}{c|}{\cellcolor[HTML]{C0C0C0}pinion} & \multicolumn{1}{c|}{\cellcolor[HTML]{C0C0C0}driving gear} \\ \hline
		$ H\unit{(HB)} $              & 250                      & 240    \\ \hline
		$ [\sigma_F]\unit{(MPa)} $    & 257.14                   & 246.86 \\ \hline
		$ [{\sigma}_H]\unit{(MPa)} $    & \multicolumn{2}{l|}{\hskip2cm 509.09}       \\ \hline
		$ [\sigma_H]_{max}\unit{(MPa)} $    & \multicolumn{2}{l|}{\hskip2cm 1540}       \\ \hline
		$ [\sigma_F]_{max}\unit{(MPa)} $    & \multicolumn{2}{l|}{\hskip2cm 440}       \\ \hline
		$ a_w\unit{(mm)} $            & \multicolumn{2}{l|}{\hskip2cm 100}           \\ \hline
		$ b\unit{(mm)} $            & \multicolumn{2}{l|}{\hskip2cm 50}           \\ \hline
		$ m\unit{(mm)} $              & \multicolumn{2}{l|}{\hskip2cm 1.5}    \\ \hline
		$ d_w\unit{(mm)} $              & 33.33                    & 166.67 \\ \hline
		$ d_a\unit{(mm)} $            & 37.23                    & 168.77 \\ \hline
		$ d_f\unit{(mm)} $            & 30.48                    & 162.02 \\ \hline
		$ d_b\unit{(mm)} $            & 31.32                     & 156.62 \\ \hline
		$ u_{hg} $              & \multicolumn{2}{l|}{\hskip2cm 5}    \\ \hline
		$ v\unit{(m/s)} $              & \multicolumn{2}{l|}{\hskip2cm 5}    \\ \hline
		$ x\unit{(mm)} $                       & 0.3                       & -0.3     \\ \hline
		$ z $                       & 21                       & 105     \\ \hline
		$ \alpha_{tw}\unit{(^\circ)} $ & \multicolumn{2}{l|}{\hskip2cm 20.65}        \\ \hline
		$ \beta_w\unit{(^\circ)} $ & \multicolumn{2}{l|}{\hskip2cm 19.09}        \\ \hline
	\end{tabular}
	\caption{Gearbox specifications}
\end{table}
\newpage \lipsum[1-10]