\subsection{Experimental Results}

\subsubsection{Reaction order with respect to \ch{Na2S2O3}}

\begin{table}[ht]
	\centering
	{\renewcommand{\arraystretch}{1.75}
	\begin{tabular}{|l|p{1.5cm}|p{1.5cm}|p{2cm}|p{2cm}|p{2cm}|p{2cm}|}
		\hline
		\multirow{2}{*}{No.} & \multicolumn{2}{l|}{Initial concentration (M)}              & \multicolumn{1}{c|}{\multirow{2}{*}{$\Delta t_1$}} & \multicolumn{1}{c|}{\multirow{2}{*}{$\Delta t_2$}} & \multicolumn{1}{c|}{\multirow{2}{*}{$\Delta t_3$}} & \multicolumn{1}{c|}{\multirow{2}{*}{$\Delta \overline{t}$}} \\ \cline{2-3}
		& \ch{Na2S2O3} & \ch{H2SO4} &                               &                               & \multicolumn{1}{c|}{}                              &                                        \\ \hline
		1                    & $0.4\cdot10^{-3}$                        & $3.2\cdot10^{-3}$                         &                               &                               &                                                    &                                        \\ \hline
		2                    & $0.8\cdot10^{-3}$                         & $3.2\cdot10^{-3}$                         &                               &                               &                                                    &                                        \\ \hline
		3                    & $1.6\cdot10^{-3}$                          & $3.2\cdot10^{-3}$                         &                               &                               &                                                    &                                        \\ \hline
		\end{tabular}}
\end{table}
From $\Delta \overline{t}$ of experiments 1 and 2, determine $m_1$ and $m_2$ (sample calculation).

\[m_1 = \log_2 \frac{\Delta \overline{t_1}}{\Delta \overline{t_2}}= \hspace{7cm}\]

\[m_2 = \log_2 \frac{\Delta \overline{t_2}}{\Delta \overline{t_3}}= \hspace{7cm}\]

Reaction order with respect to \ch{Na2S2O3} $\displaystyle = \frac{m_1+m_2}{2} =$

\subsubsection{Reaction order with respect to \ch{H2SO4}}
\begin{table}[ht]
	\centering
	{\renewcommand{\arraystretch}{1.75}
		\begin{tabular}{|l|p{1.5cm}|p{1.5cm}|p{2cm}|p{2cm}|p{2cm}|p{2cm}|}
			\hline
			\multirow{2}{*}{No.} & \multicolumn{2}{l|}{Initial concentration (M)}              & \multicolumn{1}{c|}{\multirow{2}{*}{$\Delta t_1$}} & \multicolumn{1}{c|}{\multirow{2}{*}{$\Delta t_2$}} & \multicolumn{1}{c|}{\multirow{2}{*}{$\Delta t_3$}} & \multicolumn{1}{c|}{\multirow{2}{*}{$\Delta \overline{t}$}} \\ \cline{2-3}
			& \ch{Na2S2O3} & \ch{H2SO4} &                               &                               & \multicolumn{1}{c|}{}                              &                                        \\ \hline
			1                    & $0.8\cdot10^{-3}$                        & $1.6\cdot10^{-3}$                         &                               &                               &                                                    &                                        \\ \hline
			2                    & $0.8\cdot10^{-3}$                         & $3.2\cdot10^{-3}$                         &                               &                               &                                                    &                                        \\ \hline
			3                    & $0.8\cdot10^{-3}$                          & $6.4\cdot10^{-3}$                         &                               &                               &                                                    &                                        \\ \hline
	\end{tabular}}
\end{table}


From $\Delta \overline{t}$ of experiments 1 and 2, determine $m_1$ and $m_2$ (sample calculation).

\[m_1 = \log_2 \frac{\Delta \overline{t_1}}{\Delta \overline{t_2}}= \hspace{7cm}\]

\[m_2 = \log_2 \frac{\Delta \overline{t_2}}{\Delta \overline{t_3}}= \hspace{7cm}\]

Reaction order with respect to \ch{H2SO4} $\displaystyle = \frac{m_1+m_2}{2} =$

\subsection{Answer the questions}

\begin{enumerate}
	\item In the experiment above, what is the effect of the concentrations of \ch{Na2S2O3} and \ch{H2SO4} pm the reaction rate? Rewrite the reaction rate expression, determine, the order of reactions.
	
	- The concentration of \ch{Na2S2O3} is directly proportional to reaction rate.
	
	- The concentration of \ch{H2SO4} almost does not affect to reaction rate.
	
	- Expression for reaction rate: \[v=k[\ch{Na2S2O3}]^m[\ch{H2SO4}]^n\]
	where:
	\begin{itemize}[label=]
		\item $k$: constant coefficient
		\item $m$: reaction order with respect to \ch{Na2S2O3}
		\item $n$: reaction order with respect to \ch{H2SO4}
	\end{itemize}
	Overall order of reaction: $m+n=$
	
	\item The mechanism of reactions can be rewritten as follows:
	
	\ch{H2SO4 + Na2S2O3 -> Na2SO4 + H2S2O3} (1)
	
	\ch{H2S2O3 -> H2SO3 + S v} (2)
	
	Based on the experimental results, is it plausible to conclude that the reaction (1) or (2) is the rate-determining step, which is the slowest step of the reaction? Recall that in the experiments, the amount of \ch{H2SO4} is always abundant.
	
	- Reaction (1) is ion-exchange reaction. Thus, the reaction rate is high.
	
	- Reaction (2) is reduction/oxidation reaction. Thus, the reaction rate appears to be slower than reaction (1).
	
	$\rightarrow$ Reaction (2) is the rate-determining step and is the slowest step of the reaction since the overall order of the reaction is the order of reaction (2).
	
	\item Based on the principle of experimental method, decide if the reaction rate is instantaneous or average.\\
	The reaction rate is considered to be instantaneous since the reaction rate is defined by $\frac{\Delta C}{\Delta t}$ with $\Delta C \approx 0$. This can be explained with consistency of sulfur's concentration throughout the experiment.
	
	\item Reverse the order of \ch{H2SO4} and \ch{Na2S2O3}. Does the reaction order change? Explain.\\
	The reaction rate does not change because the reaction order depends on the properties of reactions (temperature, pressure, concentration, area of contact, etc.). It is independent of experimenting orders.
\end{enumerate}