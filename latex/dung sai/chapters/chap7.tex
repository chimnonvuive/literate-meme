\chapter{ĐO CHIỀU DÀI PHÁP TIẾP TUYẾN CHUNG}

\section{Mục đích thí nghiệm}
\begin{itemize}
	\item Biết sử dụng pan me chuyên dùng để đo chiều dài pháp tuyến chung.
	\item Biết cách xác định chiều dài pháp tuyến chung.
\end{itemize}

\section{Các bước tiến hành}
Bánh răng được chọn có số hiệu 65, mođun 2.5. Theo công thức tính chiều dài pháp tuyến chung,
\[
\begin{array}{l}
L = m\cos \alpha[(n-0.5)\pi + Z\theta + 2\xi \tan \alpha]\\
= 2.5\cos 20^\circ [(7-0.5)\pi + 60 \arctan20^\circ]=50.07\unit{mm}
\end{array}
\]
\begin{table}[ht]
	\centering
	\caption{Bảng đo chiều dài pháp tuyến chung $ L' $}
	\begin{tabular}{lllllll}\toprule
		Số hiệu & 1 & 2 & 3 & 4 & 5 & Trung bình \\\midrule
		65 & 50.15 & 50.41 & 50.35 & 50.35 & 50.41 & 50.334\\\bottomrule
	\end{tabular}
\end{table}
Sai lệch giữa chiều dài lý thuyết và thực tế đo được là $ L - L'= -0.264\unit{mm} $

\section{Đánh giá kết quả}
Nhận xét: sai lệch giữa thực tế và lý thuyết không nhiều (trong khoảng chấp nhận được $ 0.5\% $). Nguyên nhân sai lệch có thể do dụng cụ đo và cách đo của sinh viên trong quá trình thí nghiệm.
