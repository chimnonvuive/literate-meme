\chapter{Pressure and Fluid Statics}% TODO: \usepackage{graphicx} required
\section{Formulas}
	Calculating magnitude of resultant force $ F_R $:
	\[F_R = (P_0+\rho g y_C\sin\theta)A=P_CA\]
	\begin{tabular}{lp{\linewidth-1.5cm}}
		where & $ P_0 $ is the pressure at the liquid surface (often is $ P_{atm} $, which can be ignored)\\
		&$ P_C $ is the pressure taken at the centroid of the rigid body surface\\
		& $ h_c =y_C\sin\theta$ is the vertical distance of the rigid body's centroid from the liquid surface\\
		& $ \theta $ is the angle of the rigid body with respect to the liquid surface
	\end{tabular}
	For horizontal plate:
	\[F_R = (P_0+\rho g h)ab\]
	For vertical plane ($ s $ is the upper vertical distance, $ b $ is the length of the plate):
	\[F_R = (P_0+\rho g (s+b/2))ab\]
	The location at which $ F_R $ acts on the body surface is:
	\[y_P=y_C+\dfrac{I_{xx,C}}{[y_C+P_0/(\rho g\sin\theta)]A}\]
	 \begin{tabular}{lp{\linewidth-1.5cm}}
	 	where & $ y_C $ is the centroid of the rigid body\\
	 	& $ I_{xx,C}  $ is the second moment of area about the $ x $-axis passing through the centroid of the rigid body (normal to the surface itself)
	 \end{tabular}
	 For rectangular plates, it can also be written as:
	 \[y_P = s+\dfrac{b}{2}+\dfrac{b^2}{12[s+b/2+P_0/(\rho g \sin\theta)]}\]
	  The vertical distance from liquid surface $ h_P = y_P\sin\theta$
\section{Problems}
\subsubsection{Q2.1}
A differential manometer consists of a U-shaped pipe of diameter $ d $, connecting two cylinders of diameter $ D $, the  instrument being filled with two insoluble liquids of specific weight $ \gamma_1 $ and $ \gamma_2  $. When the pressure difference, $ \Delta p = p_1-p_2 =0$, the interface between two liquids is at position $ 0 $ on the scale.
\begin{figure}[h]
	\centering
	\includegraphics[width=0.4\linewidth]{"2020-08-13 17.00.35 drive.google.com c4c67da5f09f"}
	\caption{Differential manometer}
	\label{fig:2020-08-13-17}
\end{figure}
\begin{enumerate}
	\item Determine the relationship between $ \Delta p $ and the displacement of the interface between the two fluids, $ h $. Given $ d=5\unit{mm},D=50\unit{mm}, \gamma_1=8530\unit{N/m^3},\gamma_2=8140\unit{N/m^3}, h=280\unit{mm}$.
	\item With the given $ \Delta p $, how many times $ h $ will decrease, if $ d=D=5\unit{mm} $.
\end{enumerate}
\paragraph{Ans:}$  $\\
Let $ h_1,h_2 $ be the height of the liquid from $ O $ on the left and right pipe, respectively.\\
The relation between $ P_1 $ and $ P_2 $ is described according to the formula:
\[\dfrac{D^2}{d^2}p_1=p_2\dfrac{D^2}{d^2}+\gamma_2(h_2-h)-\gamma_1(h_1-h)\]
Rearranging the equation yields:
\[h=\dfrac{\dfrac{D^2}{d^2}\Delta p +\gamma_1h_1-\gamma_2h_2}{\gamma_1-\gamma_2}\]

\subsubsection{Q2.2} A rectangular valve length $ b $ rotates about the horizontal axis at point $ A $. Neglecting valve thickness, determine the minimum weight $ G $ of the gate based on the parameters $  h1, h2, h3, \rho, b $ and $ g $ such that the system is balanced. Use Figure \ref{fig:2} to solve the problem.
\begin{figure}[h]
	\centering
	\includegraphics[width=0.4\linewidth]{"2020-08-13 17.45.37 drive.google.com bc14c8504f54"}
	\caption{Rotating rectangular valve}
	\label{fig:2}
\end{figure}
\paragraph{Ans:}$  $\\
Weight of the triangular block of water: $ W=\rho g V=\dfrac{1}{2\sqrt{3}}\rho g h_3^2 $\\
The horizontal force acting on vertical plane is: $ F_x = \rho g\left(h_2+\dfrac{h_3}{2}\right)h_3 $\\
The vertical force acting on horizontal plane is: $ F_y = \rho g(h_2+h_3)\dfrac{h_3}{\sqrt{3}} $\\
Projecting the forces onto $ x,y $ axes yields:
\[\begin{array}{ll}
F_H=&F_x=\dfrac{g h_{3} \rho \left(2 h_{2} + h_{3}\right)}{2}
\\
F_V=&F_y-W=\dfrac{\sqrt{3} g h_{3} \rho \left(2 h_{2} + h_{3}\right)}{6}

\end{array}\]
Thus, \\
$ F_R=\sqrt{F_H^2+F_V^2}=\sqrt{F_x^2+(F_y-W)^2}=\dfrac{1}{\sqrt{3}} gh_{3} \rho(2h_2+h_3)
$\\
$ \tan\theta=\dfrac{F_V}{F_H}=\dfrac{\sqrt{3}}{3}\Rightarrow\theta=30^\circ $\\
$ s = \dfrac{h_2}{\sqrt{3}} ,y_C = h_1$\\
$ y_P=y_C+\dfrac{I_{xx,C}}{[y_C+P_0/(\rho g\sin\theta)]A} = \dfrac{\sqrt{3} h_{3} \left(3 h_{2} + 4 h_{3}\right)}{9 \left(h_{2} + h_{3}\right)}
$\\
The system is at equilibrium when and only when:
\[ M_{A\circlearrowleft} = M_{A\circlearrowright}\]
$ \Rightarrow \dfrac{b}{2}G\cos60^\circ = y_PF_R\\\Rightarrow G = \dfrac{2}{\sqrt{3}} gh_{3} \rho(2h_2+h_3) $

\subsubsection{Q2.3}
An empty cylindrical jar of diameter $ d = 5 \unit{cm}=0.05\unit{m}$, length $L = 10 \unit{cm} =0.1\unit{m}$ is placed in the water. Determine the weight of the jar so that it reaches equilibrium below the depth $ h = 1\unit{m} $. Ignore the thickness of the wall of jar. Given $ p_a = p_{\textit{water }@10\unit{m}} = 98.1\times10^3\unitp{Pa}$.
\begin{figure}[h]
	\centering
	\includegraphics[width=0.4\linewidth]{"2020-08-15 15.00.04 drive.google.com a5d89daba201"}
	\caption{Jar under water}
	\label{fig:3}
\end{figure}
\paragraph{Ans:}$  $\\
The center of gravity of the jar is located at:\\
\[h_C = h+\dfrac{L}{2} = 1.05\unitp{m}\]
The weight of the jar:\\
\[W = F_B = \rho gh_C \dfrac{\pi d^2}{4} = 20.225\unitp{N}\]

\subsubsection{Q2.4}
A rectangular valve AB is inclined to the horizontal plane an angle $\alpha$, having width $ b $, the depths of A and B are $ h_2 $ and $ h_3 $ respectively, the pressure on the water surface in the tank is $ p_o $. The water level in the manometer tube is higher than the water level in the jar, $ h_1 $ (see Figure \ref{fig:4}).
Let $ b = 4\unit{m}, h_1 = 2 \unit{m}, h_2 = 1 \unit{m}, h_3 = 3 \unit{m},  \alpha= 45^\circ,\rho=1000\unit{kg/m^3},g=9.81\unit{m/s^2} $.
\begin{figure}[h]
	\centering
	\includegraphics[width=0.4\linewidth]{"2020-08-15 16.48.02 drive.google.com 0b8dc5aedc2f"}
	\caption{Manometer with jar}
	\label{fig:4}
\end{figure}
\begin{enumerate}
	\item Compute gauge pressure $ p_o, p_A, p_B $.
	\item Compute the force by water acting on valve $ AB $.
	\item Determine the center of pressure $ D $ (compute $ BD $).
	\item Compute the minimum force required $ F $, acting at $ B $ to remain the valve closed.
\end{enumerate}
\paragraph{Ans:}$  $\\
\begin{enumerate}
	\item $ p_o = \rho g h_1\Rightarrow p_o=19.62\unitp{kPa} $\\
	$ p_A = \rho g (h_1+h_2)\Rightarrow p_A=29.43\unitp{kPa}$\\
	$ p_B = \rho g (h_1+h_3)\Rightarrow p_B=49.05\unitp{kPa}$
	\item Weight of the triangular block of water:\\
	$ W=\rho g (h_3-h_2)^2b= 156.96\unitp{kN}$\\
	The horizontal force acting on vertical plane is:\\
	$ F_x = \rho g (h_1+\dfrac{h_3-h_2}{2})(h_3-h_2)b =235.44\unitp{kN}$\\
	The vertical force acting on horizontal plane is:\\
	$ F_y = \rho g (h_1+h_2)(h_3-h_2)b=235.44\unitp{kN} $\\
	Projecting the forces onto $ x,y $ axes yields:
	\[\begin{array}{l}
	F_H=F_x=235.44\unitp{kN}
	\\
	F_V=F_y+W=392.4\unitp{kN}
	\end{array}\]
	Thus, \\
	$ F_R=\sqrt{F_H^2+F_V^2}=457.61\unitp{kN}	$\\
	$ \tan\theta=\dfrac{F_V}{F_H}=1.667\Rightarrow\theta=59.04^\circ $
	\item $ BD = \dfrac{h_3-h_2}{\cos\alpha} =2.83\unitp{m} $
\end{enumerate}

