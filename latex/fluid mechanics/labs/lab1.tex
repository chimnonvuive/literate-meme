\subsection{Questions}
\begin{enumerate}
	\item What should we do to check the standard of the measure tools?\\
	To check the standard face of the measure tools are horizontally yet by checking the liquid level in pipe 3 and pipe 10. Their liquid level must be equal.
	\item To conduct experiments with the fluid statics experiment set, which data we have to measure?
	\begin{itemize}[label=-]
		\item Measure the relative position of bottle A to bottle T.
		\item Measure the height of liquid in pressure gauge from 1 to 10.
		\item Pressure and Temperature in the laboratory.
	\end{itemize}
	\item How many cases that fluid statics experiment need to conduct?\\
	There are 3 cases:
	\begin{enumerate}
		\item $Z_D - Z_T = 15 \div 20cm$
		\item $Z_D - Z_T = 5 \div 7cm$
		\item $Z_D - Z_T = -20 \div -15cm$
	\end{enumerate}
	\item How can we change the air pressure in bottle T between each measurements?\\
	By the crank 11 we can change the height of bottle D, which result in the change of pressure in bottle T.
	\item Why we have to measure the room pressure and temperature?
	\begin{itemize}[label=-]
		\item To calculate the air pressure in bottle T.
		\item To calculate specific gravity of the liquids in the bottle with the environment temperature.
	\end{itemize}
\end{enumerate}

\subsection{Measurements}
\begin{enumerate}
	\item Atmospheric pressure and air temperature:\\
	$P_a=755 mmHg$\\ $t\degree = 32 \degree C$\\
	Water specific gravity:\\
	$\gamma_{H_2O}=9760.95 N/m^3$
	\item With respect to the three relative positions of the tank D and the tank T, record the measured values of nine tubes and the group of tubes 2 in Table 1. \\
	\begin{table}[ht]
		\centering
		\begin{tabular}{|c|c|c|c|c|c|c|c|c|c|c|}
			\hline
			No & $L_1$ & $L_3$ & $L_4$ & $L_5$ & $L_6$ & $L_7$ & $L_8$ & $L_9$ & $L_{10}$ & Note\\ \hline
			1 & 22.6 & 36.9 & 13 & 27.2 & 16.1 & 32.7 & 13.9 & 15 & 36.9 & $D-T=15cm$ \\ \hline
			2 & 22.6 & 27.2 & 17.9 & 22.3 & 22 & 27.1 & 14.3 & 14.6 & 27.2 & $D-T=5cm$ \\ \hline
			3 & 22.5 & 7.6 & 27.7 & 12.6 & 33.1 & 15.7 & 15 & 14 & 17.6 & $D-T=-15cm$ \\ \hline
		\end{tabular}
		\caption{Measured results (Unit: cm)}
		\label{tab1lab1}
	\end{table}\\
	\begin{table}[ht]
		\centering
		\begin{tabular}{|c|c|c|c|c|}
			\hline
			No & $L_{21}$ & $L_{22}$ & $L_{23}$ & Note \\ \hline
			1 & 38.5 & 37.6 & 37.3 & $D-T=15cm$ \\ \hline
			2 & 30.4 & 28.5 & 27.8 & $D-T=5cm$ \\ \hline
			3 & 10.7 & 8.9 & 8.3 & $D-T=-15cm$ \\ \hline
		\end{tabular}
		\caption{Measurements in group of tubes 2 (Unit: cm)}
		\label{tab2lab1}
	\end{table}
\end{enumerate}

\subsection{Calculations and Report}
\begin{enumerate}
	\item Which tubes or tanks have equal water levels in the hydrostatic experiment set? Why?\\
	\begin{itemize}[label=-]
		\item Water levels in tube 3, tube 10 and tank D are equal (tubes 10 and 3 are used to observe the isobaric surface so these have the same pressure).
		\item Water levels in tube 1 and tank T are equal.
	\end{itemize}
	\item Which tubes have water levels that do not according to hydrostatic law in the hydrostatics experiment set? Why?
	\begin{itemize}[label=-]
		\item Water levels in tube $L_{21}$, $L_{22}$, $L_{23}$ don’t according to hydrostatic law because the diameter of those tubes are too small ($\leq3mm$), those tubes are used to observe capillary phenomena.
	\end{itemize}
	\item Calculate the absolute pressure, gauge pressure of the gas in the tank T and the relative error of this pressure in the measured cases. Fill the results into Table \ref{tab3lab1}.
	\item Calculate the specific gravity of three fluids in pairs of tubes: 4-5, 6-7, 8-9 and the relative error of these specific gravities for the cases. Fill the results into Table \ref{tab3lab1}.
\end{enumerate}
\begin{table}[ht]
	\centering
	{\renewcommand{\arraystretch}{1.5}
	\begin{tabular}{|c|c|c|c|c|c|c|c|c|c|}
		\hline
		\multirow{2}{*}{No.} & $p_t$ & $p_d$ & $\delta_\rho$ & $\gamma_{4-5}$ & $\gamma_{6-7}$ & $\gamma_{8-9}$ & $\delta_{\gamma4-5}$ & $\delta_{\gamma6-7}$ & $\delta_{\gamma8-9}$ \\ \cline{2-10}
		& \multicolumn{2}{c|}{$10^3N/m^2$} & $\%$ & \multicolumn{3}{c|}{$10^3N/m^3$} & \multicolumn{3}{c|}{$\%$} \\ \hline
		1 & 102.06 & 1.4 & 0.82 & 9.83 & 8.41 & 126.89 & 1.52 & 1.42 & 9.91 \\ \hline
		2 & 101.12 & 0.45 & 2.29 & 10.2 & 8.8 & 149.67 & 4.57 & 4.25 & 35.63 \\ \hline
		3 & 99.21 & -1.45 & 0.79 & 9.63 & 8.36 & 145.44 & 1.45 & 1.37 & 10.79 \\ \hline
	\end{tabular}}
	\caption{Results}
	\label{tab3lab1}
\end{table}
\subsection{Conclusion}
\begin{enumerate}[label=\alph*)]
	\item Pressure in closed tank.\\
	In 3 cases, case $D – T = 15cm$ and  $D – T = 5cm$, the pressure in tank D is greater than that in tank T.
	\item Changes of heights between 2 tanks.
	\begin{itemize}[label=-]
		\item When changing of height between 2 tanks by turning pulley handle, it changes the pressure inside each tank.
		\item From that, we can understand better about capillary phenomena.
	\end{itemize}
\end{enumerate}