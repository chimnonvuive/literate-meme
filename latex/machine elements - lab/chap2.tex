\chapter{Determination of Tensile Force of Bolts}
\section{Nomenclature}
\begin{tabular}[t]{lp{8cm}}
		$ [F_{cb}] $ & tension force at failure of common bolt, $ N $\\
		$ [F_{sb}] $ & tension force at failure of steel bolt, $ N $\\
		$ [\sigma_{cb}] $ & tension at failure of common bolt, $ MPa $\\
		$ [\sigma_{sb}] $ & tension at failure of steel bolt, $ MPa $\\
		$ d $ & nominal diameter of M8 bolt, $ mm $\\
		$ F_c $ & tension force of hydraulic cylinder\\
		$ F_{cb} $ & tension force at failure of common bolt, $ N $\\
		$ F_{sb} $ & tension force at failure of steel bolt, $ N $\\
\end{tabular}

\section{Aim}
\begin{enumerate}
	\item Help students understand more clearly about tensile force of some types of steels, the relation between central $ M_k $ and localized strain of material.
	\item Help students approach methods, measuring devices in determining tensile force.
\end{enumerate}

\section{Safety Procedures}
\begin{enumerate}
	\item Safeguard is compulsory when straining the bolt.
	\item Close the machine gate when operating.
\end{enumerate}

\section{Experimental Report}
\begin{table}[ht]
	\centering
	\renewcommand{\arraystretch}{1.5}
	\rowcolors{3}{}{lightgray!20}
	
	\begin{tabular}{cR{2.75cm}R{2.75cm}}\toprule
		\multirow{2}{*}{No.} & \multicolumn{2}{c}{Experiment with $d=8\unit{mm}$} \\ \cmidrule{2-3}
		& $F_{sb}$         & $F_{cb}$               \\ \midrule
		1                    & 33898            & 37377                           \\
		2                    & 33574            & 37053                           \\
		3                    & 34211            & 36426                           \\
		4                    & 33727            & 37053                           \\
		5                    & 34211            & 36426                           \\
		Avg              & 33323.4          & 36867                           \\ \bottomrule
	\end{tabular}
	\caption{Tension force at failure of common and steel bolts}
\end{table}
\begin{figure}
	\centering
	\includegraphics[width=150mm]{Exp2cb.png}
	\caption{Tension force at failure of common bolt}
\end{figure}
\begin{figure}
	\centering
	\includegraphics[width=150mm]{Exp2sb.png}
	\caption{Tension force at failure of steel bolt}
\end{figure}
\section{Discussion and conclusions}
Based on the data and graphs, it can be concluded that steel bolts are more durable than common bolts.\\
Through 5 experiments conducted on each type of bolt, we saw that in spite of being the same type, each experiment gave different results, which could be explained from minor dissimilarities in shape and internal structure as a consequence of:
\begin{itemize}
	\item Errors during the designing and machining.
	\item Environmental effects from transportation, preservation, climate, etc.
	\item Difference in microscopic level due to manufacturing: defects, varying temperature, mold structure, cooling rate, etc.
	\item Unreliable testing.
\end{itemize}
In summary, we can see that the fracture point of each individual bolt is different but not too much. This feat is achieved by modern and increasingly accurate manufacturing techniques.