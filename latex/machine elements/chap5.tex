\chapter{Bearing Design}
\section{Nomenclature}
\begin{tabular}[t]{lp{6.5cm}}
	$ [s] $ & permissible safety factor\\
	$ [\sigma] $ & permissible static strength, $ \unit{MPa} $\\
	$ [\tau] $ & permissible torsion, $ \unit{MPa} $\\
	$ a_w $ & shaft distance, $ \unit{mm} $\\
	$ b_O $ & rolling bearing width, $ \unit{mm}$\\
	$ C_d $ & basic dynamic load rating, $ \unit{N} $\\
	$ cb $ & role of gear on the shaft (active or passive)\\
	$ cq $ & rotational direction of the shaft\\
	$ d $ & base shaft diameter, $ \unit{mm} $\\
	$ d_w $ & gear diameter, $ \unit{mm} $\\
	$ F_a $ & axial force, $ \unit{N} $\\
	$ F_r $ & radial force, $ \unit{N} $\\
	$ F_t $ & tangential force, $ \unit{N} $\\
	$ F_x $ & applied force, $ \unit{N} $\\	
\end{tabular}
\begin{tabular}[t]{lp{6.5cm}}
	$ h_n $ & distance between bearing lid and bolt, $ \unit{mm} $\\
	$ hr $ & tooth direction\\
	$ K_x $ & surface tension concentration factor\\
	$ K_y $ & diminish factor\\
	$ K_\sigma $ & combined influence factor in tension\\
	$ K_\tau $ & combined influence factor in shear\\
\end{tabular}\newpage
\begin{tabular}[t]{lp{6.5cm}}
	$ k_d $ & temperature factor\\
	$ k_t $ & load condition factor\\
	$ k_\sigma $ & fatigue stress concentration factor in tension\\
	$ k_\tau $ & fatigue stress concentration factor in shear\\
	$ L $ & rated life in million revolutions, $ \unit{\text{million rev}} $\\
	$ L_h $ & rated life in hours, $ \unit{h} $\\
	$ l $ & length (general), $ \unit{mm} $\\
	$ l_m $ & hub length (general), $ \unit{mm} $\\
	$ M $ & moment, $ \unit{N\cdot mm} $\\
	$ M_e $ & equivalent moment, $ \unit{N\cdot mm} $\\
	$ M_{max} $ & maximum moment at the cross section, $ \unit{N\cdot mm} $\\
	$ m $ & load-life exponent\\
	$ Q $ & equivalent dynamic load, $ \unit{kN} $\\
	$ q $ & standardized coefficient of shaft diameter\\
	$ R $ & reaction force, $ \unit{N} $\\
	$ r $ & shoulder fillet radius, $ \unit{mm} $\\
	$ \bar{r} $ & position of applied force on the shaft, $\unit{mm}$\\
	$ S $ & length defined by table (6.1), $ \unit{mm} $\\
	$ s $ & calculated safety factor\\
	$ s_\sigma $ & safety factor in tensile stress\\
	$ s_\tau $ & safety factor in shear stress\\
	$ T $ & torque at the cross section, $ \unit{N\cdot mm} $\\
\end{tabular}
\begin{tabular}[t]{lp{6.5cm}}
	$ W $ & section modulus, $ \unit{mm^3} $\\
	$ W_O $ & polar section modulus, $ \unit{mm^3} $\\
	$ X $ & dynamic radial load factor\\
	$ Y $ & dynamic axial load factor\\
	$ \alpha $ & contact angle, $ ^\circ $\\
	$ \alpha_{tw} $ & traverse meshing angle, $ ^\circ $\\
	$ \beta $ & helix angle, $ ^\circ $\\
	$ \psi_\sigma $ & mean stress influence factor\\
	$ \psi_\tau $ & mean shear influence factor\\
	$ \sigma_{-1} $ & endurance limit at stress ratio of -1, $ \unit{MPa} $\\
	$ \sigma_a $ & tensile stress amplitude, $ \unit{MPa} $\\
	$ \sigma_b $ & ultimate strength, $ \unit{MPa} $\\
	$ \sigma_{ch} $ & yield limit, $ \unit{MPa} $\\
	$ \sigma_m $ & mean tensile stress, $ \unit{MPa} $\\
	$ \sigma_{td} $ & static strength, $ \unit{MPa} $\\
	$ \tau_{-1} $ & endurance limit at shear ratio of -1, $ \unit{MPa} $\\
	$ \tau_a $ & shear stress amplitude, $ \unit{MPa} $\\
	$ \tau_m $ & mean shear stress, $ \unit{MPa} $\\
	$ _{sh1} $ & subscript for shaft 1\\
	$ _{sh2} $ & subscript for shaft 2\\
	$ _x $ & subscript for x-axis\\
	$ _y $ & subscript for y-axis\\
	$ _z $ & subscript for z-axis\\
\end{tabular}

\section{Choose bearing type}
As for the types, we will examine $  \dfrac{F_a}{F_r}  $ at $ A_1 $, $ B_1 $, $ A_2 $ and $ B_2 $ in the 2 shafts from the previous chapter, where $ F_a $ is the output axial force $ |F_{z12}| = |F_{z21}| \approx 580.75 \unit{(N)} $; $ F_r $ is the magnitude of combined reaction force $ \sqrt{R_x^2+R_y^2} $ from the shaft onto the bearing, which is essentially a radial load. The larger ratio in a shaft will be our ratio of choice to determine the bearing type.\\
Taking our results from the chapter 4: \vskip2mm
{\centering
	\begin{tabular}[ht]{p{7cm}p{7cm}}
		$
		\left\{ 
		\begin{array}{l@{{} = {}}l}
		F_{rA1} & \sqrt{R_{A1x}^2+R_{A1y}^2}  \approx  1.2\unit{(kN)}\\
		F_{rB1} & \sqrt{R_{B1x}^2+R_{B1y}^2}  \approx  1.33\unit{(kN)}\\
		F_{aA1} & |F_{z12}|  \approx  0.58 \unit{(kN)}\\
		F_{aB1} & |F_{z12}|  \approx  0.58 \unit{(kN)}\\
		\end{array}
		\right.
		$ & $
		\left\{ 
		\begin{array}{l@{{} = {}}l}
		F_{rA2} & \sqrt{R_{A2x}^2+R_{A2y}^2}  \approx  3.79\unit{(kN)}\\
		F_{rB2} & \sqrt{R_{B2x}^2+R_{B2y}^2}  \approx  2.05\unit{(kN)}\\
		F_{aA2} & |F_{z21}|  \approx  0.58 \unit{(kN)}\\
		F_{aB2} & |F_{z21}|  \approx  0.58 \unit{(kN)}\\
		\end{array}
		\right.
		$
\end{tabular}}\vskip2mm


yields\vskip2mm
{\centering
	\begin{tabular}[t]{p{6cm}p{7cm}}
		$
		\left\{ 
		\begin{array}{l@{{} \approx {}}l}
		\dfrac{F_{aA1}}{F_{rA1}} & 0.46 \\
		\dfrac{F_{aB1}}{F_{rB1}} & 0.44 \\
		\end{array}
		\right.$
		& $
		\left\{ 
		\begin{array}{l@{{} \approx {}}l}
		\dfrac{F_{aA2}}{F_{rA2}} & 0.15 \\
		\dfrac{F_{aB2}}{F_{rB2}} & 0.28 \\
		\end{array}
		\right.$
\end{tabular}}\vskip2mm
Since $ 0.46>0.3 $ and $ 0.28\leq0.3 $, the pair of bearings on shaft 1 is single-row angular contact ball bearings with $ \alpha_{sh1} = 12^\circ$ and the remaining pair is single-row deep-groove bearings ($ \alpha_{sh2} = 0^\circ $); $ \text{AG} = 0 $ according to the recommendations on p.212 and p.213.\\
We also have dimensions at the cross sections $ A_1 $, $ B_1 $, $ A_2 $, $ B_2 $ from the previous chapter:\vskip2mm
{\centering
	\begin{tabular}[t]{p{6cm}p{7cm}}
		$
		\left\{ 
		\begin{array}{l@{{} = {}}l}
		b_{O1} & 21\unit{(mm)}\\
		d_{A1} & 35\unit{(mm)}\\
		d_{B1} & 35\unit{(mm)}\\
		\end{array}
		\right.$
		& $
		\left\{ 
		\begin{array}{l@{{} = {}}l}
		b_{O2} & 23\unit{(mm)}\\
		d_{A2} & 40\unit{(mm)}\\
		d_{B2} & 40\unit{(mm)}\\
		\end{array}
		\right.$
\end{tabular}}\vskip2mm
From these parameters, we will look up the tables at the end of the text. The pair of single-row angular contact ball bearings of choice is 46307, which is suitable for shaft 1 and $ C_{o1} = 25.2\unit{(kN)} $. On shaft 2, the pair of single-row deep-grove bearings are type 308, where $ C_{o2}=21.7\unit{(kN)} $.

\section{Bearing dimensions}
\subsection{Calculate basic dynamic load rating}
\[C_d = Q\sqrt[m]{L}\]
\subsubsection{Find equivalent dynamic load} Since we only use angular contact ball bearings, the following formula applies:
\[Q = (XVF_r+YF_{at})k_tk_d\]
Since the inner ring rotates, $ V = 1 $ and the $\dfrac{F_a}{VF_r}=\dfrac{F_a}{F_r}$, meaning that the ratios in the section above will be used to examine $X$ and $Y$.\\
The design problem also does not give any further information about operating temperature, which gives $ k_t = 1 $. In addition, we get $ k_d = 1 $ from table (11.3) based on the machine's condition (low load and power rating).
\paragraph{Find the ratio $ \dfrac{iF_a}{C_o} $} This ratio is calculated and applied for 2 shafts ($ i = 1 $ for single-row bearings in our case):\\
For shaft 1, $ \dfrac{F_a}{C_{o1}} \approx 0.027$\\
For shaft 2, $ \dfrac{F_a}{C_{o2}}  \approx  0.023$
\paragraph{Compare with $ e $}
From the previous section, $ \alpha_{sh1} = 12^\circ, \alpha_{sh2} = 0^\circ $. Inspecting table (11.4) and by interpolation, $ e_{sh1} \approx  0.33$, $e_{sh2}  \approx 0.22 $. These values are then compared to $ \left| \dfrac{F_a}{VF_r} \right| $ to look up the correct column.
\paragraph{Find $ X,Y $} Table (11.3) and interpolation are used in finding these values:\\
For shaft 1, $ \left| \dfrac{F_a}{VF_r} \right| = \left| \dfrac{F_{z12}}{VR_{A1y}} \right| \approx 0.46 > e_{1} \Rightarrow X_{1} = 0.56, Y_{1} = 2.1$.\\
For shaft 2, $ \left| \dfrac{F_a}{VF_r} \right| = \left| \dfrac{F_{z21}}{VR_{B2y}} \right| \approx 0.28 > e_{2} \Rightarrow X_{2}  \approx  0.45, Y_{2}  \approx  1.64$.
\paragraph{Find $ F_{at} $} For shaft 1, additional radial forces are also applied to the pair of angular contact ball bearings. From table (11.5), the first arrangement is used in the gearbox. Therefore, $ \mathbf{F_{sA1}} \upharpoonleft\upharpoonright \mathbf{F_{z12}} $ and $ \mathbf{F_{sB1}} \upharpoonleft\downharpoonright \mathbf{F_{z12}} $ (the direction of $ \mathbf{F_{z12}} $ can be found at Figure \ref{force on shaft}) Following the sign convention on p.218 and combining with equation (11.8), (11.10), (11.11a) and (11.11b):\\
At cross section $ A_1 $:\\
$ F_{sB1} = e_1\sqrt{R_{B1x}^2+R_{B1y}^2}  \approx 0.43\unit{(kN)}$\\
$ \displaystyle\sum F_{aA1} = F_{sB1} - F_{z12} \approx -0.15 \unit{(kN)}$\\
At cross section $ B_1 $:\\
$ F_{sA1} = e_1\sqrt{R_{A1x}^2+R_{A1y}^2}  \approx 0.41\unit{(kN)}$\\
$ \displaystyle\sum F_{aB1} = F_{sA1} + F_{z12} \approx 0.99 \unit{(kN)}$\\
From equation (11.11a) and (11.11b):\\
$  \displaystyle\sum F_{aA1} \leq F_{sA1} \Rightarrow F_{atA1} = F_{sA1}  \approx  0.41\unit{(kN)} $\\
$ \displaystyle \sum F_{aB1} > F_{sB1} \Rightarrow F_{atB1} = \sum F_{aB1}  \approx  0.99\unit{(kN)} $

In contrast, shaft 2 does not have such additional forces since $ \alpha_{sh2} = 0^\circ $. Therefore, $ F_{atA2} = F_{atB2} = |F_{z21}| = 0.58\unit{(kN)} $.

\paragraph{Find $ Q_1,Q_2 $}
Recall that the forces are in absolute values, the parameters are substituted to the formula to obtain:\\
$ Q_{A1} \approx 1.56 \unit{(kN)} $\\
$ Q_{B1} \approx 2.82 \unit{(kN)} $\\
$ Q_{A2} \approx2.66  \unit{(kN)} $\\
$ Q_{B2} \approx 1.87 \unit{(kN)} $\\
We will compare these values and choose the larger load (according to the recommendation on p.219):\\
$ Q_{A1} < Q_{B1} \Rightarrow Q_1 = Q_{B1}  \approx 2.82 \unit{(kN)}$\\
$ Q_{A2} > Q_{B2} \Rightarrow Q_2 = Q_{A2}  \approx 2.66 \unit{(kN)}$\\

\subsubsection{Find rated life}
Equation (11.2) is rearranged to calculate $ L $:
\[L = L_h60n_{sh}\times10^{-6}\]
In our transmission system, since the gearbox is a speed reducer working 2 shifts daily, we approximate $ L_h \approx 32000 \unit{(hours)}$ according to table (11.2), which gives:\\
$ L_1 \approx 5274 \unit{(\text{million rev})} $\\
$ L_2 \approx 1054.8 \unit{(\text{million rev})} $\\
Combining the results and letting $ m=3 $ (ball bearings are used in this case) yield:\\
$ C_{d1} = 14385.63 \unit{(N)} $