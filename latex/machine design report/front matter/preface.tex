\chapter*{Abstract}
In machine design, every machine element must be calculated in a systematic matter. In this course, students are provided with essential skills to formulate almost every dimension manually, thus further improving their engineering skills before engaging the high-energy, fast-paced workforce.

When a machine element is being developed, it must satisfy some key engineering specifications such as being able to operate under designated lifespan, low cost and high efficiency. Other aspects are less important but also determined the overall design of the element include compactness, noise emission, appearance, etc.

To optimize the process of machine design, the general principles are considered as follows:
\begin{enumerate}
	\item Identify the working principle and workload of the machine.
	\item Formulate the overall working principle to satisfy the problem. Proposing feasible solutions and evaluating them to find the optimal design specifications.
	\item Find force and moment diagram exerting on machine parts and characteristics of the workload.
	\item Choose appropriate materials to make use of their properties and improve efficiency as well as reliability of individual elements.
	\item Calculate dynamics, strength, safety factor, etc. to specify dimensions.
	\item Design machine structure, parts to satisfy working condition and assembly.
	\item Create presentation, instruction manual and maintenance.
\end{enumerate}
In this report, I will design a fairly simple system to provide a concrete example of finishing all the tasks above.