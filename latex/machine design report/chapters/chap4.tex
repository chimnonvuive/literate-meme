\chapter{Gearbox Design (Helix gears)}
\section*{Nomenclature}
\begin{tabular}[t]{p{0.1\linewidth}p{0.35\linewidth}}
	$ [\sigma] $ & permissible stress, $ \unit{MPa} $\\
	$ [\sigma]_{max} $ & permissible stress due to overloading, $ \unit{MPa} $\\
	$ \text{AG} $ & accuracy grade of gear\\
	$ a $ & center distance, $ \unit{mm} $\\
	$ b $ & face width, $ \unit{mm} $\\
	$ c $ & gear meshing rate\\
	$ d $ & pitch circle, $ \unit{mm} $\\
	$ d_a $ & addendum diameter, $ \unit{mm} $\\
	$ d_b $ & base diameter, $ \unit{mm} $\\
	$ d_f $ & deddendum diameter, $ \unit{mm} $\\
%	$ F_a $ & axial force, $ \unit{N} $\\
%	$ F_r $ & radial force, $ \unit{N} $\\
%	$ F_t $ & tangential force, $ \unit{N} $\\
	$ H $ & surface roughness, HB\\
\end{tabular}
\begin{tabular}[t]{p{0.05\linewidth}p{0.4\linewidth}}
	$ K_d $ & coefficient of gear material\\	
	$ K $ & overall load factor\\
	$ K_{C} $ & load placement factor\\
	$ K_{L} $ & aging factor\\
	$ K_{v} $ & dynamic load factor at meshing area\\
	$ K_{\alpha} $ & load distribution factor on gear teeth\\
	$ K_{\beta} $ & load distribution factor on top land\\
	$ K_{qt} $ & overloading factor\\
\end{tabular}\newpage
\noindent\begin{tabular}[t]{p{0.05\linewidth}p{0.4\linewidth}}
	$ m_t $ & traverse module, $ \unit{mm} $\\
	$ m $ & root of fatigue curve in stress test\\
	$ m_n $ & normal module, $ \unit{mm} $\\
	$ N_{E} $ & working cycle of equivalent tensile stress\\
	$ N_{O} $ & working cycle of bearing stress\\
	$ S_F $ & safety factor of bending stress\\
	$ S_H $ & safety factor of contact stress\\
	$ v $ & rotational velocity, $ \unit{m/s} $\\
	$ Y_F $ & tooth shape factor\\
	$ Y_R $ & surface roughness factor\\
	$ Y_s $ & stress concentration factor\\
	$ Y_\beta $ & helix angle factor\\
	$ Y_\varepsilon $ & contact ratio factor\\
	$ Z_R $ & surface roughness factor of the working's area\\
	$ Z_v $ & speed factor\\
	$ z_H $ & contact surface's shape factor\\
	$ z_M $ & material's mechanical properties factor \\
	$ z_{min} $ & minimum number of teeth corresponding to $ \beta $\\
\end{tabular}
\begin{tabular}[t]{p{0.05\linewidth}p{0.4\linewidth}}
	
	$ z_v $ & virtual number of teeth\\
	$ z_\varepsilon $ & meshing condition factor\\
	$ \alpha $ & normal pressure angle. In TCVN 1065-71, $ \alpha = 20^\circ $\\
	$ \alpha_t $ & traverse pressure angle, $ ^\circ $\\
	$ \varepsilon_\alpha $ & traverse contact ratio\\
	$ \varepsilon_\beta $ & face contact ratio\\
	$ \beta $ & helix angle, $ ^\circ $\\
	$ \beta_b $ & base circle helix angle, $ ^\circ $\\
	$ \psi_{ba} $ & width to shaft distance ratio\\
	$ \psi_{bd} $ & face width factor \\
	$ \sigma_b $ & ultimate strength, $ \unit{MPa} $\\
	$ \sigma_{ch} $ & yield limit, $ \unit{MPa} $\\
	$ \sigma_{lim}^o $ & permissible stress corresponding to working cycle, $ \unit{MPa} $\\
	$ \sigma_{max} $ & stress due to overloading, $ \unit{MPa} $ \\
	$ _{1} $ & subscript for the pinion \\
	$ _{2} $ & subscript for the driven gear\\
	$ _F $ & subscript relating to bending stress\\
	$ _H $ & subscript relating to contact stress\\
	$ _q $ & subscript for the quick transmission stage\\
	$ _s $ & subscript for the slow transmission stage\\
	$ _w $ & subscript for the value after correction\\
\end{tabular}

\paragraph{Known parameters} From Chapter 1 and 2, we know that:\\
$ L=8\unit{years} $, $ K_{ng}=260\unit{days} $, $ Ca=1\unit{shift} $\\
$ T_1=T$, $T_2=0.7T$, $ t_1=15\unit{s}$, $t_2=11\unit{s} $\\
$ n_{sh1}=2922\unit{rpm} $, $ n_{sh2}=905.02\unit{rpm} $, $ n_{sh3}=292.20\unit{rpm} $\\
$ u_h=10 $, $ u_1 =u_q=3.23$, $ u_2=u_s=3.1 $

This chapter will increase readability by calculating both stages at the same time with the first one being quick stage and the latter is slow stage.
\section{Choose material}
Because all gears are the same in material and working hours except for their angular rotational speed, this section applies for both pairs.

The material of choice for the 2 pair of gears is steel 40X. The specifications are $ \text{HB} 250 $, $ \sigma_b = 850\unit{MPa} $, $ \sigma_{ch} = 550 \unit{MPa}$, see Table 6.1  \cite{tk1}.

From Table 6.2 \cite{tk1}, $ \sigma_{Hlim}^o = 2\text{HB} + 70$, $ S_H = 1.1 $, $ \sigma_{Flim}^o = 1.8\text{HB} $, $ S_F = 1.75 $.

Therefore, they have the same properties except for their surface roughness $ H $, since $ H_2 = H_1 - 10 \div 15$.

For the pinion, $ H_1=\text{HB}250 \Rightarrow \sigma_{Hlim1}^o = 570\unit{MPa}$, $ \sigma_{Flim1}^o = 450\unit{MPa}$.

For the driven gear, $ H_2=\text{HB}240 \Rightarrow \sigma_{Hlim2}^o = 550\unit{MPa}$, $ \sigma_{Flim2}^o = 432\unit{MPa}$.

In this part, distinguishing between 2 stages is unnecessary since the variables are material-dependent, which in this case all the gears have identical choice of material. Therefore, unless otherwise specified, a single subscript $ 1 $ or $ 2 $ indicates the variable applies for both stages

\section{Calculate $ [\sigma_H] $ and $ [\sigma_F] $}
The permissible stresses are calculated as follows, see Equation 6.1 and 6.2 \cite{tk1}:
\[[\sigma_{H}]=\dfrac{\sigma_{Hlim}^o}{S_H}Z_RZ_vK_{xH}K_{HL}\]
\[[\sigma_{F}]=\dfrac{\sigma_{Flim}^o}{S_F}Y_RY_sK_{xF}K_{FL}\]

\textbf{Calculate working cycle of bearing stress} The stress is found using the formula below, see Equation 6.5 \cite{tk1}:\\
$ N_{HO1} = 30H_1^{2.4} = 17067789.40\unit{(cycles)}$.\\
$ N_{HO2} = 30H_2^{2.4} = 15474913.67\unit{(cycles)}$.\\

\textbf{Calculate working cycle of equivalent tensile stress} Since $ H_1,H_2\leq\text{HB}350 $, $ m_H=6 $, $ m_F=6 $. Also, both pairs of gears are meshed indefinitely, which makes $ c=1 $. From working condition, we calculate:\\
$ L_h = 8\unit{\left( \dfrac{hours}{shift}\right)}\times Ca  K_{ng} L=16640\unit{(hours)}$\\
Find $ N_{HE} $ and $ N_{FE} $  using equation 6.7 and 6.8 \cite{tk1}:\\
$ N_{HE1q} = 60n_{sh1}cL_h\left[ \left( \dfrac{T_1}{T}\right)^3\dfrac{t_1}{t_1+t_2} + \left( \dfrac{T_2}{T}\right)^3\dfrac{t_2}{t_1+t_2}\right] = 2.11\times10^9\unit{(cycles)}$\\
$ N_{HE2q} = 60n_{sh2}cL_h\left[ \left( \dfrac{T_1}{T}\right)^3\dfrac{t_1}{t_1+t_2} + \left( \dfrac{T_2}{T}\right)^3\dfrac{t_2}{t_1+t_2}\right] = 0.65\times10^9\unit{(cycles)}$\\
$ N_{FE1q} = 60n_{sh1}cL_h\left[ \left( \dfrac{T_1}{T}\right)^{m_F}\dfrac{t_1}{t_1+t_2} + \left( \dfrac{T_2}{T}\right)^{m_F}\dfrac{t_2}{t_1+t_2}\right] = 1.83\times10^9\unit{(cycles)}$\\
$ N_{FE2q} = 60n_{sh2}cL_h\left[ \left( \dfrac{T_1}{T}\right)^{m_F}\dfrac{t_1}{t_1+t_2} + \left( \dfrac{T_2}{T}\right)^{m_F}\dfrac{t_2}{t_1+t_2}\right] = 0.57\times10^9\unit{(cycles)}$\\
$ N_{HE1s} = 60n_{sh2}cL_h\left[ \left( \dfrac{T_1}{T}\right)^3\dfrac{t_1}{t_1+t_2} + \left( \dfrac{T_2}{T}\right)^3\dfrac{t_2}{t_1+t_2}\right] = 0.65\times10^9\unit{(cycles)}$\\
$ N_{HE2s} = 60n_{sh3}cL_h\left[ \left( \dfrac{T_1}{T}\right)^3\dfrac{t_1}{t_1+t_2} + \left( \dfrac{T_2}{T}\right)^3\dfrac{t_2}{t_1+t_2}\right] = 0.21\times10^9\unit{(cycles)}$\\
$ N_{FE1s} = 60n_{sh2}cL_h\left[ \left( \dfrac{T_1}{T}\right)^{m_F}\dfrac{t_1}{t_1+t_2} + \left( \dfrac{T_2}{T}\right)^{m_F}\dfrac{t_2}{t_1+t_2}\right] = 0.57\times10^9\unit{(cycles)}$\\
$ N_{FE2s} = 60n_{sh3}cL_h\left[ \left( \dfrac{T_1}{T}\right)^{m_F}\dfrac{t_1}{t_1+t_2} + \left( \dfrac{T_2}{T}\right)^{m_F}\dfrac{t_2}{t_1+t_2}\right] = 0.18\times10^9\unit{(cycles)}$\\

\textbf{Calculate aging factor} 
For steel, $ N_{FO1}= N_{FO2}= 4\times10^6\unitp{MPa}$. Equation 6.3 and 6.4 \cite{tk1}  gives (if the factors are smaller than 1, round them up to 1, see p.94 \cite{tk1}):\\
$ K_{HL1q} = \sqrt[m_H]{N_{HO1}/N_{HE1q}} = 0.45 < 1 \Rightarrow K_{HL1q} = 1$\\
$ K_{HL2q} = \sqrt[m_H]{N_{HO2}/N_{HE2q}} = 0.54 < 1 \Rightarrow K_{HL2q} = 1$\\
$ K_{FL1q} = \sqrt[m_F]{N_{FO1}/N_{FE1q}} = 0.36 < 1 \Rightarrow K_{FL1q} = 1$\\
$ K_{FL2q} = \sqrt[m_F]{N_{FO2}/N_{FE2q}} = 0.44 < 1 \Rightarrow K_{FL2q} = 1$\\
$ K_{HL1s} = \sqrt[m_H]{N_{HO1}/N_{HE1s}} = 0.54 < 1 \Rightarrow K_{HL1s} = 1$\\
$ K_{HL2s} = \sqrt[m_H]{N_{HO2}/N_{HE2s}} = 0.65 < 1 \Rightarrow K_{HL2s} = 1$\\
$ K_{FL1s} = \sqrt[m_F]{N_{FO1}/N_{FE1s}} = 0.44 < 1 \Rightarrow K_{FL1s} = 1$\\
$ K_{FL2s} = \sqrt[m_F]{N_{FO2}/N_{FE2s}} = 0.53 < 1 \Rightarrow K_{FL2s} = 1$\\

\textbf{Calculate $ [{\sigma}_H] $, $ [{\sigma}_{F1}] $, $ [{\sigma}_{F2}] $}
Since the motor works in one direction, $ K_{FC} = 1$, which means all $ K $ factors are equal to $ 1 $ and we can safely skip them.

For initial estimation, assume $ Z_RZ_vK_{xH} = 1 $ and $ Y_RY_sK_{xF} = 1 $, we obtain the permissible stresses (notice they are equal in both quick and slow transmission stages):\\
$ [{\sigma}_{H1}] = \sigma_{Hlim1}^o/S_{H} = 518.18 \unit{(MPa)}$\\
$ [{\sigma}_{H2}] = \sigma_{Hlim2}^o/S_{H} = 500 \unit{(MPa)}$\\
$ [{\sigma}_{F1}] = \sigma_{Flim1}^o/S_{F} = 257.14\unit{(MPa)}$\\
$ [{\sigma}_{F2}] = \sigma_{Flim2}^o/S_{F} = 246.86 \unit{(MPa)}$

The mean permissible contact stress must be lower than 1.25 times of either $ [{\sigma}_{H1}] $ or $ [{\sigma}_{H2}] $, whichever is smaller. In this case, it is $ 1.25[{\sigma}_{H2}]$ or $ 625\unit{MPa} $:
\[[{\sigma}_{H}] =\dfrac{1}{2}\left( [{\sigma}_{H1}]+[{\sigma}_{H2}]\right)  = 509.09 \unit{(MPa)}\leq 625\]
which satisfy the condition.

In case of overloading, the permissible contact and bending stresses are calculated as follows:\\
$ [{\sigma}_H]_{max} = 2.8\sigma_{ch} = 1540\unitp{MPa} $\\
$ [{\sigma}_F]_{max} = 0.8\sigma_{ch} = 440\unitp{MPa} $

\section{Determine basic specifications of the Transmission system}

\subsection{Determine basic parameters}
	Figure \ref{fig:problem} shows both pairs are helical, which  gives $ K_a = 43 $, Table 6.5 \cite{tk1}. Also, the entire speed reducer has asymmetrical design, resulting in $ \psi_{ba} = 0.4$, Table 6.6 \cite{tk1}. This value is then used in Equation 6.16 \cite{tk1} to find $ \psi_{bd} $:\\
$ \psi_{bdq} = 0.53\psi_{ba}(u_q+1) = 0.90 $\\
$ \psi_{bds} = 0.53\psi_{ba}(u_s+1) = 0.87 $

Using interpolation, we approximate the factors, Table 6.7 \cite{tk1}:\\
$ K_{H\beta q} = 1.06 $, $ K_{F\beta q} = 1.14 $\\
$ K_{H\beta s} = 1.09 $, $ K_{F\beta s} = 1.19 $\\
Since the gear system only consists of involute gears and it is also a speed reducer gearbox, we estimate $ a $ using Equation 6.15a \cite{tk1}. Also, we round up the center distance to the nearest multiple of 5 for small production, p.99 \cite{tk1}:\\
$ a_q = K_a(u_{q}+1)\sqrt[3]{\dfrac{T_{sh1}K_{H\beta q}}{[{\sigma}_H]^2u_{q}\psi_{ba}}} = 76.06 \unitp{mm} \Rightarrow a_{q}= 85 \unit{mm}$\\
$ a_s = K_a(u_{s}+1)\sqrt[3]{\dfrac{T_{sh2}K_{H\beta s}}{[{\sigma}_H]^2u_{s}\psi_{ba}}} = 110.00 \unitp{mm} \Rightarrow a_{s}= 120 \unit{mm}$
\subsection{Determine gear meshing parameters}
In this section, we will find all necessary specifications of a gear. Due to complexity and maintenance, the undercutting process will be omitted throughout the calculations, that is, it is not factored in any variables of the gear drives.

\textbf{Find $ m $} Using Equation 6.17 \cite{tk1} and Table 6.8 \cite{tk1}, we determine $ m $ for each pair of gears:\\
$ m_{tq} = (0.01\div0.02)a_{q} = 0.85 \div 1.7 \unitp{mm} \Rightarrow m_{tq}=\SI{1.5}{mm}$\\
$ m_{ts} = (0.01\div0.02)a_{s} = 1.2 \div 2.4 \unitp{mm} \Rightarrow m_{ts}=\SI{2}{mm}$

\textbf{Find $ z_1 $, $ z_2 $, $ b $} Arbitrarily choose $ \beta = 20^\circ $ in the range $ 8 \div 20^\circ $. Combining Equation 6.18 and 6.20 \cite{tk1} to calculate $ z_1 $. Then, find $ z_2 $ and $ b $.\\
$ z_{1q} = \dfrac{2a_{q}\cos\beta}{m_{tq}(u_q+1)} = 25.18 \Rightarrow z_{1q} = 26$\\
$ z_{2q} = u_{q}z_{1q} = 83.95 \Rightarrow z_{2q} = 84 $\\
$ b_q = \psi_{ba}a_{q} = 34.00\unit{(mm)}$\\
$ z_{1s} = \dfrac{2a_{s}\cos\beta}{m_{ts}(u_s+1)} = 27.52 \Rightarrow z_{1s} = 28$\\
$ z_{2s} = u_{s}z_{1s} = 86.72 \Rightarrow z_{2s} = 87 $\\
$ b_s = \psi_{ba}a_{s} = 48.00\unit{(mm)}$

\textbf{Correct $ \beta $} The helix angles are corrected to compensate for rounding center distances and number of teeth, Equation 6.32 \cite{tk1}:\\
$ \beta_{wq} = \dfrac{m_{tq}(z_{1q}+z_{2q})}{2a_q} = 13.93 ^\circ$\\
$ \beta_{ws} = \dfrac{m_{ts}(z_{1s}+z_{2s})}{2a_s} = 16.6 ^\circ$

%$ y_q = \dfrac{a_{wq}}{m_{tq}} - 0.5(z_{1q}+z_{2q}) = 1.69$\\
%$ k_{yq} = \dfrac{1000y_q}{z_{1q}+z_{2q}} = 15.41 $\\
%$ y_s = \dfrac{a_{ws}}{m_{ts}} - 0.5(z_{1s}+z_{2s}) =0.59 $\\
%$ k_{ys} = \dfrac{1000y_s}{z_{1s}+z_{2s}} = 4.96 $\\
%From $ k_y $, we inspect Table 6.10a \cite{tk1} and obtain $ k_x $. The results are $ k_{xq}=1.63 $, $ k_{xs}=0.19 $.
%
%The only thing left to do  is to find $ x_1 $, $ x_2 $ and $ \alpha_{tw} $.\\
%$ \Delta_{yq}=\dfrac{k_{xq}(z_{1q}+z_{2q})}{1000}= 0.18$\\
%$ x_{1q}=0.5(y_q+\Delta_{yq}-\dfrac{(z_{2q}-z_{1q})y}{z_{1q}+z_{2q}}) =0.49$\\
%$ x_{2q}=y_q+\Delta_{yq}-x_{1q} =1.38$\\
%$ \alpha_{twq}=\arccos\left[\dfrac{(z_{1q}+z_{2q})m_{tq}\cos\alpha}{2a_{wq}}\right]= 24.27^\circ$\\
%$ \Delta_{ys}=\dfrac{k_{xs}(z_{1s}+z_{2s})}{1000}= 0.02$\\
%$ x_{1s}=0.5(y_s+\Delta_{ys}-\dfrac{(z_{2s}-z_{1s})y}{z_{1s}+z_{2s}}) =0.16$\\
%$ x_{2s}=y_s+\Delta_{ys}-x_{1s} =0.46$\\
%$ \alpha_{tws}=\arccos\left[\dfrac{(z_{1s}+z_{2s})m_{ts}\cos\alpha}{2a_{ws}}\right]= 21.49^\circ$

%\paragraph{Find $ x_1 $, $ x_2 $} To find $ x_1 $ and $ x_2 $, we will follow the calculation scheme provided in p.103. Since $ \beta_w \approx 13.59^\circ \in (10,15]$, $ z_{min} = 11$, which leads to $ z_1 $ satisfying condition $ z_1 \geq z_{min} + 2 > 10 $, according to table (6.9). Combined with $ u_{hg} = 5 \geq 3.5 $, we obtain $ x_1 = 0.3 $, $ x_2 = -0.3 $, disregarding the calculation of $ y $.

\subsection{Basic parameters}

For quick stage transmission:\\
\begin{tabular}[t]{p{0.5\linewidth}}
	$ d_{1q} = \dfrac{m_{tq}z_{1q}}{\cos\beta_{wq}} = 40.18\unit{(mm)} $\\
	$ d_{2q} = \dfrac{m_{tq}z_{2q}}{\cos\beta_{wq}} = 129.82\unit{(mm)} $\\
	$ d_{a1q} = d_{1q} + 2m_{tq} = 43.18\unit{(mm)}$\\
	$ d_{a2q} = d_{2q} + 2m_{tq} = 132.82\unit{(mm)}$\\
	$ d_{f1q} = d_{1q} - 2.5m_{tq} = 36.43\unit{(mm)}$\\
	$ d_{f2q} = d_{2q} - 2.5m_{tq} = 126.07\unit{(mm)}$\\
\end{tabular}~
\begin{tabular}[t]{p{0.5\linewidth}}
	$ d_{b1q} = d_{1q}\cos\alpha = 37.76\unit{(mm)}$\\
	$ d_{b2q} = d_{2q}\cos\alpha = 121.99 \unit{(mm)}$\\
	$ \alpha_{tq} = \arctan\dfrac{\tan\alpha}{\cos\beta_{wq}} = 20.56^\circ $\\
	$ v_{1q} = \dfrac{\pi d_{1q}n_{sh1}}{60000} = 6.15\unit{(m/s)}$
\end{tabular}\\\\
For slow stage transmission:\\
\begin{tabular}[t]{p{0.5\linewidth}}
	$ d_{1s} = \dfrac{m_{ts}z_{1s}}{\cos\beta_{ws}} = 58.43\unit{(mm)} $\\
	$ d_{2s} = \dfrac{m_{ts}z_{2s}}{\cos\beta_{ws}} = 181.57\unit{(mm)} $\\
	$ d_{a1s} = d_{1s} + 2m_{ts} = 62.43\unit{(mm)}$\\
	$ d_{a2s} = d_{2s} + 2m_{ts} = 185.57\unit{(mm)}$\\
	$ d_{f1s} = d_{1s} - 2.5m_{ts} = 53.43\unit{(mm)}$\\
	$ d_{f2s} = d_{2s} - 2.5m_{ts} = 176.57\unit{(mm)}$\\
\end{tabular}~
\begin{tabular}[t]{p{0.5\linewidth}}
	$ d_{b1s} = d_{1s}\cos\alpha = 54.91\unit{(mm)}$\\
	$ d_{b2s} = d_{2s}\cos\alpha = 170.62 \unit{(mm)}$\\
	$ \alpha_{ts} = \arctan\dfrac{\tan\alpha}{\cos\beta_{ws}} = 20.80^\circ $\\
	$ v_{1s} = \dfrac{\pi d_{1s}n_{sh2}}{60000} = 2.77\unit{(m/s)}$
\end{tabular}

\section{Stress analysis}
\subsection{Correct $ [{\sigma}_{H}] $, $ [\sigma_{F1}] $ and $ [\sigma_{F2}] $}
In reality, the factors of Equation 6.1 and 6.2 \cite{tk1} do not equal 1. As a result, we will try to approximate them given the values calculated above:
\[
\begin{array}{l@{{} = {}}l}
[\sigma_{Hw}] & [{\sigma}_H]Z_RZ_VK_{xH}\\

[\sigma_{Fw}] & [{\sigma}_F]Y_RY_SK_{xF}\\
\end{array}
\]
Assuming smooth surface condition, $ Z_R = 1 $.\\
All gears have low surface hardness:\\
$ Z_{vq} = 0.85v_{1q}^{0.1} = 1.02$\\
$ Z_{vs} = 0.85v_{1s}^{0.1} = 0.94$\\
All gears have small addendum, $ K_{xH} = 1$.\\
The gears are properly polished, which means $ Y_R=1.2 $\\
$ Y_{sq} = 1.08-0.0695\ln(m_{tq}) = 1.05 $\\
$ Y_{ss} = 1.08-0.0695\ln(m_{ts}) = 1.03 $\\
Since $ d_{a1},d_{a2}\leq400\unit{(mm)} $, $ K_{xF}=1 $. Then, multiplying all variables yields:\\
\begin{tabular}[t]{ll}
	$ [{\sigma}_{Hwq}] = 518.90\unit{(MPa)}$ & $ [{\sigma}_{Hws}] = 479.12\unit{(MPa)}$\\
	$ [{\sigma}_{Fw1q}] = 324.56\unit{(MPa)}$& $ [{\sigma}_{Fw1s}] = 318.39\unit{(MPa)}$\\
	$ [{\sigma}_{Fw2q}] = 311.58\unit{(MPa)}$& $ [{\sigma}_{Fw2s}] = 305.66\unit{(MPa)}$\\
\end{tabular}
\subsection{Contact stress analysis}
The contact stress applied on a gear surface must satisfy Equation 6.33 \cite{tk1}.

\textbf{Find $ z_M $} According to Table 6.5 \cite{tk1}, $ z_M = 274 $.

\textbf{Find $ z_H $} Applying Equation 6.34 \cite{tk1} and 6.35 \cite{tk1}:\\
$ \beta_{bq} = \arctan\left( \cos\alpha_{tq}\tan\beta_{wq}\right) = 13.08^\circ \Rightarrow z_{Hq} = \sqrt{2\dfrac{\cos\beta_{bq}}{\sin(2\alpha_{tq})}} = 1.72$\\
$ \beta_{bs} = \arctan\left( \cos\alpha_{ts}\tan\beta_{ws}\right) = 15.57^\circ \Rightarrow z_{Hs} = \sqrt{2\dfrac{\cos\beta_{bs}}{\sin(2\alpha_{ts})}} = 1.70$

\textbf{Find $ z_\varepsilon $} Obtaining $ z_\varepsilon $ through calculations:\\
$ \varepsilon_{\alpha q} = \dfrac{\sqrt{d_{a1q}^2-d_{b1q}^2}+\sqrt{d_{a2q}^2-d_{b2q}^2}-2a_{q}\sin\alpha_{tq}}{2\pi m_{tq}\dfrac{\cos\alpha_{tq}}{\cos\beta_q}} = 1.52$\\
$ \varepsilon_{\beta q} = b_q\dfrac{\sin\beta_q}{m_{tq}\pi} >1 \Rightarrow z_{\varepsilon q} = \varepsilon_{\alpha q}^{-0.5} = 0.81 $\\
$ \varepsilon_{\alpha s} = \dfrac{\sqrt{d_{a1s}^2-d_{b1s}^2}+\sqrt{d_{a2s}^2-d_{b2s}^2}-2a_{s}\sin\alpha_{ts}}{2\pi m_{ts}\dfrac{\cos\alpha_{ts}}{\cos\beta_s}} = 1.43$\\
$ \varepsilon_{\beta s} = b_s\dfrac{\sin\beta_s}{m_{ts}\pi} >1 \Rightarrow z_{\varepsilon s} = \varepsilon_{\alpha s}^{-0.5} = 0.84 $

\textbf{Find $ K_H $ and $ K_F $} We find $ K_H $, $ K_F $ using Equation 6.39 and 6.45 \cite{tk1}:\\
From Table 6.13 \cite{tk1}:\\
$ v_{1q}\leq 10 \unitp{m/s}\Rightarrow \text{AG}_q = 8 $ \\
$ v_{1s}\leq 4 \unitp{m/s}\Rightarrow \text{AG}_s = 9 $ \\
From Table P2.3 \cite{tk1}, using interpolation, we approximate:\\
$ K_{Hvq} = 1.06$, $ K_{Fvq} = 1.17$\\
$ K_{Hvs} = 1.04$, $ K_{Fvs} = 1.10$\\
From Table 6.14 \cite{tk1}, using interpolation, we approximate:\\
$ K_{H\alpha q} = 1.1$, $ K_{F\alpha q} = 1.29 $ \\
$ K_{H\alpha s} = 1.13$, $ K_{F\alpha s} = 1.37 $ \\		
Knowing the results of $ K_{H\beta} $ from previous section, multiply all the values to obtain $ K_H $:\\
$ \Rightarrow K_{Hq} = 1.24 $, $ K_{Fq} = 1.73 $\\
$ \Rightarrow K_{Hs} = 1.28 $, $ K_{Fs} = 1.79 $

\textbf{Find $ \sigma_H $} After calculating $ z_M $, $ z_H $, $ z_\varepsilon $, $ K_H $, we get the following result:\\
For quick stage with transmission ratio $ u_q $ and input torque $ T_{sh1} $:
\[
\sigma_{Hq} = z_Mz_{Hq}z_{\varepsilon q}\sqrt{2T_{sh1}K_{Hq}\dfrac{u_q+1}{b_qu_qd_{1q}^2}} \leq [{\sigma}_{Hwq}]
\]
\[\sigma_{Hq} = 447.07 \unitp{MPa}\leq 518.90 \] 
For slow stage with transmission ratio $ u_s $ and input torque $ T_{sh2} $:
\[
\sigma_{Hs} = z_Mz_{Hs}z_{\varepsilon s}\sqrt{2T_{sh3}K_{Hs}\dfrac{u_s+1}{b_su_sd_{1s}^2}} \leq [{\sigma}_{Hws}]
\]
\[\sigma_{Hs} = 476.18 \unitp{MPa}\leq 479.12 \] 
\subsection{Bending stress analysis}
For safety reasons, Equation 6.43 and 6.44 \cite{tk1} must be met for both pairs of gears.

\textbf{Find $ Y_\varepsilon $} Using $ \varepsilon_\alpha $ calculated in the previous section, we find $ Y_\varepsilon$:\\
$ Y_{\varepsilon q}= \varepsilon_{\alpha q}^{-1} = 0.66 $\\
$ Y_{\varepsilon s}= \varepsilon_{\alpha s}^{-1} = 0.70 $

\textbf{Find $ Y_\beta $} The value of $ Y_\beta $ is calculated using the equation on p.108 \cite{tk1}:\\
$ Y_{\beta q} = 1-\dfrac{\beta_{wq}}{140}= 0.90$\\
$ Y_{\beta s} = 1-\dfrac{\beta_{ws}}{140}= 0.88$

\textbf{Find $ Y_F $} Using the formula $ z_v = z\cos^{-3}(\beta_w) $ and Table 6.18 \cite{tk1}:\\
$ z_{v1q} = z_{1q}\cos^{-3}(\beta_{wq}) = 28.44\Rightarrow Y_{F1q} = 3.83 $\\
$ z_{v2q} = z_{2q}\cos^{-3}(\beta_{wq}) = 91.87\Rightarrow Y_{F2q} = 3.60 $\\
$ z_{v1s} = z_{1s}\cos^{-3}(\beta_{ws}) = 31.81\Rightarrow Y_{F1s} = 3.78 $\\
$ z_{v2s} = z_{2s}\cos^{-3}(\beta_{ws}) = 98.85\Rightarrow Y_{F2s} = 3.60 $

\textbf{Find $ K_F $} The value of $ K_F $ has already been found in the previous section.

\textbf{Find $ \sigma_F $} Replacing the normal module in Equation 6.43 \cite{tk1} with $ m_n = m_t\cos\beta_w $, substituting all the values yields:

For quick stage with input torque $ T_{sh1} $:
\[
\begin{array}{l@{{} \leq {}}l}
\sigma_{F1q} = 2\dfrac{T_{sh1}K_{Fq}Y_{\varepsilon q} Y_{\beta q} Y_{F1q}}{b_qd_{1q}m_{tq}\cos\beta_{wq}} & [\sigma_{Fw1q}]\\ 
\sigma_{F2q} = \dfrac{\sigma_{F1q}Y_{F2q}}{Y_{F1q}} & [\sigma_{Fw2q}]
\end{array}
\]
gives the results:
\[
\begin{array}{l@{{} \leq {}}l}
\sigma_{F1q} = 91.48 \unitp{MPa} & 324.56 \unit{MPa}\\
\sigma_{F2q} = 86.05 \unitp{MPa} & 311.58 \unit{MPa}
\end{array}
\]

For slow stage with input torque $ T_{sh2} $:
\[
\begin{array}{l@{{} \leq {}}l}
\sigma_{F1s} = 2\dfrac{T_{sh2}K_{Fs}Y_{\varepsilon s} Y_{\beta s} Y_{F1s}}{b_sd_{1s}m_{ts}\cos\beta_{ws}} & [\sigma_{Fw1s}]\\ 
\sigma_{F2s} = \dfrac{\sigma_{F1s}Y_{F2s}}{Y_{F1s}} & [\sigma_{Fw2s}]
\end{array}
\]
gives the results:
\[
\begin{array}{l@{{} \leq {}}l}
\sigma_{F1s} = 111.83 \unitp{MPa} & 318.39 \unit{MPa}\\
\sigma_{F2s} = 106.47 \unitp{MPa} & 305.66 \unit{MPa}
\end{array}
\]
which is well below the yielding strength.
\subsection{Overloading analysis}
Inspecting Table P1.3 \cite{tk1}, the motor we chose in Chapter 2 has $ K_{qt}=2.2 $. Using the values of $ [\sigma_H]_{max} $ and $ [{\sigma}_F]_{max} $ calculated in previous section combined with Equation 6.48 and 6.49 \cite{tk1}, we are able to verify the stresses are below overloading limits.

For quick stage transmission:
\[
\begin{array}{l}
\sigma_{Hmaxq}=\sigma_{Hq}\sqrt{K_{qt}} = 663.11 \unitp{MPa} \leq 1540 \unit{MPa}\\
\sigma_{F2maxq}=\sigma_{F2q}K_{qt} = 189.31 \unitp{MPa} \leq 440 \unit{MPa}\\
\sigma_{F1maxs}=\sigma_{F1s}K_{qt} = 246.03 \unitp{MPa} \leq 440 \unit{MPa}
\end{array}
\]

For slow stage transmission:
\[
\begin{array}{l}
\sigma_{Hmaxs}=\sigma_{Hs}\sqrt{K_{qt}} = 706.29 \unitp{MPa} \leq 1540 \unit{MPa}\\
\sigma_{F1maxq}=\sigma_{F1q}K_{qt} = 201.25 \unitp{MPa} \leq 440 \unit{MPa}\\
\sigma_{F2maxs}=\sigma_{F2s}K_{qt} = 234.24 \unitp{MPa} \leq 440 \unit{MPa}
\end{array}
\]
which satisfy the conditions. In summary, we have the following table:
%\subsection{Force on shafts}
%$ F_t = \dfrac{2T_{sh1}}{d_{w1}} \approx 2402.28 \unit{(N)}$\\
%$ F_r = F_t\tan\alpha_{tw} \approx 899.55 \unit{(N)}$\\
%$ F_a = F_t\tan\beta_w\approx 580.75 \unit{(N)}$\\

\begin{table}[ht]
	\centering
	\begin{tabular}[t]{|
			>{\columncolor[HTML]{C0C0C0}}l |p{2.5cm}|p{2.5cm}|}
		\hline
		& \multicolumn{1}{c|}{\cellcolor[HTML]{C0C0C0}pinion} & \multicolumn{1}{c|}{\cellcolor[HTML]{C0C0C0}driving gear} \\ \hline
		$ a_w\unit{(mm)} $            & \multicolumn{2}{l|}{\hskip2cm 100}           \\ \hline
		$ b\unit{(mm)} $            & \multicolumn{2}{l|}{\hskip2cm 50}           \\ \hline
		$ m\unit{(mm)} $              & \multicolumn{2}{l|}{\hskip2cm 1.5}    \\ \hline
		$ d_w\unit{(mm)} $              & 33.33                    & 166.67 \\ \hline
		$ d_a\unit{(mm)} $            & 37.23                    & 168.77 \\ \hline
		$ d_f\unit{(mm)} $            & 30.48                    & 162.02 \\ \hline
		$ d_b\unit{(mm)} $            & 31.32                     & 156.62 \\ \hline
		$ u_{hg} $              & \multicolumn{2}{l|}{\hskip2cm 5}    \\ \hline
		$ v\unit{(m/s)} $              & \multicolumn{2}{l|}{\hskip2cm 5}    \\ \hline
		$ z $                       & 21                       & 105     \\ \hline
		$ \alpha_{tw}\unit{(^\circ)} $ & \multicolumn{2}{l|}{\hskip2cm 20.65}        \\ \hline
		$ \beta_w\unit{(^\circ)} $ & \multicolumn{2}{l|}{\hskip2cm 19.09}        \\ \hline
	\end{tabular}
	\caption{Gearbox specifications}
\end{table}