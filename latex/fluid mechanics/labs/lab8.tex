\subsection{Questions}
\begin{enumerate}
	\item Theoretically,what is the condition to have cavitation phenomenon?\\
	Cavitation appears inside the liquid when static pressure there is  equal or less than saturated vapor pressure of that liquid at the same temperature.
	\item To conduct experiments, what tools do we use? How many times for 1 flow rate?
	\begin{itemize}[label=-]
		\item Stopwatch, volumetric flask.
		\item We measure 3 times for 1 flow rate.
	\end{itemize}
	\item How many cases do the experiment need conducting? How many times we measure in each case and what metric do we measure?
	\begin{itemize}[label=-]
		\item There are 2 cases: adjust valve 1 while valve 2 is fixed, and vice versa.
		\item We need to measure $Q$, $P_1$, $P_2$, $P_3$ before cavitation and after cavitation 3 times.
	\end{itemize}
\end{enumerate}

\subsection{Measurements and Calculations}
\begin{itemize}[label=-]
	\item Water temperature: $35 \degree C$
	\item Water specific weight: $994$ $kg/m^3$
	\item Saturated vapor pressure $P_v$ (Absolute pressure): $5622Pa$
	\item Atmospheric pressure $P_{at}$: $755mmHg = 100658.412 Pa$
	\item Saturated vapor pressure $P_{v(residue)}=P_v-P_{at}=-95036.412Pa$
\end{itemize}
{
	\newcolumntype{L}[1]{>{\raggedright\let\newline\\\arraybackslash\hspace{0pt}}m{#1}}
	\newcolumntype{C}[1]{>{\centering\let\newline\\\arraybackslash\hspace{0pt}}m{#1}}
	\setlength{\parskip}{1em}
	\setlength{\parindent}{0in}
	\renewcommand*\arraystretch{1.5}
\begin{table}[h]
	\centering
	\begin{tabular}{llllll}
		\hline
		\multicolumn{2}{l}{\multirow{2}{*}{}}  & Flow rate     & $P_1$  & $P_2$  & $P_3$  \\ \cline{3-6} 
		\multicolumn{2}{l}{}                   & $(\frac{l}{s})$ & $(10^5Pa)$ & $(10^5Pa)$ & $(10^5Pa)$ \\ \hline
		\multirow{3}{*}{Before cavitation} & 1 & 0.3652        & 1.1    & -0.72  & 0.053  \\
		& 2 & 0.3635        & 1.08   & -0.72  & 0.045  \\
		& 3 & 0.3630        & 1.11   & -0.72  & 0.067  \\ \hline
		\multicolumn{2}{l}{Average}            & 0.3639              &    1.097    &   -0.72     &    0.067    \\ \hline
		\multirow{3}{*}{After cavitation}  & 1 & 0.4541        & 1.9    & -0.94  & 0.63   \\
		& 2 & 0.4465        & 1.8    & -0.938 & 0.57   \\
		& 3 & 0.4345        & 1.75   & -0.22  & 0.54   \\ \hline
		\multicolumn{2}{l}{Average}            &      0.445         &   1.817     &    -0.93    &      0.58  \\ \hline
	\end{tabular}
	\caption{Measurements after changing valve 1}
	\label{tab1lab8}
\end{table}}
\newpage
{
	\newcolumntype{L}[1]{>{\raggedright\let\newline\\\arraybackslash\hspace{0pt}}m{#1}}
	\newcolumntype{C}[1]{>{\centering\let\newline\\\arraybackslash\hspace{0pt}}m{#1}}
	\setlength{\parskip}{1em}
	\setlength{\parindent}{0in}
	\renewcommand*\arraystretch{1.5}
\begin{table}[h]
	\centering
	\begin{tabular}{llllll}
		\hline
		\multicolumn{2}{l}{\multirow{2}{*}{}}  & Flow rate     & $P_1$  & $P_2$  & $P_3$  \\ \cline{3-6} 
		\multicolumn{2}{l}{}                   & $(\frac{l}{s})$ & $(10^5Pa)$ & $(10^5Pa)$ & $(10^5Pa)$ \\ \hline
		\multirow{3}{*}{Before cavitation} & 1 & 0.363         & 1.75   & -0.72  & 0.6    \\
		& 2 & 0.424         & 1.76   & -0.7   & 0.62   \\
		& 3 & 0.423         & 1.75   & -0.71  & 0.61   \\ \hline
		\multicolumn{2}{l}{Average}            & 0.403              &    1.753    &      -0.71  &    0.61    \\ \hline
		\multirow{3}{*}{After cavitation}  & 1 & 0.486         & 1.675  & -0.93  & 0.34   \\
		& 2 & 0.487         & 1.650  & -0.925 & 0.36   \\
		& 3 & 0.483         & 1.66   & -0.925 & 0.38   \\ \hline
		\multicolumn{2}{l}{Average}            &      0.485         &   1.662     &   -0.927    &  0.36      \\ \hline
	\end{tabular}
	\caption{Measurements after changing valve 2}
	\label{tab2lab8}
\end{table}}

{
	\newcolumntype{L}[1]{>{\raggedright\let\newline\\\arraybackslash\hspace{0pt}}m{#1}}
	\newcolumntype{C}[1]{>{\centering\let\newline\\\arraybackslash\hspace{0pt}}m{#1}}
	\setlength{\parskip}{1em}
	\setlength{\parindent}{0in}
	\renewcommand*\arraystretch{1.5}
\begin{table}[ht]
	\begin{tabular}{lllllllll}
		\hline
		\multirow{3}{*}{} & \multicolumn{4}{l}{Control through valve 1}               & \multicolumn{4}{l}{Control through valve 2}               \\ \cline{2-9} 
		& $Q$             & $P_1$          & $P_2$          & $h_1$ & $Q$             & $P_1$          & $P_2$          & $h_1$ \\
		& $(\frac{l}{s})$ & $(
		10^5Pa)$ & $(10^5Pa)$ & $(m)$ & $(\frac{l}{s})$ & $(10^5Pa)$ & $(10^5Pa)$ & $(m)$ \\ \hline
		Before cavitation & 0.3639          & 1.097          & -0.72          & 10.62 & 0.403           & 1.753          & -0.71          & 11.65 \\
		After cavitation  & 0.445           & 1.817          & -0.93          & 12.60 & 0.415           & 1.662          & -0.927         & 13.27 \\ \hline
	\end{tabular}
	\caption{Comparison between $Q$, $P_1$, $P_2$, $h_1$ for both cases}
	\label{tab3lab8}
\end{table}}

\subsection{Conclusion}
\begin{enumerate}[label=\alph*)]
	\item Describe the cavitation phenomenon.\\
	At balanced position of ventury has air bubbles with crackle sound. When adjust one of the valves, the air bubbles also change.
	\item Compare the magnitude of Q, P1, P2 and h1 in 2 cases, larger or smaller or equal?. Compare P2 and saturated vapor pressure and explain the difference between them (if they have).
	\begin{itemize}[label=-]
		\item About $Q$:
		\begin{itemize}[label=+]
			\item Adjust valve 1: the value of Q before and after the phenomenon have nearly same number.
			\item Adjust valve 2: the value of Q after cavitation larger than the previous one.
		\end{itemize}
		\item About $P_1$, $P_2$:
		\begin{itemize}[label=+]
			\item Adjust valve 1:\\
			$P_{1(before)}<P_{1(after)}$\\
			$P_{2(before)}>P_{2(after)}$
			\item Adjust valve 2:\\
			$P_{1(before)}>P_{1(after)}$\\
			$P_{2(before)}>P_{2(after)}$
		\end{itemize}
		
	\end{itemize}
	\item Explain why either have or not have, the difference of 3 parameters in 2 cases.\\
	Compare 2 cases we don’t see major differences between them but one difference is the flow rate that we have measured in case 2, the flow rate after cavitation is larger than the one in case 1. It proves that the damage in case 2 is more severed than in case 1. Proof is $h_1$ in case 1 $<$ $h_1$ in case 2.
	\item From that point, explain clearly how to control the flow rate to anti-cavitation and by which valve is more efficient.
	\begin{itemize}[label=-]
		\item We conclude that control the anti-cavitation flow rate  by valve 1 more efficient. By that way, we can slow the the velocity hence reduce the press at the balanced point (narrow section).
		\item Controlling by valve 2 is possible. However, since the volume of flow rate increases, the total energy loss is also more significant.
	\end{itemize}
	
\end{enumerate}