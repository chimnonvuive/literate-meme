\chapter{Modeling the motor}

\section{Electrical modeling}

In this modeling project, the motor is DC brushed type. The type of control is field control, which means the armature supply voltage is fixed. Before deriving the transfer function, it is worth to mention that all variables are in standard SI units.

The magnetic torque of the motor is generated by the interaction of the stator field and the rotor field and is given by
\begin{equation}\label{e1}
	T_m(t) = k_mi_g(t)
\end{equation}

where
\begin{itemize}
	\item $ T_m(t) $ is the motor torque.
	\item $ k_m $ is a constant. The value depends on the motor's characteristics (number of poles, number of conductors, etc.)
	\item $ i_g(t) $ is the rotor current (i.e. armature current).
\end{itemize}

%The back e.m.f is generated in the rotor windings (also known as armature), which follows Lenz's law. The value is given by
%\[
%v_b = k'i_f\omega_m
%\]
%
%where $ \omega_m $ is the angular speed of the motor.

%In principle, using law of conservation of energy, the power generated by electrical elements is equal to the power generated by mechanical elements:
%\[
%T_m(t) \dfrac{d\theta_m(t)}{dt} = i_gv_b
%\]
%
%where $ \omega_m $ is the angular speed of the motor.

Using Kirchoff's voltage law, the armature circuit equation is
\begin{equation}\label{e2}
v_g(t) = (R_g+R_f)i_g(t) + (L_g+L_f) \dfrac{di_g(t)}{dt}
\end{equation}

where
\begin{itemize}
	\item $ v_g $ is the supply voltage to the rotor (i.e. armature voltage). The value is constant since this is field control design.
	\item $ R_g $ is the resistance of the armature windings.
	\item $ R_f $ is the resistance of the field windings.
	\item $ L_g $ is the leakage inductance of the armature.
	\item $ L_f $ is the inductance of the field windings.
\end{itemize}

Assuming the generator voltage $ v_g(t) $ is proportional to the field current $ i_f(t) $, the relation between $ V_g(t) $ and $ V_f(t) $ using Laplace transform is
\begin{equation}\label{e3}
	V_g(s) = k_gI_f(s) = \dfrac{k_g}{R_f+L_fs}V_f(s)
\end{equation}

Using Laplace transform for equations \ref{e1} and \ref{e2} combining with \ref{e3}, the relation is obtained
\begin{equation}\label{e4}
	\dfrac{T_m(s)}{V_g(s)} = \dfrac{k_gk_m}{R_g+R_f + (L_g+L_f)s} = \dfrac{k_gk_m}{R_e+L_es}
\end{equation}

\section{Mechanical modeling}
The motor has its own inertia $ J_m $ and viscous damping coefficient $ b_m $. Therefore, equivalent parameters are necessary to form relations between the rotational output load $ \theta_L(t) $ and motor torque $ T_m(t) $.

From machine element course, it is known that the relation between torque, rotational displacement and number of teeth follows the formula (assuming negligible backlash):
\begin{equation}\label{e5}
	\dfrac{T_1}{T_2} = \dfrac{\theta_2}{\theta_1} = \dfrac{N_1}{N_2}
\end{equation}
where
\begin{itemize}
	\item $ T_1 $ is the torque generated by gear 1.
	\item $ T_2 $ is the torque generated by gear 2.
	\item $ \theta_2 $ is the rotational displacement of gear 2.
	\item $ \theta_1 $ is the rotational displacement of gear 1.
	\item $ N_1 $ is the number of teeth of gear 1.
	\item $ N_2 $ is the number of teeth of gear 2.
\end{itemize}

Using equation \ref{e5}, the motor and load is related by
\begin{equation}\label{e6}
	\dfrac{T_m(t)}{T_L(t)} = \dfrac{\theta_L(t)}{\theta_m(t)} = \dfrac{N_1}{N_2} = \dfrac{T_m(t)}{\left(J_e\dfrac{d^2\theta_L(t)}{dt^2} + b_e\dfrac{d\theta_L(t)}{dt}\right)} = n
\end{equation}
where
\begin{itemize}
	\item $ T_L(t) $ is the torque of the load.
	\item $ \theta_L(t) $ is the rotational displacement of the load.
	\item $ \theta_m(t) $ is the rotational displacement of the motor.
	\item $ J_e = J_m/n^2+J_L $ is the equivalent inertia of the motor and load. The inertia is adjusted with gear ratio $ N_2/N_1 $.
	\item $ b_e = b_m/n^2+b_L $ is the equivalent viscous damping coefficient of the motor and load. The inertia is adjusted with gear ratio $ N_2/N_1 $.
	\item $ n = N_1/N_2 $ is the gear ratio.
\end{itemize}

Using Laplace transform, equation \ref{e6} becomes
\begin{equation}\label{e7}
T_m(s) = \Theta_L(s) n\left(J_es^2 + b_es\right)
\end{equation}

\section{Forming the transfer function}
Combining equation \ref{e4} and \ref{e7}, the required transfer function is
\begin{equation}\label{e8}
	\dfrac{\Theta_L(s)}{V_f(s)} = \dfrac{k_gk_m/(nL_eJ_e)}{s(\dfrac{R_e}{L_e} + s)(s + \dfrac{b_e}{J_e})}
\end{equation}

This is the transfer function of the plant in the unity feedback system. For illustrating how designing a PID controller is implemented,  the parameters are chosen as follows (selected with consultations from the internet to resemble real-case scenario):
\begin{table}[ht]
	\centering
	\begin{tabular}{|c|c|}\hline
		$ k_g $ & 0.1\\\hline
		$ k_m $ & 1\\\hline
		$ L_f $ & 0.05\\\hline
		$ n $ & 1/10\\\hline
		$ J_L $ & 0.1\\\hline
		$ J_m $ & 0.01\\\hline
		$ b_L $ & 0.5\\\hline
		$ b_m $ & 0.5\\\hline
		$ R_f $ & 1000\\\hline
		$ R_g $ & 1000\\\hline
	\end{tabular}
\end{table}

Thus,\\
$ R_e = R_g + R_f = 1000 + 1000 = 2000 $\\
$ L_e = L_g + L_f = 0.025 + 0.025 = 0.05 $\\
$ J_e = J_m/n^2 + J_L = 0.01/0.1^2 + 0.1 = 1.1 $\\
$ b_e = J_m/n^2 + J_L = 0.5/0.1^2 + 0.5 = 5.5 $

and the transfer function is
\begin{equation}\label{e10}
	\dfrac{\Theta_L(s)}{V_f(s)} = \dfrac{18.2}{s\left(s+5\right)\left(s + 40000\right)}
\end{equation}

Since the pole $ s = -40000 $ has negligible effect compares to $ s=0 $ and $ s=-5 $, the second order approximation of the transfer function is

\begin{equation}\label{e11}
		\dfrac{\Theta_L(s)}{V_f(s)} = G(s) = \dfrac{18.2}{s\left(s+5\right)}
\end{equation}