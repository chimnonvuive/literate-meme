\chapter{Pressure and Fluid Statics}% TODO: \usepackage{graphicx} required

\subsubsection{Q2.1}
A differential manometer consists of a U-shaped pipe of diameter $ d $, connecting two cylinders of diameter $ D $, the  instrument being filled with two insoluble liquids of specific gravity $ \gamma_1 $ and $ \gamma_2  $. When the pressure difference, $ \Delta p = p_1-p_2 =0$, the interface between two liquids is at position $ 0 $ on the scale.
\begin{figure}[h]
	\centering
	\includegraphics[width=0.4\linewidth]{"2020-08-13 17.00.35 drive.google.com c4c67da5f09f"}
	\caption{Differential manometer}
	\label{fig:2020-08-13-17}
\end{figure}
\begin{enumerate}
	\item Determine the relationship between $ \Delta p $ and the displacement of the interface between the two fluids, $ h $. Given $ d=5\unit{mm},D=50\unit{mm}, \gamma_1=8530\unit{N/m^3},\gamma_2=8140\unit{N/m^3}, h=280\unit{mm}$.
	\item With the given $ \Delta p $, how many times $ h $ will decrease, if $ d=D=5\unit{mm} $.
\end{enumerate}
\paragraph{Ans:}$  $\\
Let $ h_1,h_2 $ be the height of the liquid from $ O $ on the left and right pipe, respectively.\\
The relation between $ P_1 $ and $ P_2 $ is described according to the formula:
\[\dfrac{D^2}{d^2}p_1=p_2\dfrac{D^2}{d^2}+\gamma_2(h_2-h)-\gamma_1(h_1-h)\]
Rearranging the equation yields:
\[h=\dfrac{\dfrac{D^2}{d^2}\Delta p +\gamma_1h_1-\gamma_2h_2}{\gamma_1-\gamma_2}\]

\subsubsection{Q2.2} A rectangular valve length $ b $ rotates about the horizontal axis at point $ A $. Neglecting valve thickness, determine the minimum weight $ G $ of the gate based on the parameters $  h1, h2, h3, \rho, b $ and $ g $ such that the system is balanced. Use figure \ref{fig:2} to solve the problem.
\begin{figure}[h]
	\centering
	\includegraphics[width=0.4\linewidth]{"2020-08-13 17.45.37 drive.google.com bc14c8504f54"}
	\caption{Rotating rectangular valve}
	\label{fig:2}
\end{figure}
