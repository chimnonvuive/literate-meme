\chapter{Chain Drive Design}
\section{Nomenclature}
\begin{tabular}[t]{p{0.05\linewidth}p{0.4\linewidth}}
	$ [i] $ & permissible impact times per second\\
	$ [s] $ & permissible safety factor\\
	$ [P] $ & permissible power,$ \unit{kW} $\\
	$ [\sigma_H] $ & permissible contact stress,$ \unit{MPa} $\\
	$ A $ & cross sectional area of chain hinge,$ \unit{mm^2} $\\
	$ a $ & real center distance,$ \unit{mm} $\\
	$ a_i $ & estimated center distance,$ \unit{mm} $\\
	$ a_{max} $ & maximum center distance,$ \unit{mm} $\\
	$ a_{min} $ & minimum center distance,$ \unit{mm} $\\
	$ B $ & width between inner link plate,$ \unit{mm} $\\
	$ d $ & chordal diameter,$ \unit{mm} $\\
	$ d_a $ & addendum diameter,$ \unit{mm} $\\
	$ d_f $ & dedendum diameter,$ \unit{mm} $\\
	$ d_l $ & roller diameter,$ \unit{mm} $\\
	$ d_O $ & pin diameter,$ \unit{mm} $\\
	$ E $ & modulus of elasticity,$ \unit{MPa} $\\
	$ F_0 $ & sagging force,$ \unit{N} $\\
\end{tabular}%
\begin{tabular}[t]{p{0.05\linewidth}p{0.4\linewidth}}
	$ F_1 $ & tight side tension force,$ \unit{N} $\\
	$ F_2 $ & slack side tension force,$ \unit{N} $\\
	$ F_r $ & force on the shaft,$ \unit{N} $\\
	$ F_t $ & effective peripheral force,$ \unit{N} $\\
	$ F_v $ & centrifugal force,$ \unit{N} $\\
	$ F_{vd} $ & contact force,$ \unit{N} $\\
	$ i $ & impact times per second\\
	$ K_d $ & weight distribution factor\\
	$ k $ & overall factor\\
	$ k_a $ & center distance and chain's length factor\\
	$ k_{bt} $ & lubrication factor\\
	$ k_c $ & rating factor\\
	$ k_d $ & dynamic load factor\\
	$ k_{dc} $ & chain tension  factor\\
	$ k_f $ & loosing factor\\
	$ k_n $ & coefficient of rotational speed\\
	$ k_r $ & number of tooth factor\\
	$ k_x $ & chain weight factor\\
	$ k_z $ & coefficient of number of teeth\\
\end{tabular}\newpage\noindent
\begin{tabular}[t]{p{0.05\linewidth}p{0.4\linewidth}}
	$ k_0 $ & arrangement of drive factor\\
	$ n $ & angular rotational speed, $ \unit{rpm} $\\
	$ n_{01} $ & experimental angular rotational speed,$ \unit{rpm} $\\
	$ P_t $ & calculated power,$ \unit{kW} $\\
	$ p $ & pitch,$ \unit{mm} $\\
	$ p_{max} $ & permissible sprocket pitch,$ \unit{mm} $\\
	$ Q $ & permissible load,$ \unit{N} $\\
	$ q $ & mass per unit length,$ \unit{kg/m} $\\
	$ s $ & safety factor\\
\end{tabular}%
\begin{tabular}[t]{p{0.05\linewidth}p{0.4\linewidth}}
	$ v $ & instantaneous velocity along the chain,$ \unit{m/s} $\\
	$ x $ & chain length in pitches, the number of links\\
	$ x_c $ & an even number of links\\
	$ z $ & number of teeth of a sprocket\\
	$ z_{max} $ & maximum number of teeth of the driven sprocket\\
	$ \sigma_H $ & contact stress,$ \unit{MPa} $\\
	$ _1 $  & subscript for driving sprocket\\
	$ _2 $  & subscript for driven sprocket\\
\end{tabular}

\textbf{Known parameters} From Chapter 1, we know that:\\
The chain type is roller.\\
$ n_{sh3}= 132.27 \unit{rpm}$\\
$ P_{ch} = 6.26\unit{kW}, u_{ch}=4.5 $

\section{Find the chain drive pitch}
The driving sprocket is connected to shaft 3, $ n_1 = n_{sh3} = 132.27\unit{rpm} $.

\textbf{Calculate $ z $}
The number of teeth determines the how stable is the rotational speed, which relates to the impact intensity and service life:\\
$ z_1 = 29 - 2u_{ch} = 29 - 2 \times 2.03 = 25 \geq 19 $ (rounded to the nearest odd number)\\
$ z_2 = u_{ch}z_1 = 2.03 \times 25 = 51 \leq 120$ (Equation 5.1 \cite{tk1})

\textbf{Calculate $ [P] $} An experiment is conducted to find the optimal pitch given the permissible power and angular rotational speed. A roller chain drive having $ 25 $ teeth on the driving sprocket is tested in 8 different cases of $ n_{01} $ in somewhat similar condition with our design purpose, p.80 \cite{tk1}. In this problem, $ z_1=25 $; $ n_1=132.27\unit{rpm} $, which is close to $ n_{01}=200\unit{rpm} $, which yields $ k_z = {25}/{z_1} = 25/25 = 1 $ and $  k_n = {n_{01}}/{n_{1}} = 200/132.27 = 1.51 $. 

Another step is to specify the working condition of the chain, Table 5.6:
\begin{itemize}
	\item The centerline between 2 sprockets is parallel with the ground. $ k_0 = 1 $
	\item Center distance $ a= (30\div 50)p $, which is similar to the experiment. $ k_a= 1 $
	\item Center distance is modifiable through displacing the sprockets. $ k_{dc}=1 $
	\item Moderate impact load. $ k_d=1.5 $
	\item 1 shift. $ k_c=1 $
	\item Dusty condition with moderate lubrication quality. $ k_{bt} =1.3$
\end{itemize}
Then, we can obtain the value $ [P] $, Equation 5.3 \cite{tk1}:
\begin{multline*}
	P_t=P_{ch}k_0k_ak_{dc}k_dk_ck_{bt}k_zk_n\\
	=6.26\times1\times1\times1\times1.5\times1\times1.3\times1\times1.51=18.46\unitp{kW}\leq[P]
\end{multline*}

Inspecting Table 5.5 \cite{tk1} at column $ n_{01}=200\unit{rpm} $, the closet value is $ [P]=19.3\unit{kW} $. Knowing $ [P] $, we have $ p= 31.75 \unit{mm} $, Table 5.5 \cite{tk1}. Consequently, $ d_c=9.55\unit{mm} $, $ B=27.46\unit{mm} $. Consulting Table 5.8 \cite{tk1}, the pitch is indeed suitable.

\section{Determine basic parameters of the chain drive}
\subsection{Find number of links and center distance}
The parameters are found:

\textbf{Find $ x_c $} The  $ a_{min} = 30p = 30\times 31.75= 952.50 \unit{(mm)} $, $ a_{max} = 50p = 50\times 31.75 = 1587.50 \unit{(mm)}$. Limiting the range of choice for $ a $ in $ [a_{min},a_{max}] $, we can approximate $ a = 1300 \unit{mm} $ and find $ x_c $:
\[x = \dfrac{2a}{p} + \dfrac{z_1+z_2}{2} + \dfrac{(z_2-z_1)^2p}{4\pi^2a}\\ = \dfrac{2\times 1300}{31.75} + \dfrac{25+51}{2} + \dfrac{(51-25)^2\times 31.75}{4\pi^2\times 1300}=120.31\]
%\begin{multline*}
%	
%\end{multline*}

Then, round $ x $ up to the nearest even number gives $ x_c = 122$.

\textbf{Find $ a $}
Using $ x_c $ to find the correct center distance, Equation 5.13 \cite{tk1}. In addition, it is recommended to loose the chain an amount of $ 0.002\div0.004 a $ to reduce tension. Choosing the amount of $ 0.003a $, the coefficient $ 0.997 $ is multiplied in the formula below:
\begin{multline*}
	a = \dfrac{0.997p}{4}\left[x_c-\dfrac{z_2+z_1}{2}+\sqrt{\left(x_c-\dfrac{z_2+z_1}{2}\right)^2-2\left(\dfrac{z_2-z_1}{\pi}\right)^2}\right]\\ = \dfrac{0.997\times31.75}{4}\left[122-\dfrac{51+25}{2}+\sqrt{\left(122-\dfrac{51+25}{2}\right)^2-2\left(\dfrac{51-25}{\pi}\right)^2}\right]= 1019.99 \unitp{mm}
\end{multline*}

\textbf{Other parameters} The values below are necessary for modeling the chain drive:\\ 
$ d_1 = p/\sin\left(\dfrac{\pi}{z_1}\right) = 31.75/\sin\left(\dfrac{180}{25}\right) = 253.32\unitp{mm}$\\
$ d_2 = p/\sin\left(\dfrac{\pi}{z_2}\right) = 31.75/\sin\left(\dfrac{180}{51}\right) = 515.75\unitp{mm}$\\
$ d_{a1} = p\left( 0.5 + \cot \dfrac{180}{z_1} \right) = 31.75 \left( 0.5 + \cot \dfrac{180}{25} \right) = 267.20 \unitp{mm}$\\
$ d_{a2} = p\left( 0.5 + \cot \dfrac{180}{z_2} \right) = 31.75 \left( 0.5 + \cot \dfrac{180}{51} \right) =  530.65 \unitp{mm}$

Look up to find $ d_l = 19.05 \unit{(mm)} $, see Table 5.2 \cite{tk1}:\\
$ d_{f1} = d_1 - 2(0.502d_l+0.05) = 253.32 - 2(0.502\times 19.05+0.05) = 234.08 \unitp{mm}$\\
$ d_{f2} = d_2 - 2(0.502d_l+0.05) = 515.75 - 2(0.502\times 19.05+0.05) = 496.50 \unitp{mm}$

\section{Strength of chain drive}
\subsection{Impact frequency analysis}
After determine the center distance, it is necessary to validate the impact frequency. From Table 5.9 \cite{tk1}, it  is $ [i]=25 $. Calculating $ i $ gives:
\[i=\dfrac{z_1n_1}{15x}=\dfrac{25\times 132.27}{15\times 120.31}=1.83<[i]\]

\subsection{Safety factor analysis}
In order to operate safely, the chain drive's safety factor must satisfy the following condition:
\[s = \dfrac{Q}{k_dF_t+F_0+F_v} \geq [s]\]
Rotational speed of the smaller sprocket is determined using the formula below:
\[v_1=\dfrac{n_1pz_1}{60000}= 2.6 \unitp{m/s}\]

\textbf{Find $ k_d $}: Assuming moderate workload, choose $ k_d = 1.2 $.

\textbf{Find $ F_t $, $ F_v $ and $ F_0 $}: Knowing $ p $, it is easy to look up the values $ Q=56700 \unit{N}$ and $ q=2.6\unit{kg/m} $ from Table 5.2 \cite{tk1}:\\
$ F_t = {10^3P_{ch}}/{v_1} = 2410.48 \unitp{N} $\\
$ F_v=qv_1^2= 17.54 \unitp{N} $\\
$ F_0=9.81\times10^{-3}k_fqa = 156.11\unitp{N}$

\textbf{Find $ k_f $}: Let the chain drive be parallel to the ground, we get $ k_f = 6 $.

\textbf{Find $ [s] $}: The limit $ [s] =8.71$ is found using interpolation, Table 5.10 \cite{tk1}.

Replacing all the variables gives:
\[s=18.49\geq8.71\]
which satisfies the condition.
\subsection{Contact stress analysis}
The following condition must be met, Equation 5.18 \cite{tk1}:
\[\sigma_H = 0.47\sqrt{\dfrac{k_r(F_tk_d+F_{vd})E}{AK_d}}\leq[\sigma_H]\]
Since the chain drive only has one strand, $ K_d = 1$.

\textbf{Find $ [\sigma_H] $} Quenched 45 steel is the material of use for the chain drive, which has $ \text{HB}210 $, $ [\sigma_H] = 600 \unit{(MPa)} $ and $ E = 2.1\times10^5 \unit{(MPa)} $, see Table 5.11 \cite{tk1}.

\textbf{Find $ F_{vd} $} For 1-strand chain, $ F_{vd} = 13\times10^{-7}n_1p^3 = 6.22 \unit{(N)}$

\textbf{Find $ k_r $} Since $ z_1 $ is used to estimate $ k_r $, $ k_r= 0.47$.

\textbf{Find E} Assuming the sprockets and chain are made up from the same material (steel), $ E=2.1\times10^5\unit{MPa} $

\textbf{Find $ A $} From the given parameters and value $ p $, the area $ A = 180 \unit{mm^2} $, Table 5.12 \cite{tk1}.

Knowing $ k_d $ and $ F_t $, we get the result:
\[\sigma_H = 591.29 \unit{(MPa)} \leq 600 \unit{MPa}\]
which is satisfactory.

\section{Force on shaft}
Applying the following equations, see p.87 \cite{tk1}:\\
$ F_2 = F_0 + F_v = 173.65 \unit{(N)}$\\
$ F_1 = F_t + F_2 = 2584.13 \unit{(N)}$\\
Choose $ k_x=1.15 $ to obtain $ F_r $, Equation 5.20 \cite{tk1}:\\
$ F_r = k_xF_t = 2772.05\unit{(N)} $

\section{Other parameters}

In summary, we have the following table:

\begin{table}[ht]
	\centering
	\begin{tabular}{|
			>{\columncolor[HTML]{C0C0C0}}l |p{2.5cm}|p{2.5cm}|}
		\hline
		& \multicolumn{1}{c|}{\cellcolor[HTML]{C0C0C0}driving} & \multicolumn{1}{c|}{\cellcolor[HTML]{C0C0C0}driven} \\ \hline
		$ [P] \unit{(kW)} $      & \multicolumn{2}{l|}{\hskip2cm 11}       \\ \hline
		$ a\unit{(mm)} $              & \multicolumn{2}{l|}{\hskip2cm 1019.99}    \\ \hline
		$ B\unit{(mm)} $              & \multicolumn{2}{l|}{\hskip2cm 22.61}    \\ \hline
		$ d\unit{(mm)} $              & 170.92                   & 768.22 \\ \hline
		$ d_a\unit{(mm)} $              & 181.22                    & 780.50 \\ \hline
		$ d_f\unit{(mm)} $              & 154.36                    & 752.16 \\ \hline
		$ d_l\unit{(mm)} $              & \multicolumn{2}{l|}{\hskip2cm 15.88}    \\ \hline
		$ d_O\unit{(mm)} $              & \multicolumn{2}{l|}{\hskip2cm 7.95}    \\ \hline
		$ i $            & \multicolumn{2}{l|}{\hskip2cm 2.92}          \\ \hline
		$ p\unit{(mm)} $            & \multicolumn{2}{l|}{\hskip2cm 25.4}           \\ \hline
		$ Q \unit{(N)} $      & \multicolumn{2}{l|}{\hskip2cm 56700}      \\ \hline
		$ u_{ch} $              & \multicolumn{2}{l|}{\hskip2cm 4.5}    \\ \hline
		$ v\unit{(m/s)} $              & \multicolumn{2}{l|}{\hskip2cm 2.6}    \\ \hline
		$ z $                       & 21                       & 95     \\ \hline
	\end{tabular}
	\caption{Chain drive specifications}
\end{table}