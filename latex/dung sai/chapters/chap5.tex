\chapter{ĐO LỖ CÔN THEO PHƯƠNG PHÁP ĐO GIÁN TIẾP}

\section{Mục đích thí nghiệm}
\begin{itemize}
	\item Tìm hiểu sơ bộ kết cấu máy dựa trên nguyên tắc quang cơ, biết sử dụng máy để đo kích thước ngoài.
	\item Nắm được nguyên tắc dùng bi cầu để đo lỗ côn.
\end{itemize}

\section{Các bước tiến hành}

\begin{table}[ht]
	\centering
	\caption{Bảng đo các thông số ($ n=5 $)}
	\begin{tabular}{lllllll}\toprule
		Thông số & Lần 1 & Lần 2 & Lần 3 & Lần 4 & Lần 5 & Trung bình \\\midrule
		$ D $ & 33.08 & 33.4 & 33.09 & 33.03 & 33.10 & 33.14\\
		$ d $ & 23.79 & 23.71 & 23.69 & 23.61 & 23.63 & 23.686\\
		$ h_1 $ & 23.604 & 23.6045 & 23.6375 & 23.627 & 23.155 & 23.5256\\
		$ h_2 $ & 68.102 & 68.042 & 68.062 & 68.092 & 68.074 & 68.0744\\\bottomrule				
	\end{tabular}
\end{table}

Giá trị trung bình:\\
$ \overline{\alpha} = 6.817^\circ $\\

Giá trị độ lệch chuẩn:\\
$\sigma_\alpha = 0.113^\circ $\\

Kết quả thí nghiệm cho thấy độ côn của lỗ là $ \alpha = 6.817 \pm 0.113 ^\circ $.

\section{Nhận xét và đánh giá kết quả đo}
Kết quả đo được có độ chính xác khá cao bởi vì:
\begin{itemize}
	\item Xác định đường kính viên bi bằng thước Banme 0.01mm và dung nguyên tắc ABBE nên kết quả khá chính xác.
	\item Dùng thước Đơ li nô mét chính xác đến 0.001mm nên các kích thước $ h_1 $ và $ h_2 $ cũng chính xác đến $ \mu m $. 
	\item Với cách đo gián tiếp bằng máy đờlinnômét ta có góc côn và miền sai số, miền sai số này rất nhỏ so với giá trị góc côn nên ta đạt được độ chính xác cao.
\end{itemize}
Nguyên nhân sai số : Đây là phương pháp đo gián tiếp góc nghiêng của lỗ thông qua việc đo trực tiếp các thông số $ D,d,h_1,h_2 $, vì vậy dẫn đến sai số trong quá trình đó trực tiếp , đồng thời sai số làm tròn trong công thức tính toán dẫn đến sai số cuối cùng của kết quả đo.