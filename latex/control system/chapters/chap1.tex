\chapter{Modeling the motor}

\section{Electrical modeling}

In this modeling project, the motor is DC brushed type. The type of control is field control, which means the armature supply voltage is fixed.

The magnetic torque of the motor is generated by the interaction of the stator field and the rotor field and is given by
\begin{equation}\label{e1}
	T_m(t) = ki_gi_f(t)
\end{equation}

where
\begin{itemize}
	\item $ T_m(t) $ is the motor torque.
	\item $ i_g $ is the rotor current (i.e. armature current).
	\item $ i_f(t) $ is the stator current (i.e. field current).
	\item $ k $ is a constant. The value depends on the motor's characteristics.
\end{itemize}

Since the supply voltage for the armature is constant, the product $ ki_g $ is replaced with $ k_g $ to simplify the equation. Thus, equation \ref{e1} becomes $ T_m(t) = k_g i_f(t) $.
%The back e.m.f is generated in the rotor windings (also known as armature), which follows Lenz's law. The value is given by
%\[
%v_b = k'i_f\omega_m
%\]
%
%where $ \omega_m $ is the angular speed of the motor.

%In principle, using law of conservation of energy, the power generated by electrical elements is equal to the power generated by mechanical elements:
%\[
%T_m(t) \dfrac{d\theta_m(t)}{dt} = i_gv_b
%\]
%
%where $ \omega_m $ is the angular speed of the motor.

The field circuit equation is
\begin{equation}\label{e2}
	v_f(t) = R_fi_f(t) + L_f \dfrac{di_f(t)}{dt}
\end{equation}

where $ v_f(t) $ is the supply voltage to the stator; $ R_f $ is the resistance of the field windings; $ L_f $ is the inductance of the field windings.

Substituting $ i_f(t) $ in equation \ref{e2} with $ T_m(t) $ in equation \ref{e1} yields
\begin{equation}\label{e3}
	v_f(t) = R_f\dfrac{T_m(t)}{k_g} + \dfrac{L_f}{k_g}\dfrac{dT_m(t)}{dt}
\end{equation}

Using Laplace transform, equation \ref{e3} becomes
\begin{equation}\label{e4}
	V_f(s) = T_m(s)\left(\dfrac{L_f}{k_g}s + \dfrac{R_f}{k_g}\right)
\end{equation}

\section{Mechanical modeling}
The motor has its own inertia $ J_m $ and viscous damping coefficient $ b_m $. Therefore, equivalent parameters are necessary to form relations between the rotational output load $ \theta_L(t) $ and motor torque $ T_m(t) $.

From machine element course, it is known that the relation between torque, rotational displacement and number of teeth follows the formula (assuming negligible backlash):
\begin{equation}\label{e5}
	\dfrac{T_1}{T_2} = \dfrac{\theta_2}{\theta_1} = \dfrac{N_1}{N_2}
\end{equation}
where
\begin{itemize}
	\item $ T_1 $ is the torque generated by gear 1.
	\item $ T_2 $ is the torque generated by gear 2.
	\item $ \theta_2 $ is the rotational displacement of gear 2.
	\item $ \theta_1 $ is the rotational displacement of gear 1.
	\item $ N_1 $ is the number of teeth of gear 1.
	\item $ N_2 $ is the number of teeth of gear 2.
\end{itemize}

Using equation \ref{e5}, the motor and load is related by
\begin{equation}\label{e6}
	\dfrac{T_m(t)}{T_L(t)} = \dfrac{\theta_L(t)}{\theta_m(t)} = \dfrac{N_1}{N_2} = \dfrac{T_m(t)}{\left(J_e\dfrac{d^2\theta_L(t)}{dt^2} + b_e\dfrac{d\theta_L(t)}{dt}\right)} = n
\end{equation}
where
\begin{itemize}
	\item $ T_L(t) $ is the torque of the load.
	\item $ \theta_L(t) $ is the rotational displacement of the load.
	\item $ \theta_m(t) $ is the rotational displacement of the motor.
	\item $ J_e = J_m\left({N_2}/{N_1}\right)^2+J_L $ is the equivalent inertia of the motor and load. The inertia is adjusted with gear ratio $ N_2/N_1 $.
	\item $ b_e = b_m\left({N_2}/{N_1}\right)^2+b_L $ is the equivalent viscous damping coefficient of the motor and load. The inertia is adjusted with gear ratio $ N_2/N_1 $.
	\item $ n = N_1/N_2 $ is the gear ratio.
\end{itemize}

Using Laplace transform, equation \ref{e6} becomes
\begin{equation}\label{e7}
T_m(s) = \Theta_L(s) n\left(J_es^2 + b_es\right)
\end{equation}

Combining equation \ref{e4} and \ref{e7}, the required transfer function is
\begin{equation}\label{e8}
	V_f(s) = \Theta_L(s) n\left(J_es^2 + b_es\right)\left(\dfrac{L_f}{k_g}s + \dfrac{R_f}{k_g}\right)
\end{equation}

Rewriting in terms of fraction yields
\begin{equation}\label{e9}
\dfrac{\Theta_L(s)}{V_f(s)} = \dfrac{\dfrac{k_g}{L_fnJ_e}}{s\left(s+\dfrac{b_e}{J_e}\right)\left(s + \dfrac{R_f}{L_f}\right)}
\end{equation}

This is the transfer function of the plant in the unity feedback system. For illustrating how PID design is implemented, 

