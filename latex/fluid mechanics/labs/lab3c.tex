\subsection{Questions}
\begin{enumerate}
	\item Is the water level $L$ inside the nanometer tube equal to the value of pressure $\displaystyle L=\frac{\rho}{\gamma}+z$? Explain.\\
	Yes. The position of experimental system is $z_i=0$, thus $\displaystyle \frac{\rho}{\gamma}=\frac{\rho}{\gamma}+z$.
	\item How to change the flow rate inside the pipe?\\
	The lab equipment consists of 1 venturi meter with diameter $d$ varies from $24mm$ (section 1-1, 3-3) to $14mm$ (section 2-2). Use valve 1 and valve 2 to adjust the flow rate inside the pipe.
	\item  When the flow rate is constant but the cross-section of the pipe changes correspondingly, how does velocity change?\\
	When the flow rate is kept constant and goes from small to large cross-sectional area, the flow's velocity at time $t$ decreases. Accordingly, its total kinetic energy also declines.
\end{enumerate}

\subsection{Measurements}
\paragraph{}
Measure the water level $Z$ of 3 venturi meters corresponding to 3 flow rate states. The results are in Table \ref{tab1lab3c}:
\begin{table}[ht]
	\centering
	\begin{tabular}{|c|p{1.5cm}|p{1.5cm}|p{1.5cm}|c|c|}
		\hline
		\multirow{2}{*}{State of flow rate} & \multicolumn{3}{c|}{Water level Z on 3 venturi meters} & \multirow{2}{100pt}{Water level inside the flow meter ($mm$)} & \multirow{2}{*}{$Q$ ($\frac{l}{min}$)} \\
		\cline{2-4}
		& 1 & 2 & 3 & & \\ \hline
		1 & 140 & 50 & 125 & 209 & 8.58 \\ \hline
		2 & 160 & 60 & 140 & 219 & 9.36 \\ \hline
		3 & 190 & 70 & 170 & 230 & 9.906 \\ \hline
	\end{tabular}
	\caption{Measurement for 3 states of flow rate}
	\label{tab1lab3c}
\end{table}

\subsection{Calculations}
\begin{enumerate}
	\item Determine the pressure difference between cross-sections 1-2 and 3-2. Calculate recovery factor $R$ according to formula (3.5). Write down the result into Table \ref{tab2lab3c}.\\
	\begin{table}[h]
		\centering
		\begin{tabular}{|c|c|c|c|}
			\hline
			Measurement no. & 1 & 2 & 3	\\ \hline
			$Z_3-Z_2$ 		& 75& 80 & 100 	\\ \hline
			$Z_1-Z_2$ 		& 90	& 100	& 120 	\\ \hline
			$R$ 			& 83.3\%	& 80\% 	& 83.3\%	\\ \hline
		\end{tabular}
		\caption{Pressure difference and recovery factor}
		\label{tab2lab3c}
	\end{table}
	\item With 3 values measured from venturi meter, determine the calculated flow rate $Q_c$ inside the pipe using formula (3.8) and the measured flow rate $Q_m$ using flow meter. From the results, find the coefficient venturi C and write them down into Table \ref{tab3lab3c}.
	\newpage
	\begin{table}[h]
		\centering
		\begin{tabular}{|c|c|c|c|}
			\hline
			State of flow rate 		&  1  	&  2  	&  3 \\ \hline
			$Q_c$ (l/s)				&	0.143	&	0.156	&0.165	 \\ \hline
			$Q_m$ (l/s)				&	0.218	&0.229		&	0.251	\\ \hline
			$C = \frac{Q_c}{Q_m}$	&	0.656	&0.681		&0.657	\\	\hline
		\end{tabular}
		\caption{Coefficient venturi C}
		\label{tab3lab3c}
	\end{table}
	From Table \ref{tab3lab3c}, $C_{avg}=0.66467$
\end{enumerate}

\subsection{Conclusion}
\begin{enumerate}[label=\alph*)]
	\item Are the results of $R$ and $C$ reasonable? Explain.\\
	The values of $R$ and $C$ are reasonable because after 3 times changing the flow rate $Q$, we measure approximately the same in each measurement. This is a non-ideal flow so there is friction loss, which explains why the value of $R<1$ and $C<1$. Only when the flow was an ideal flow did $C =1$.
	\item Are $R$ and $C$ dependent on the flow rate $Q$? Explain.\\
	$R$ and $C$ depend on the flowrate $Q$ because from the beginning of the cross section 1-1 to 2-2, we have energy loss from the shrinkage of the channel. From section 2-2 to 3-3, energy loss due to the widening of the channel. Base on 2 losses, we evaluate the recovering ability of the cylindrical water column at section 3-3 in comparison with section 1-1 equal the recovery coefficient $R$ and the differences of pressure level on the multitube nanometer, and the pressure difference depend on the flowrate $R$.
\end{enumerate}