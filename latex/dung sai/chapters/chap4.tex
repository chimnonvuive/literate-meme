\chapter{XÁC ĐỊNH KÍCH THƯỚC MẪU}

\section{Mục đích thí nghiệm}
\begin{itemize}
	\item Biêt sử dụng đồng hồ so.
	\item Biêt sử dụng các loại mẫu đo.
	\item Biết lựa chọn mẫu và bảo quản mẫu.
\end{itemize}

\section{Các dụng cụ}
Đồng hồ so, bộ gá đồng hồ so có mặt phẳng chuẩn.

\section{Các bước tiến hành}

Chi tiết được chọn mang số hiệu 5, có kích thước $ A = 69.963 $, $ B = 59.679 $, $ C = 50.738 $. Độ chính xác của kích thước được quy định $ A^{\pm 0.04} $, $ B^{\pm 0.05} $, $ C^{\pm 0.06} $.

\begin{table}[ht]
	\centering
	\caption{Bảng đo độ chính xác của các kích thước $ A^{\pm 0.04} $, $ B^{\pm 0.05} $, $ C^{\pm 0.06} $}
	\begin{tabular}{p{0.11\linewidth}lll}\toprule
		\multirow{2}{\linewidth}{Chi tiết số 5} & \multicolumn{3}{l}{Sai số}\\ \cmidrule{2-4}
		& $ A $ & $ B $ & $ C $\\\midrule
		1 & 0.08 & 0.06 & 0.02\\
		2 & 0.08 & 0.06 & 0.02\\
		3 & 0.07 & 0.06 & 0.02\\
		4 & 0.06 & 0.06 & 0.02\\
		5 & 0.06 & 0.06 & 0.02\\\bottomrule
	\end{tabular}
\end{table}

Các căn mẫu tổ hợp theo yêu cầu, sinh viên ghi lại cách chọn của mình.\\
\[
\begin{array}{l}
A = 40 + 20 + 8.5 + 1.3 = 69.8\\
B = 40 + 10 + 8.5 + 1.3 = 59.8\\
C = 40 + 8.5 + 1.2 = 49.7\\
\end{array}
\]

So với mẫu đo, các kích thước lần lượt có độ sai số là:
\[
\begin{array}{l}
A = 69.963 - 69.8 = 0.163\\
B = 59.679 - 59.8 = -0.121\\
C = 50.738 - 49.7 = 0.038\\
\end{array}
\]

Sau khi bù trừ phần lệch được bảng sau:

\begin{table}[ht]
	\centering
	\caption{Bảng sai số sau khi bù trừ phần lệch căn mẫu so với kích thước cần kiểm tra}
	\begin{tabular}{p{0.11\linewidth}lll}\toprule
		\multirow{2}{\linewidth}{Chi tiết số 5} & \multicolumn{3}{l}{Sai số}\\ \cmidrule{2-4}
		& $ A $ & $ B $ & $ C $\\\midrule
		1 & 0.243 & -0.061 & 0.058\\
		2 & 0.243 & -0.061 & 0.058\\
		3 & 0.233 & -0.061 & 0.058\\
		4 & 0.223 & -0.061 & 0.058\\
		5 & 0.223 & -0.061 & 0.058\\\bottomrule
	\end{tabular}
\end{table}

\section{Đánh giá và nhận xét kết quả đo}
Từ số liệu cho ở bảng kết quả:
\begin{itemize}
	\item Mặt  A,  B  vượt quá miền dung sai cho phép. Cả 2 mặt đều không đạt yêu cầu.
	\item Mặt  C không vượt quá miền dung sai cho phép. Mặt C đạt yêu cầu.
\end{itemize}

Do vậy, chi tiết không đạt yêu cầu.

Nguyên nhân sai lệch: sai số chủ quan do người đo.

Khi ghép hai mẫu với nhau: ta lau sạch, xoa 2 mặt làm việc nhẹ nhàng lên nhau, sao cho chúng dính lại, mục đích để việc đo đạt được chính xác tránh sai số do khe hở các mẫu tạo nên.