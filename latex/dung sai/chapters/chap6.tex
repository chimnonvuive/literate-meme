\chapter{ĐO ĐỘ ĐẢO VÀNH RĂNG}

\section{Mục đích thí nghiệm}
\begin{itemize}
	\item Biết cách đo độ đảo hướng tâm nói chung, trên cơ sở đo độ đảo vành răng.
	\item Là một trong các yếu tố quan trọng về độ chính xác động học của bánh răng.
	\item Biết xử lý về đầu đo khi gặp bề mặt phức tạp.
\end{itemize}

\section{Dụng cụ và vật đo}
\begin{itemize}
	\item Một bánh răng có $ m=2\div 3 $, $ z=20\div 25 $.
	\item Đồng hồ so 0.01mm.
	\item Đồ gá đồng hồ so.
	\item Bàn máp.
	\item Đồ gá chống tâm.
	\item Một trục gá mài có độ ô van 0.005 và lắp xít với lỗ bánh răng.
	\item Một con lăn có kích thước thích hợp.
\end{itemize}

\section{Các bước tiến hành}
\begin{table}[ht]
	\centering
	\caption{Bảng răng số hiệu}
	\begin{tabular}{llllll}\toprule
		STT & Giá trị $ R $ & STT & Giá trị $ R $ & STT & Giá trị $ R $\\\midrule
		1 & 0.00 & 9 & 0.16 & 17 & -0.14 \\
		2 & 0.05 & 10 & 0.18 & 18 & -0.16\\
		3 & 0.06 & 11 & 0.10 & 19 & -0.11\\
		4 & 0.11 & 12 & 0.07 & 20 & -0.16\\
		5 & 0.16 & 13 & -0.01 & 21 & -0.14 \\
		6 & 0.17 & 14 & -0.05 & 22 & -0.13\\
		7 & 0.08 & 15 & -0.16 & 23 & -0.10\\
		8 & 0.16 & 16 & -0.11 & 24 & -0.09\\
		\bottomrule
	\end{tabular}
\end{table}

Độ đảo hướng tâm $ R_{max} - R_{min} = 0.18 - (-0.16) = 0.34 \unit{mm} $

\section{Nhận xét kết quả}
\begin{itemize}
	\item Độ đảo hướng tâm của vành răng, là sự thay đổi lớn nhất khoảng cách từ tâm quay đến đường chia của răng, sau một vòng quay.
	\item Độ chính xác của phép đo còn phụ thuộc việc chọn con lăn. Tâm của con lăn
	cần phải nằm trên vòng chia của bánh răng thì mới chính xác.
	\item Phép đo độ đảo vành răng đơn giản, dễ thực hiện và chính xác.
	\item Các sai số trong phép đo:
	\begin{itemize}
		\item Sai số động học của bánh răng, là sai lệch lớn nhất về góc quay của bánh răng sau một vòng quay, khi nó ăn khớp với bánh răng mẫu chính xác.
		\item Độ dao động khoảng cách tâm đo sau một vòng, là sự thay đổi lớn nhất của khoảng cách tâm giữa bánh răng có sai số (bánh răng đo) và bánh răng mẫu chính xác ăn khớp khít với nhau, khi quay bánh răng đo đi một vòng. 
		\item Sai số tích luỹ bước răng , là hiệu đại số lớn nhất của các giá trị sai số tích luỹ $ k $ bước răng, với tất cả các giá trị $ k $ từ $ 2 $ đến $ z/2 $ ($ z $ là số răng của bánh răng).
		\item Sai số lăn răng, là sai số lớn nhất về góc quay giữa bánh răng gia công và dụng cụ cắt răng (dao phay răng).
		\item Dao động khoảng pháp tuyến chung, là sự dịch chuyển biên dạng răng theo hướng tiếp tuyến trong phạm vi một vòng quay của bánh răng.
	\end{itemize}
\end{itemize}