\chapter{Gear Design (Stage One)}

\section{Material selection}
For small scale machines, type I material, which is often quenched and has low hardness ($ \text{HB}\leq 350 $), is sufficient. Precise manufacture is possible albeit low permissible stresses. In this design problem, the gear drive uses quenched 45 steel, whose specification is displayed on Table 6.1 \cite{tk1}:
\begin{itemize}
	\item Brinell hardness $ \text{HB} 205 $
	\item ultimate strength $ \sigma_b = 600\unit{MPa} $
	\item yield limit $ \sigma_{ch} = 340 \unit{MPa}$
\end{itemize}

\section{Stress estimation}
From working condition, the service life in hours are found:
\[L_h = 8\unit{\left( \dfrac{hours}{shift}\right)}\times Ca  K_{ng} L= 8 \times 1\times 260\times 8 = 16640\unitp{hours}\]
where all parameters are given in the design problem.

Since many variables are different in both gears, it should be convenient to introduce 4 subscripts for the parameters. Unless otherwise stated,
\begin{itemize}
	\item subscript $ _1 $ denotes the pinion attached to shaft 1.
	\item subscript $ _2 $ denotes the driven gear attached to shaft 2.
	\item subscript $ _F $ denotes bending stress.
	\item subscript $ _H $ denotes contact stress.
\end{itemize}
\subsection{Estimate the permissible contact stress $ [\sigma_H] $ and bending stress $ [\sigma_F] $}
The permissible stresses are initially estimated using Equation 6.1a and 6.2a \cite{tk1}. The exact calculation will be considered in the next section:
\[[\sigma_H]=\dfrac{\sigma_{Hlim}^o}{S_H}K_{HL}\]
\[[\sigma_F]=\dfrac{\sigma_{Flim}^o}{S_F}K_{FC}K_{FL}\]

%Preliminary estimation of these values assumes $ Z_RZ_vK_{xH} = 1 $ and $ Y_RY_sK_{xF} = 1 $. These factors will be accounted for in the next sections.

\paragraph{Find permissible stress corresponding to a working cycle $ \sigma_{lim}^o $}
The value depends on hardness. The pinion also should be harder than the driving gear $ \text{HB} 10 \div 15 $. For quenched 45 steel, the formulas below are used, Table 6.2 \cite{tk1}:
\[ \sigma_{Hlim}^o = 2\text{HB} + 70\]
\[ \sigma_{Flim}^o = 1.8\text{HB}\]

For the pinion with $ H_1=\text{HB}205$:
\[ \sigma_{Hlim1}^o = 2 \times 205 + 70 = 480 \unitp{MPa}\]
\[ \sigma_{Flim1}^o = 1.8\times 205 = 369 \unitp{MPa}\]

For the driven gear with $ H_2=H_1-15=205-15=\text{HB}190 $:
\[ \sigma_{Hlim2}^o = 2 \times 190 + 70 = 450 \unitp{MPa}\]
\[ \sigma_{Flim2}^o = 1.8\times 190 = 342 \unitp{MPa}\]

\paragraph{Find safety factor $ S $}
The safety factors $ S_H=1.1 $, $ S_F=1.75 $ depend on material type, Table 6.2 \cite{tk1}.

\paragraph{Find load placement factor $ K_{FC} $}
$ K_{FC} = 1$ for unidirectional operation (i.e. load placement is in one way).

\paragraph{Find aging factor $ K_L $}
The factor is determined by Equation 6.3 and 6.4 \cite{tk1} for contact stress and bending stress, respectively. If calculation yields a value smaller than 1, it is then rounded up to 1 instead, p.94 \cite{tk1}:\\
\[ K_{HL1} = \sqrt[m_H]{N_{HO1}/N_{HE1}} = \sqrt[6]{10600601.81/1.05 \times10^9} = 0.46 < 1 \Rightarrow K_{HL1} = 1\]
\[ K_{HL2} = \sqrt[m_H]{N_{HO2}/N_{HE2}} = \sqrt[6]{8833440.68/0.37 \times10^9} = 0.54 < 1 \Rightarrow K_{HL2} = 1\]
\[ K_{FL1} = \sqrt[m_F]{N_{FO1}/N_{FE1}} = \sqrt[6]{4000000/0.91 \times10^9} = 0.40 < 1 \Rightarrow K_{FL1} = 1\]
\[ K_{FL2} = \sqrt[m_F]{N_{FO2}/N_{FE2}} = \sqrt[6]{4000000/0.32 \times10^9} = 0.48 < 1 \Rightarrow K_{FL2} = 1\]
where
\begin{itemize}
	\item $ m $ is the root of fatigue curve in the stress test of a material. Since $ H_1,H_2 \leq 350 $, $ m_H=6 $ and $ m_F=6 $.
	\item $ N_O $ is the working cycle of equivalent bearing stress. Using Equation 6.5 \cite{tk1}:\\
	\[ N_{HO1} = 30H_1^{2.4} = 30\times 205^{2.4} = 10600601.81\unitp{cycles}\]
	\[ N_{HO2} = 30H_2^{2.4} = 30\times 190^{2.4} = 8833440.68\unitp{cycles}\]
	where all parameters are given in previous sections.
	
	For bending stress, $ N_{FO1}= N_{FO2}= 4000000\unitp{MPa}$ since both gears use steel material.
	\item $ N_E $ is the working cycle of equivalent tensile stress. From Equation 6.7 and 6.8 \cite{tk1}:
	\begin{flalign*}
	N_{HE1} &= 60n_{sh1}cL_h\left[ \left( \dfrac{T_1}{T}\right)^3\dfrac{t_1}{t_1+t_2} + \left( \dfrac{T_2}{T}\right)^3\dfrac{t_2}{t_1+t_2}\right] \\
	&= 60 \times 1455 \times 1 \times 16640 \left[ \left( \dfrac{T}{T}\right)^3\dfrac{15}{15+11} + \left( \dfrac{0.7T}{T}\right)^3\dfrac{11}{15+11}\right] \\
	&= 1.05 \times10^9\unitp{cycles}
	\end{flalign*}
	\begin{flalign*}
	N_{HE2} &= 60n_{sh2}cL_h\left[ \left( \dfrac{T_1}{T}\right)^3\dfrac{t_1}{t_1+t_2} + \left( \dfrac{T_2}{T}\right)^3\dfrac{t_2}{t_1+t_2}\right]\\
	&= 60 \times 514.42 \times 1 \times 16640 \left[ \left( \dfrac{T}{T}\right)^6\dfrac{15}{15+11} + \left( \dfrac{0.7T}{T}\right)^6\dfrac{11}{15+11}\right] \\
	&= 0.37 \times10^9\unitp{cycles}
	\end{flalign*}
	\begin{flalign*}
	N_{FE1} &= 60n_{sh1}cL_h\left[ \left( \dfrac{T_1}{T}\right)^{m_F}\dfrac{t_1}{t_1+t_2} + \left( \dfrac{T_2}{T}\right)^{m_F}\dfrac{t_2}{t_1+t_2}\right] \\
	&= 60 \times 1455 \times 1 \times 16640 \left[ \left( \dfrac{T}{T}\right)^6\dfrac{15}{15+11} + \left( \dfrac{0.7T}{T}\right)^6\dfrac{11}{15+11}\right]\\
	&= 0.91 \times10^9\unitp{cycles}
	\end{flalign*}
	\begin{flalign*} 
	N_{FE2} &= 60n_{sh2}cL_h\left[ \left( \dfrac{T_1}{T}\right)^{m_F}\dfrac{t_1}{t_1+t_2} + \left( \dfrac{T_2}{T}\right)^{m_F}\dfrac{t_2}{t_1+t_2}\right]\\
	&= 60 \times 514.42 \times 1 \times 16640 \left[ \left( \dfrac{T}{T}\right)^6\dfrac{15}{15+11} + \left( \dfrac{0.7T}{T}\right)^6\dfrac{11}{15+11}\right] \\
	&= 0.32 \times10^9\unitp{cycles}
	\end{flalign*}
	where\begin{itemize}
		\item $ c=1 $ is the gear meshing rate. The value $ 1 $ indicates the gears are meshed indefinitely.
		\item $ T_1,T_2,t_1,t_2 $ are given in the design problem.
		\item $ n_{sh1}, n_{sh2} $ are given in Chapter 1.
		\item $ L_h, m_F $ are given in previous sections.
	\end{itemize}
\end{itemize}

\paragraph{Calculate preliminary values of $ [\sigma_H] $, $ [\sigma_{F1}] $, $ [\sigma_{F2}] $} Replacing all the values gives:\\
\[ [\sigma_{H1}] = \sigma_{Hlim1}^oK_{HL1}/S_H = 480\times 1/1.1 = 436.36 \unitp{MPa}\]
\[ [\sigma_{H2}] = \sigma_{Hlim2}^oK_{HL2}/S_H = 450\times 1/1.1 = 409.09 \unitp{MPa}\]
\[ [\sigma_{F1}] = \sigma_{Flim1}^oK_{FC}K_{FL1}/S_F = 369\times 1\times 1/1.75  = 210.86\unitp{MPa}\]
\[ [\sigma_{F2}] = \sigma_{Flim2}^oK_{FC}K_{FL2}/S_F = 342\times 1\times 1/1.75  = 195.43 \unitp{MPa}\]

The permissible contact stress of the gear drive $ [\sigma_H] $ must be lower than 1.25 times of either $ [\sigma_{H1}] $ or $ [\sigma_{H2}] $, whichever is smaller. In this case, it is $ 1.25[\sigma_{H2}]$:
\begin{multline*}
	[\sigma_H] =\dfrac{[\sigma_{H1}]+[\sigma_{H2}]}{2}= \dfrac{436.36+409.09}{2} = 422.73 \unitp{MPa}\\
	\leq 1.25[\sigma_{H2}] = 1.25\times 409.09= 511.36\unitp{MPa}
\end{multline*}

\subsection{Find permissible contact stress $ [\sigma_H]_{max} $ and bending stress $ [\sigma_F]_{max} $ due to overload}
Preventing overload failure is necessary. For quenched material, the permissible overload contact stress $ [\sigma_H]_{max} $ is
\[ [\sigma_H]_{max} = 2.8\sigma_{ch} = 2.8\times 340 = 952.00\unitp{MPa} \]

For material with hardness smaller than $ \text{HB}350 $, the permissible overload bending stress $ [\sigma_F]_{max} $ is
\[ [\sigma_F]_{max} = 0.8\sigma_{ch} = 0.8\times 340 = 272.00\unitp{MPa} \]

\section{Basic specifications of the transmission system}
Undercutting process is omitted due to complexity and maintenance. That is, it is not factored in any variables of the gear drive.
\subsection{Determine basic parameters}
\subsubsection{Find center distance $ a $}
For involute gears, the center distance $ a $ is estimated using Equation 6.15a \cite{tk1} and rounded up to a multiple of 5 (small production type, p.99 \cite{tk1}):
\begin{multline*}
a= K_a(u+1)\sqrt[3]{\dfrac{TK_{H\beta}}{[\sigma_H]^2u\psi_{ba}}}	=  43\times(2.83+1)\times\sqrt[3]{\dfrac{46423.73\times 1.03}{422.73^2\times 2.83\times 0.30}}\\
= 120.14 \unitp{mm} \Rightarrow a= 160 \unit{mm}
\end{multline*}
where
\begin{itemize}
	\item $ K_a=43 $ is the material factor, Table 6.5 \cite{tk1}. The helical gear drive made of steel (both gears) corresponds to the value in the table.
	\item $ u =u_1 = 2.83$ is the speed ratio of the gear drive, which is given in Chapter 1.
	\item $ T=T_{sh1}= 46423.73\unit{N\cdot mm}$ is the torque exerted on the pinion, which is transmitted from the motor to shaft 1.
	\item $ K_{H\beta} = 1.03 $ is the load distribution factor on gear teeth, Table 6.7 \cite{tk1}. In the table, the gear placement in the speed reducer is similar to diagram 5
	\item $ \psi_{ba} = 0.30 $ is the width to shaft distance ratio, Table 6.6 \cite{tk1}. The pinion is asymmetrical about the bearings and both gears surface have hardness smaller than $ \text{HB}350 $. The value is obtained after extensive calculation and changing parameters, which includes hardness, center distance and traverse module. In addition, the remaining gear drive contributes to $ \psi_{ba} $, eventually reducing the center distance further, saving material cost.
	\item $ [\sigma_H] $ is given in previous section.
\end{itemize}

\subsubsection{Find face width ratio $ \psi_{bd} $}
$ \psi_{bd} $ is the face width factor. Using Equation 6.16 \cite{tk1}:
\[ \psi_{bd} = 0.53\psi_{ba}(u+1) = 0.53\times 0.30\times(2.83+1) = 0.69 \]
where all parameters are identified. The calculated value satisfies even with the minimum permissible factor  $ \psi_{bdmax} = 1 $, Table 6.6 \cite{tk1}.
%Figure \ref{fig:problem} shows both pairs are helical, which  gives $ K_a = 43 $, Table 6.5 \cite{tk1}. Also, the pinion is , resulting in $ \psi_{ba} $, . Then, $ \psi_{bd} $ is calculated using Equation 6.16 \cite{tk1}:\\
%, which is smaller than the permissible ratio $ \psi_{bdmax}=1.2 $, Table 6.6 \cite{tk1}.
%
%The gear placement in the speed reducer is similar to diagram 5 of Table 6.7 \cite{tk1}. $ K_{H\beta} = 1.03 $, $ K_{F\beta} = 1.1 $
\subsubsection{Find traverse module $ m_t $}
Using Equation 6.17 and Table 6.8 \cite{tk1}, the traverse module $ m $ is
\[ m_t = (0.01\div0.02)a = (0.01\div0.02)\times 160 =  1.6 \div 3.2 \unitp{mm} \Rightarrow m_t=2.00\unit{mm}\]

The choice is based on calculation to satisfy strength condition, as mentioned in choosing $ \psi_{ba} $.
\subsubsection{Find number of teeth $ z $}
Through calculation, helix angle $ \beta = 20^\circ $ should be preferred (permissible value is $ 8 \div 20^\circ $). Using Equation 6.31 and 6.20 \cite{tk1}, the pinion's number of teeth $ z_1 $ is
\[ z_1 = \dfrac{2a\cos\beta}{m_t(u+1)}= \dfrac{2\times 160\cos 20^\circ}{2.00\times(2.83+1)} = 40.17 \Rightarrow z_1 = 41\]
\[ z_2 = uz_1 = 2.83\times 41 = 115.97 \Rightarrow z_2 = 116\]
where all parameters are given in previous sections. The number of teeth selection follows the \textit{hunting tooth ratio} \cite{Ishibashi1981} as mentioned in Chapter 1. The ratio yields $ z_2/z_1=116/41=2.829 $, which is essentially not much different from the intended value $ u =2.83$. 

\subsubsection{Correct $ \beta $} The helix angles are corrected to compensate for rounding center distance and number of teeth, Equation 6.32 \cite{tk1}:
\[\beta = \arccos\dfrac{m_t(z_2+z_1)}{2a} = \arccos\dfrac{2.00\times(116+41)}{2\times 160} = 11.11 ^\circ\]

\subsection{Other parameters}
The values below are necessary for modeling the gear drive:\\
\[ b = \psi_{ba}a = 0.3\times 160= 48.00\unitp{mm}\]
\[ d_1 = {m_tz_1}/{\cos\beta} = {2.00\times 41}/{\cos9.07^\circ} = 83.57\unitp{mm}\]
\[ d_2 = {m_tz_2}/{\cos\beta} = {2.00\times 116}/{\cos9.07^\circ} = 236.43\unitp{mm}\]
\[ d_{a1} = d_1 + 2m_t = 83.57 + 2\times 2.00 = 87.57\unitp{mm}\]
\[ d_{a2} = d_2 + 2m_t = 236.43 + 2\times 2.00 = 240.43\unitp{mm}\]
\[ d_{b1} = d_1\cos\alpha_n = 83.57\times \cos 20^\circ = 78.53\unitp{mm}\]
\[ d_{b2} = d_2\cos\alpha_n = 236.43\times \cos 20^\circ = 222.17 \unitp{mm}\]
\[ d_{f1} = d_1 - 2.5m_t = 83.57 - 2.5\times 2.00 = 78.57\unitp{mm}\]
\[ d_{f2} = d_2 - 2.5m_t = 236.43 - 2.5\times 2.00 = 231.43\unitp{mm}\]
\[ \alpha_t = \arctan\left({\tan\alpha}/{\cos\beta}\right) = \arctan\left({\tan 20^\circ}/{\cos 11.11^\circ}\right) = 20.35^\circ\]
\[ v_1 = {\pi d_1n_{sh1}}/{60000} = {\pi \times 83.57 \times 1455}/{60000} = 6.37\unitp{m/s}\]
where
\begin{itemize}
	\item $ b $ is the face width,$ \unit{mm} $.
	\item $ d $ is the pitch circle diameter, $ \unit{mm} $.
	\item $ d_a $ is the addendum diameter, $ \unit{mm} $.
	\item $ d_b $ is the projection diameter, $ \unit{mm} $.
	\item $ d_f $ is the dedendum diameter, $ \unit{mm} $.
	\item $ \alpha_t $ is the traverse pressure angle,$ ^\circ $.
	\item $ \alpha_n=20^\circ $ is the normal pressure angle following TCVN 1065-71 standard of Vietnam.
	\item $ v $ is the average speed of the pinion,$ \unit{m/s} $.
\end{itemize}

\section{Stress analysis}
Tolerance grade affects the factors through stress analysis. Since $ v_1=6.37 \unit{m/s} \leq 10 \unit{m/s} $ and the gear drive is helical gear, the grade is 8.
\subsection{Correct the permissible contact stress $ [\sigma_H] $ and bending stresses $ [\sigma_{F1}] $, $ [\sigma_{F2}] $}
After preliminary estimation, the factors $ Z_RZ_vK_{xH} $ and $ Y_RY_sK_{xF} $ are included to increase accuracy of permissible stress values, Equation 6.1 and 6.2 \cite{tk1}:\\
\[[\sigma_H]'=[\sigma_H]Z_RZ_vK_{xH} = 422.73\times 1\times 1.02\times 1 = 432.38\unit{MPa}\]
\[[\sigma_{F1}]'=[\sigma_{F1}]Y_RY_sK_{xF} = 210.86\times 1\times 1.03\times 1 = 217.57\unit{MPa}\]
\[[\sigma_{F2}]'=[\sigma_{F2}]Y_RY_sK_{xF} = 195.43\times 1\times 1.03\times 1 = 201.65\unit{MPa}\]
where
\begin{itemize}
	\item $ Z_R = 1 $ is the surface roughness factor of the working's area, corresponding to roughness deviation  less than $ 1.25\unit{\mu m} $. Hence, the gear drive is gone through honing process.
	\item $ Z_v $ is the speed factor. Since surface hardness of the gear drive is less than $ \text{HB}350 $, the formula is
	\[ Z_v = 0.85v_1^{0.1} = 0.79\times 6.37^{0.1} = 1.02 \]
	\item $ K_x = 1$ is the size factor. Since both gears have $ d_{a1}, d_{a2} \leq 400\unit{mm} $, $ K_{xH}=K_{xF}=1 $.
	\item $ Y_R=1 $ is the surface roughness factor. The gears are unpolished for economical purpose.
	\item $ Y_s $ is the stress concentration factor. Using the equation on p.92 \cite{tk1}, the factor is
	\[Y_s = 1.08-0.0695\ln(m_t) = 1.08-0.0695\ln(2.00) = 1.03\]
	\item other parameters are given in previous sections.
\end{itemize}

\subsection{Contact stress analysis}

The contact stress applied on a gear surface must satisfy Equation 6.33 \cite{tk1}:
\begin{multline*}
\sigma_H = z_Mz_Hz_\varepsilon\sqrt{2TK_H\dfrac{u+1}{bud_1^2}} \\
= 274\times 1.74 \times 0.79 \times \sqrt{2\times 46423.73 \times 1.21 \times\dfrac{2.83+1}{48.00 \times 2.83 \times 83.57^2}}\\
= 251.85 \unitp{MPa} \leq [\sigma_H]' = 432.38 \unit{MPa}
\end{multline*}

where
\begin{itemize}
	\item $ z_M = 274 $ is the material's mechanical properties factor, Table 6.5 \cite{tk1}. The value is chosen since both gears are steel material and helical type.
	\item $ z_H $ is the contact surface's shape factor. Using Equation 6.37 \cite{tk1}, the factor is
	\[ z_H = \sqrt{\dfrac{2\cos\beta_b}{\sin(2\alpha_t)}} = \sqrt{\dfrac{2\times\cos 10.43^\circ}{\sin(2\times 20.35^\circ)}} = 1.74\]
	where $ \beta_b $ is the base circle helix angle. Using Equation 6.35 \cite{tk1}, the angle is
	\[ \beta_b = \arctan\left( \cos\alpha_t\tan\beta\right) = \arctan\left( \cos 20.35^\circ \times \tan 11.11^\circ\right) = 10.43^\circ\]
	\item $ z_\varepsilon $ is meshing condition factor. Depending on the value of face contact ratio $ \varepsilon_\beta $, 1 in 3 Equation 6.36a, 6.36b, 6.36c \cite{tk1} is used. Using Equation 6.37 \cite{tk1}, the ratio is
	\[\varepsilon_\beta = b\dfrac{\sin\beta}{m_t\pi} = 48.00\times\dfrac{\sin 11.11^\circ}{2.00\times\pi}=1.20\]
	Since $ \varepsilon_\beta >1 $, Equation 6.36c \cite{tk1} is used:
	\[z_\varepsilon = \varepsilon_\alpha^{-0.5} = 1.61^{-0.5} = 0.79\]	
	where $ \varepsilon_\alpha $ is the traverse contact ratio. Using Equation 6.38a \cite{tk1}, the factor is
	\begin{flalign*}
	\varepsilon_\alpha &= \dfrac{\sqrt{d_{a1}^2-d_{b1}^2}+\sqrt{d_{a2}^2-d_{b2}^2}-2a\sin\alpha_t}{2\pi m_t{\cos\alpha_t}/{\cos\beta}}\\
	&= \dfrac{\sqrt{87.57^2-78.53^2}+\sqrt{240.43^2-222.17^2}-2\times 160\times\sin 20.35^\circ}{2\pi \times 2.00\times{\cos 20.35^\circ}/{\cos 11.11^\circ}}\\
	&= 1.61
	\end{flalign*}
	where all parameters are identified in previous sections.
	\item $ T=T_{sh1}=46423.73\unit{N\cdot mm} $ is the torque exerted on the pinion.
	\item $ K_H $ is the overall factor. Using Equation 6.39 \cite{tk1}, the factor is
	\[ K_H = K_{H\beta}K_{Hv}K_{H\alpha} = 1.03\times 1.06 \times 1.10  = 1.21\]
	where
	\begin{itemize}
		\item $ K_{H\beta} = 1.03 $ is the load distribution factor on gear teeth, Table 6.7 \cite{tk1}. The value is given previous section.
		\item $ K_{Hv} = 1.06 $ is the dynamic load factor at meshing area, Table P2.3 \cite{tk1}. The value corresponds to helical gear and tolerance grade 8.
		\item $ K_{H\alpha} = 1.10 $ is the load distribution factor on top land, Table 6.14 \cite{tk1}. The value corresponds to the speed $ v_1=6.37\unit{m/s} $ and tolerance grade 8.
	\end{itemize}
\end{itemize}


\subsection{Bending stress analysis}
For safety reasons, the gear drive must follow Equation 6.43 and 6.44 \cite{tk1}. The bending stress $ \sigma_{F1} $ on the pinion and $ \sigma_{F2} $ on the driving gear are

\begin{multline*}
	\sigma_{F1} = \dfrac{2TK_FY_\varepsilon Y_\beta Y_{F1}}{bd_1m_t\cos\beta} [\sigma_{F1}]= \dfrac{2\times 46423.73 \times 1.66 \times 0.62 \times 0.92 \times 3.68}{48.00 \times 83.57 \times 2.00 \times\cos 11.11^\circ}\\
	= 41.09 \unitp{MPa} \leq [\sigma_{F1}]' = 217.57 \unit{MPa}
\end{multline*}
\[\sigma_{F2} = \sigma_{F1}Y_{F2}/Y_{F1} = {41.09 \times 3.60}/3.68 = 40.17 \unitp{MPa} \leq [\sigma_{F2}]' = 201.65 \unit{MPa}\]
where
\begin{itemize}
	\item $ T=T_{sh1}=46423.73\unit{N\cdot mm} $ is the torque exerted on the pinion. In this case, the source is the motor.
	\item $ K_F $ is the overall factor. Using Equation 6.39 \cite{tk1}, the factor is
	\[ K_F = K_{F\beta}K_{Fv}K_{F\alpha} = 1.08 \times 1.18 \times 1.30 = 1.66 \]
	where
	\begin{itemize}
		\item $ K_{F\beta} = 1.08 $ is the load distribution factor on gear teeth, Table 6.7 \cite{tk1}. In the table, the gear placement in the speed reducer is similar to diagram 5.
		\item $ K_{Fv} = 1.18 $ is the dynamic load factor at meshing area, Table P2.3 \cite{tk1}, corresponding to helical gear and tolerance grade 8.
		\item $ K_{F\alpha} = 1.30 $ is the load distribution factor on top land, Table 6.14 \cite{tk1}. The value corresponds to the speed $ v_1=6.37 \unit{m/s} $ and tolerance grade 8.
	\end{itemize}
	\item $ Y_\varepsilon $ is the contact ratio factor. Using equation on p.108 \cite{tk1}, the factor is
	\[ Y_\varepsilon = 1/\varepsilon_\alpha = 1/1.61 = 0.62 \]
	where $ \varepsilon_\alpha $ is given in previous section.
	\item $ Y_\beta $ is the helix angle factor. Using equation on p.108 \cite{tk1}, the factor is
	\[ Y_\beta = 1-{\beta}/{140^\circ}=  1-{11.11^\circ}/{140^\circ}= 0.92 \]
	\item $ Y_F $ is the tooth shape factor, Table 6.18 \cite{tk1}. The factor depends on the value of $ z_v $, which is the virtual number of teeth. Using equation on p.108 \cite{tk1} and the table, $ Y_{F1} $ and $ Y_{F2} $ are
	\[ z_{v1} = z_1\cos^{-3}(\beta) = 41\times\cos^{-3}(11.11^\circ) = 43.40\Rightarrow Y_{F1} = 3.68\]
	\[ z_{v2} = z_2\cos^{-3}(\beta) = 116\times\cos^{-3}(11.11^\circ) = 122.78\Rightarrow Y_{F2} = 3.60\]
	\item $ b,d_1,m_t,\beta $ are given in previous sections.
\end{itemize}

\subsection{Overload analysis}
From the load diagram, the overload factor $ K_{qt} $ is
\begin{flalign*}
	K_{qt}&=\left[\dfrac{\left({T_1}/{T}\right)^2t_1 + \left({T_2}/{T}\right)^2t_2}{t_1+t_2}\right]^{-1/2}\\
	&= \left[\dfrac{\left({T}/{T}\right)^2\times15 + \left({0.7T}/{T}\right)^2\times11}{15+11}\right]^{-1/2}\\
	& = 1.13
\end{flalign*}

Using the values of $ [\sigma_H]_{max} $ and $ [\sigma_F]_{max} $ calculated in previous section combined with Equation 6.48 and 6.49 \cite{tk1}, it is possible to verify the stresses are below overload limits. Replacing all the variables gives\\
\[ \sigma_{Hmax}=\sigma_H\sqrt{K_{qt}} = 251.85\times\sqrt{1.13} = 267.63 \unitp{MPa} \leq 952.00 \unit{MPa} \]
\[ \sigma_{F1max}=\sigma_{F1}K_{qt} =41.09\times 1.13 = 41.09 \unitp{MPa} \leq 272.00 \unit{MPa}\]
\[ \sigma_{F2max}=\sigma_{F2}K_{qt} =40.17\times 1.13 = 45.36 \unitp{MPa} \leq 272.00 \unit{MPa}\]

In summary, the table is obtained

\begin{table}[ht]
	\centering
	\caption{Gear drive final specification}
	\begin{tabular}{lp{0.2\linewidth}p{0.2\linewidth}p{0.2\linewidth}}\toprule
		& Gear drive & Pinion & Driven gear \\ \midrule
		Material 			&	-		&	quenched 45 steel&	quenched 45 steel\\
		$ a\unitp{mm}    $	&	160		&	-		&	-		\\
		$ b\unitp{mm}    $	&	-		&	48.00	&	48.00	\\
		$ d\unitp{mm}    $	&	-		&	83.57	&	236.43	\\
		$ d_a\unitp{mm}  $	&	-		&	87.57	&	240.43	\\
		$ d_f\unitp{mm}  $	&	-		&	78.57	&	231.43	\\
		$ m\unitp{mm}    $	&	2.00	&	-		&	-		\\
		$ \alpha_t\unitp{^\circ}    $	&	20.35	&	-		&	-		\\
		$ \beta\unitp{^\circ}    $	&	11.11	&	-		&	-		\\
		$ z  $	&	-		&	41	&	116	\\
		\bottomrule
	\end{tabular}
	\label{chap4spec}
\end{table}