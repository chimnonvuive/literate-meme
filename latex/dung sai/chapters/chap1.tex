\chapter{KIỂM TRA SAI SỐ HÌNH DÁNG CHI TIẾT TRỤ TRƠN TRONG MẶT CẮT NGANG VÀ DỌC}

\section{Mục đích Thí nghiệm}
\begin{itemize}
	\item Biêt sử dụng pan me, đồng hồ so.
	\item Biết cách kiểm tra sai số hình dáng của loại chi tiết điển hình là trụ trơn.
\end{itemize}

\section{Các dụng cụ}
Bàn máp; pan me; khối V; đồng hồ so

\section{Các bước tiến hành}
\subsection{Đo sai số hình dáng trong mặt cắt dọc}
\begin{table}[ht]
	\centering
	\caption{Thông số đo đạc của các mặt cắt dọc (trong mỗi mặt cắt ngang chỉ đo ở hai cặp đường kính vuông góc với nhau)}
	\begin{tabular}{llllllllll}\toprule
		Chi tiết số 4 & \multicolumn{3}{l}{Mặt cắt I-I} & \multicolumn{3}{l}{Mặt cắt II-II} & \multicolumn{3}{l}{Mặt cắt III-III}\\
		\cmidrule(r{4pt}){2-4} \cmidrule(r{4pt}){5-7} \cmidrule(r{4pt}){8-10}
		& $ AA' $ & $ BB' $ & $ CC' $ & $ AA' $ & $ BB' $ & $ CC' $ & $ AA' $ & $ BB' $ & $ CC' $\\\midrule
		Đường sinh thứ 1 & 0 & - & - & 0.01 & - & - & -0.09 & - & -\\
		Đường sinh thứ 2 & - & 0 & - & - & 0.06 & - & - & -0.05 & -\\
		Đường sinh thứ 3 & - & - & 0 & - & - & 0.03 & - & - & -0.09\\\bottomrule
	\end{tabular}
\end{table}
\subsection{Đo sai số hình dáng trong mặt cắt ngang}
\begin{table}[h]
	\centering
	\caption{Thông số đo đạc của các mặt cắt tại các đường kính (trong mỗi mặt cắt ngang chỉ đo ở hai cặp đường kính vuông góc với nhau)}
	\begin{tabular}{lllll}\toprule
		Chi tiết số 4 & $ AA' $ & $ BB' $ & $ CC' $ & $ DD' $\\\midrule
		Mặt cắt I-I & 3.09 & 3.11 & 3.11 & 3.08\\
		Mặt cắt II-II & 3.12 & 3.12 & 3.11 & 3.13\\
		Mặt cắt III-III & 3.16 & 3.11 & 3.12 & 3.10\\\bottomrule
	\end{tabular}
\end{table}
Mặt cắt I-I: $ \Delta_{oval} = d_{max} - d_{min} = 3.11-3.08 = 0.03 \unit{mm} $\\

Mặt cắt II-II: $ \Delta_{oval} = d_{max} - d_{min} = 3.13-3.11 = 0.02 \unit{mm} $\\

Mặt cắt III-III: $ \Delta_{oval} = d_{max} - d_{min} = 3.16-3.10 = 0.06 \unit{mm} $

\subsection{Đo độ đa cạnh}

\begin{table}[h]
	\centering
	\caption{Số liệu đo đạc tại mặt cắt các tiết diện}
	\begin{tabular}{llll}\toprule
		Chi tiết số 4 & \multicolumn{3}{l}{Trị số đo $ \Delta h $ tại các mặt cắt}\\\cmidrule(r{4pt}){2-4}
		Tiết diện đo & I-I & II-II & III-III\\\midrule
		$ A-A' $ & 0 & 0.01 & 0.01\\
		$ B-B' $ & 0.01 & 0 & 0.01\\
		$ C-C' $ & 0 & 0.01 & 0.01\\\bottomrule
	\end{tabular}
\end{table}
Mặt cắt I-I: $ \Delta_c = \Delta h_{max}/2 = 0.01/2 = 0.005 \unit{mm} $\\

Mặt cắt II-II: $ \Delta_c = \Delta h_{max}/2 = 0.01/2 = 0.005 \unit{mm} $\\

Mặt cắt III-III: $ \Delta_c = \Delta h_{max}/2 = 0.01/2 = 0.005 \unit{mm} $

\section{Đánh giá và nhận xét kết quả đo}
Các chi tiết có dạng côn với độ ovan ở mức chấp nhận được $ \Delta_{oval} = 0.05 \unit{mm} $ và độ đa cạnh $ \Delta_c =  0.005 \unit{mm} $.

Các chi tiết có 5 loại sai số cơ bản: độ côn, độ tang trống, độ cong sin, độ yên ngựa, chữ nhật theo mặc cắt dọc và ba loại sai số: độ tròn, độ ovan, độ đa cạnh trong mặt cắt ngang.

Nguyên nhân sai lệch: do dụng cụ đo và sai số chủ quan do người đo.

Kết luận: Chi tiết đạt yêu cầu về độ đảo mặt đầu và độ đảo hướng tâm.