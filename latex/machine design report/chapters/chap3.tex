\chapter{Chain Drive Design}
\section*{Nomenclature}
\begin{tabular}[t]{p{0.05\linewidth}p{0.4\linewidth}}
	$ [i] $ & permissible impact times per second\\
	$ [s] $ & permissible safety factor\\
	$ [P] $ & permissible power,$ \unit{kW} $\\
	$ [\sigma_H] $ & permissible contact stress,$ \unit{MPa} $\\
	$ A $ & cross sectional area of chain hinge,$ \unit{mm^2} $\\
	$ a $ & real center distance,$ \unit{mm} $\\
	$ a_i $ & estimated center distance,$ \unit{mm} $\\
	$ a_{max} $ & maximum center distance,$ \unit{mm} $\\
	$ a_{min} $ & minimum center distance,$ \unit{mm} $\\
	$ B $ & width between inner link plate,$ \unit{mm} $\\
	$ d $ & chordal diameter,$ \unit{mm} $\\
	$ d_a $ & addendum diameter,$ \unit{mm} $\\
	$ d_f $ & dedendum diameter,$ \unit{mm} $\\
	$ d_l $ & roller diameter,$ \unit{mm} $\\
	$ d_O $ & pin diameter,$ \unit{mm} $\\
	$ E $ & modulus of elasticity,$ \unit{MPa} $\\
	$ F_0 $ & sagging force,$ \unit{N} $\\
\end{tabular}%
\begin{tabular}[t]{p{0.05\linewidth}p{0.4\linewidth}}
	$ F_1 $ & tight side tension force,$ \unit{N} $\\
	$ F_2 $ & slack side tension force,$ \unit{N} $\\
	$ F_r $ & force on the shaft,$ \unit{N} $\\
	$ F_t $ & effective peripheral force,$ \unit{N} $\\
	$ F_v $ & centrifugal force,$ \unit{N} $\\
	$ F_{vd} $ & contact force,$ \unit{N} $\\
	$ i $ & impact times per second\\
	$ K_d $ & weight distribution factor\\
	$ k $ & overall factor\\
	$ k_a $ & center distance and chain's length factor\\
	$ k_{bt} $ & lubrication factor\\
	$ k_c $ & rating factor\\
	$ k_d $ & dynamic load factor\\
	$ k_{dc} $ & chain tension  factor\\
	$ k_f $ & loosing factor\\
	$ k_n $ & coefficient of rotational speed\\
	$ k_r $ & number of tooth factor\\
	$ k_x $ & chain weight factor\\
	$ k_z $ & coefficient of number of teeth\\
\end{tabular}\newpage\noindent
\begin{tabular}[t]{p{0.05\linewidth}p{0.4\linewidth}}
	$ k_0 $ & arrangement of drive factor\\
	$ n $ & angular rotational speed, $ \unit{rpm} $\\
	$ n_{01} $ & experimental angular rotational speed,$ \unit{rpm} $\\
	$ P_t $ & calculated power,$ \unit{kW} $\\
	$ p $ & pitch,$ \unit{mm} $\\
	$ p_{max} $ & permissible sprocket pitch,$ \unit{mm} $\\
	$ Q $ & permissible load,$ \unit{N} $\\
	$ q $ & mass per unit length,$ \unit{kg/m} $\\
	$ s $ & safety factor\\
\end{tabular}%
\begin{tabular}[t]{p{0.05\linewidth}p{0.4\linewidth}}
	$ v $ & instantaneous velocity along the chain,$ \unit{m/s} $\\
	$ x $ & chain length in pitches, the number of links\\
	$ x_c $ & an even number of links\\
	$ z $ & number of teeth of a sprocket\\
	$ z_{max} $ & maximum number of teeth of the driven sprocket\\
	$ \sigma_H $ & contact stress,$ \unit{MPa} $\\
	$ _1 $  & subscript for driving sprocket\\
	$ _2 $  & subscript for driven sprocket\\
\end{tabular}

\textbf{Known parameters} From Chapter 1, we know that:\\
The chain type is roller.\\
$ n_{sh3}= 132.27 \unit{rpm}$\\
$ P_{ch} = 6.26\unit{kW}, u_{ch}=4.5 $

\section{Find the chain drive pitch}
Chain hinge wear is one of the main failure modes, which poses a risk of damaging the entire system.  In this section, the method to find the right pitch is due to the strength criterion of the chain hinge, derived into Equation 5.3 \cite{tk1}. The driving sprocket is connected to shaft 3, $ n_1 = n_{sh3} = 132.27\unit{rpm} $.

\textbf{Calculate $ z $}
The number of teeth determines the how stable is the rotational speed, which relates to the impact intensity and service life:\\
$ z_1 = 29 - 2u_{ch} = 29 - 2 \times 2.03 = 25 \geq 19 $ (rounded to the nearest odd number)\\
$ z_2 = u_{ch}z_1 = 2.03 \times 25 = 51 \leq 120$ (Equation 5.1 \cite{tk1})

\textbf{Calculate $ [P] $} An experiment is conducted to find the optimal pitch given the permissible power and angular rotational speed. A roller chain drive having $ 25 $ teeth on the driving sprocket is tested in 8 different cases of $ n_{01} $ in somewhat similar condition with our design purpose, p.80 \cite{tk1}. In this problem, $ z_1=25 $; $ n_1=132.27\unit{rpm} $, which is close to $ n_{01}=200\unit{rpm} $, which yields $ k_z = {25}/{z_1} = 25/25 = 1 $ and $  k_n = {n_{01}}/{n_{1}} = 200/132.27 = 1.51 $. 

Another step is to specify the working condition of the chain, Table 5.6:
\begin{itemize}
	\item The centerline between 2 sprockets is parallel with the ground. $ k_0 = 1 $
	\item Center distance $ a= (30\div 50)p $, which is similar to the experiment. $ k_a= 1 $
	\item Center distance is modifiable through displacing the sprockets. $ k_{dc}=1 $
	\item Moderate impact load. $ k_d=1.5 $
	\item 1 shift. $ k_c=1 $
	\item Dusty condition with moderate lubrication quality. $ k_{bt} =1.3$
\end{itemize}
Then, we can obtain the value $ [P] $, Equation 5.3 \cite{tk1}:
\begin{multline*}
	P_t=P_{ch}kk_zk_n=P_{ch}k_0k_ak_{dc}k_dk_ck_{bt}k_zk_n\\
	=6.26\times1\times1\times1\times1.5\times1\times1.3\times1\times1.51=18.46\unitp{kW}\leq[P]
\end{multline*}

Inspecting Table 5.5 \cite{tk1} at column $ n_{01}=200\unit{rpm} $, the closet value is $ [P]=19.3\unit{kW} $. Knowing $ [P] $, the pitch is $ p= 31.75 \unit{mm} $, Table 5.5 \cite{tk1}. Consequently, $ d_c=9.55\unit{mm} $, $ B=27.46\unit{mm} $. Because minimizing damage from impact onto the drive is essential, Table 5.8 \cite{tk1} is consulted. In this case, the pitch is indeed suitable.

\section{Determine basic parameters of the chain drive}
\subsection{Find number of links and center distance}
The parameters are found:

\textbf{Find $ x_c $:} The  $ a_{min} = 30p = 30\times 31.75= 952.50 \unit{(mm)} $, $ a_{max} = 50p = 50\times 31.75 = 1587.50 \unit{(mm)}$. Limiting the range of choice for $ a $ in $ [a_{min},a_{max}] $ to reduce the effect from chain weight, we can approximate $ a = 1300 \unit{mm} $ and find $ x_c $:
\begin{flalign*}
	x &= \dfrac{2a}{p} + \dfrac{z_1+z_2}{2} + \dfrac{(z_2-z_1)^2p}{4\pi^2a}\\
	& = \dfrac{2\times 1300}{31.75} + \dfrac{25+51}{2} + \dfrac{(51-25)^2\times 31.75}{4\pi^2\times 1300} =120.31
\end{flalign*}

$ x $ is rounded up to the nearest even number. $ x_c = 122$

\textbf{Find $ a $:}
Using $ x_c $ to find the correct center distance, Equation 5.13 \cite{tk1}. In addition, it is recommended to loose the chain an amount of $ 0.002\div0.004 a $ to reduce tension. Choosing the amount of $ 0.003a $, the coefficient $ 0.997 $ is included in the formula below:
\begin{flalign*}
	a &= \dfrac{0.997p}{4}\left[x_c-\dfrac{z_2+z_1}{2}+\sqrt{\left(x_c-\dfrac{z_2+z_1}{2}\right)^2-2\left(\dfrac{z_2-z_1}{\pi}\right)^2}\right]\\ 
	&= \dfrac{0.997\times31.75}{4}\left[122-\dfrac{51+25}{2}+\sqrt{\left(122-\dfrac{51+25}{2}\right)^2-2\left(\dfrac{51-25}{\pi}\right)^2}\right]\\
	&= 1019.99 \unitp{mm}
\end{flalign*}
%\begin{dmath}
%	a = \dfrac{0.997p}{4}\left[x_c-\dfrac{z_2+z_1}{2}+\sqrt{\left(x_c-\dfrac{z_2+z_1}{2}\right)^2-2\left(\dfrac{z_2-z_1}{\pi}\right)^2}\right] = \dfrac{0.997\times31.75}{4}\left[122-\dfrac{51+25}{2}+\sqrt{\left(122-\dfrac{51+25}{2}\right)^2-2\left(\dfrac{51-25}{\pi}\right)^2}\right] = 1019.99 \unitp{mm}
%\end{dmath}

\textbf{Find other parameters:} The values below are necessary for modeling the chain drive (the last two values use $ d_l = 19.05 \unit{mm} $ from Table 5.2 \cite{tk1}):\\ 
\[d_1 = p/\sin\left(\dfrac{\pi}{z_1}\right) = 31.75/\sin\left(\dfrac{180}{25}\right) = 253.32\unitp{mm}\]
\[d_2 = p/\sin\left(\dfrac{\pi}{z_2}\right) = 31.75/\sin\left(\dfrac{180}{51}\right) = 515.75\unitp{mm}\]
\[d_{a1} = p\left( 0.5 + \cot \dfrac{180}{z_1} \right) = 31.75 \left( 0.5 + \cot \dfrac{180}{25} \right) = 267.20 \unitp{mm}\]
\[d_{a2} = p\left( 0.5 + \cot \dfrac{180}{z_2} \right) = 31.75 \left( 0.5 + \cot \dfrac{180}{51} \right) =  530.65 \unitp{mm}\]
\[d_{f1} = d_1 - 2(0.502d_l+0.05) = 253.32 - 2(0.502\times 19.05+0.05) = 234.08 \unitp{mm}\]
\[d_{f2} = d_2 - 2(0.502d_l+0.05) = 515.75 - 2(0.502\times 19.05+0.05) = 496.50 \unitp{mm}\]

\section{Strength of chain drive}
\subsection{Impact frequency analysis}
After determine the center distance, it is necessary to compare the impact frequency with its permissible value, which  is $ [i]=25 $, Table 5.9 \cite{tk1}. Replacing all the variables gives:
\[i=\dfrac{z_1n_1}{15x}=\dfrac{25\times 132.27}{15\times 120.31}=1.83<[i]\]
which satisfies the condition.

\subsection{Safety factor analysis}
Overloading often occurs at the beginning of operation or due to large load, which could damage the chain drive. In order to operate safely, the chain drive's safety factor must satisfy the following condition:
\[s = \dfrac{Q}{k_dF_t+F_0+F_v} \geq [s]\]

\textbf{Find $ v $} Calculated for other variables. The rotational speed of the smaller sprocket is determined using the formula:
\[	v_1={n_1pz_1}/{60000}= {132.27\times 31.75 \times 25}/{60000} = 2.6 \unitp{m/s}\]

\textbf{Find $ Q $ and $ q $} Table 5.2 \cite{tk1} and $ p =31.75$ are used. $ Q=88500 \unit{N}$ and $ q=3.8\unit{kg/m} $

\textbf{Find $ k_d $}: The workload is moderate. When turning on the machine, it is around $ 1.5 $ times the amount of the nominal load. $ k_d = 1.2 $

\textbf{Find $ k_f $}: The chain drive is parallel to the ground. $ k_f = 6 $

\textbf{Find $ F_t $, $ F_v $ and $ F_0 $} Substituting the values above to obtain the values:
\[	F_t = {1000P_{ch}}/{v_1} = {1000\times 6.26}/{2.6} = 3578.34 \unitp{N}\]
\[	F_v =qv_1^2= 3.8\times 6.26^2 = 11.64 \unitp{N}\]
\[	F_0 =9.81\times10^{-3}k_fqa = 9.81\times10^{-3}k_fqa = 295.92\unitp{N}\]

\textbf{Find $ [s] $}: The limit is found using interpolation, Table 5.10 \cite{tk1}.  $ [s] =7.66$

Replacing all the variables gives:
\[s=\dfrac{88500}{1.2\times 3578.34+295.92+11.64}=19.23\geq 7.66\]
which satisfies the condition.

\subsection{Contact stress analysis}
The following condition must be met, Equation 5.18 \cite{tk1}:
\[\sigma_H = 0.47\sqrt{\dfrac{k_r(F_tk_d+F_{vd})E}{AK_d}}\leq[\sigma_H]\]

\textbf{Find $ [\sigma_H] $} The chain drive has $ z_2=51>30 $ and $ v_1=2.6\unit{mm}<5 \unit{mm} $. Thus the suitable material is quenched 45 steel, which has $ [\sigma_H] = 500 \unit{MPa} $, Table 5.11 \cite{tk1} and $ E_{sprocket} = 205000 \unit{MPa}$ \cite{Mareau2013ExperimentalAN}.

\textbf{Find $ K_d $} Since the chain drive only has one strand, $ K_d = 1$.

\textbf{Find $ F_{vd} $} For 1-strand chain, $ F_{vd} = 13\times10^{-7}n_1p^3 = 13\times10^{-7} \times  132.27 \times 31.75^3 = 6.22 \unitp{N}$

\textbf{Find $ k_r $} Since $ z_1 $ is used to estimate $ k_r $ on p.87 \cite{tk1}, $ k_r= 0.42$.

\textbf{Find E} Assuming the sprockets and chain are made up from the same material (steel). The chain is cast iron ASTM-A48 No. 30A, which has $ E_{chain} = 117000 \unit{MPa} $ \cite{mott_vavrek_wang_2018}. The equivalent modulus of elasticity is then:
\[E = \dfrac{2E_{sprocket}E_{chain}}{E_{sprocket}+E_{chain}} = \dfrac{2\times 205000\times 117000}{205000 + 117000}= 148975.16 \unitp{MPa}\]

\textbf{Find $ A $} The projected area of the hinge depends on the pitch $ p=31.75\unit{mm} $, Table 5.12 \cite{tk1}. $ A = 262 \unit{mm^2} $.

The value $ k_d=1.5 $ and $ F_t =3578.34 \unit{N}$ are found above. Replacing all the variables gives:
\begin{multline*}
	\sigma_H = 0.47\sqrt{\dfrac{0.42\times(3578.34\times 1.5+ 6.22)\times148975.16}{262\times 1}} \\
	=  476.25 \unitp{MPa} \leq 500 \unit{MPa}
\end{multline*}
which satisfies the condition.

\section{Force on shaft}
Apply the following equations, p.87 \cite{tk1}:
\begin{flalign*}
	F_2 &= F_0 + F_v = 295.92 + 11.64 = 307.55 \unit{(N)}\\
	F_1 &= F_t + F_2 = 3578.34+307.55 = 3885.89 \unit{(N)}
\end{flalign*}
%\s $ $\\
%\s $ $\\
The chain drive is parallel to the ground corresponding to $ k_x=1.15 $. Apply Equation 5.20 \cite{tk1}:
\[F_r = k_xF_t = 1.15 \times 3578.34 = 2772.05\unit{(N)}\]


In summary, we have the following table:

\begin{table}[ht]
	\centering
	\begin{tabular}{lp{0.2\linewidth}p{0.2\linewidth}p{0.2\linewidth}}\toprule
		& Chain drive & Driving sprocket & Driven sprocket \\ \midrule
%		$ [P] \unitp{kW} $	&	19.3	&	-		&	-		\\
		$ a\unitp{mm}    $	&	1323.02	&	-		&	-		\\
		$ B\unitp{mm}    $	&	27.46	&	-		&	-		\\
		$ d\unitp{mm}    $	&	-		&	253.32	&	515.75	\\
		$ d_a\unitp{mm}  $	&	-		&	267.20	&	530.65	\\
		$ d_c\unitp{mm}  $	&	9.55	&	-		&	-		\\
		$ d_f\unitp{mm}  $	&	-		&	234.08	&	496.50	\\
		$ d_l\unitp{mm}  $	&	19.05	&	-		&	-		\\
%		$ i              $	&	1.83	&	-		&	-		\\
%		$ P \unitp{kW}   $	&	19.3	&	-		&	-		\\
%		$ p\unitp{mm}    $	&	31.75	&	-		&	-		\\
%		$ u_{ch}         $	&			&	-		&	-		\\
%		$ v\unitp{m/s}   $	&			&	-		&	-		\\
		$ z              $	&			&	25		&	51		\\\bottomrule
	\end{tabular}
	\caption{Chain drive specifications}
\end{table}