\chapter{Choose Motor}

\section*{Nomenclature}
\begin{tabular}[t]{p{0.05\linewidth}p{0.4\linewidth}}
	$ n_{sh} $ & rotational speed of shaft, $ \unit{rpm} $\\
	$ P_{mo} $ & calculated motor power to drive the system, $ \unit{kW} $\\
	$ P_{sh} $ & operating power of shaft, $ \unit{kW} $\\
	$ P_w $ & operating power of the belt conveyor given a workload, $ \unit{kW} $\\
	$ u_1 $ & transmission ratio of quick stage\\
	$ u_2 $ & transmission ratio of slow stage\\
	$ u_{ch} $ & transmission ratio of chain drive\\
	$ u_{h} $ & transmission ratio of speed reducer
\end{tabular}
\begin{tabular}[t]{p{0.05\linewidth}p{0.4\linewidth}}
	$ u_{sys} $ & transmission ratio of the system\\
	$ T_{mo} $ & motor torque, $ \unit{N\cdot mm} $\\
	$ T_{sh} $ & shaft torque, $ \unit{N\cdot mm} $\\
	$ \eta_b $ & bearing efficiency\\
	$ \eta_c $ & coupling efficiency\\
	$ \eta_{ch} $ & chain drive efficiency\\
	$ \eta_{hg} $ & helical gear efficiency\\
	$ \eta_{sys} $ & efficiency of the system\\
	$ _1 $ & subscript for shaft 1\\
	$ _2 $ & subscript for shaft 2\\
	$ _3 $ & subscript for shaft 3
\end{tabular}

\section{Choose motor for the mixing tank}
The choice of motor will affect the entire system, so it is necessary to pick the right one.

\textbf{Calculate system overall efficiency} From Table 2.3 \cite{tk1}:
\begin{itemize}
	\item 1 elastic coupling which connects the motor and the speed reducer. $ \eta_c = 1 $
	\item 4 sealed rolling bearings. 3 of which belong to the speed reducer and the last one is used for the shaft of the mixing tank. $ \eta_b = 0.99 $
	\item 2 sealed pairs of helical gear drives which connect the shafts inside the speed reducer. $ \eta_{hg} = 0.97 $
	\item 1 sealed roller chain drive connecting the speed reduce and the mixing tank. $ \eta_{ch} = 0.96 $
\end{itemize}
Aggregate all efficiencies yields: 
\[\eta_{sys} = \eta_c\eta_b^4\eta_{hg}^2\eta_{ch} = 1 \times 0.99^4 \times 0.97^2 \times 0.96 = 0.87\]

\textbf{Calculate required power for operation}
The power $ P $ from Chapter 1 applies for systems with single loading input. In case of varying load each cycle, the equivalent power is calculated using  Equation 2.13 \cite{tk1}:
\[P_w = P\sqrt{\dfrac{\left(\dfrac{T_1}{T}\right)^2t_1 + \left(\dfrac{T_2}{T}\right)^2t_2}{t_1+t_2}} = 7 \times \sqrt{\dfrac{\left(\dfrac{T}{T}\right)^2\times15 + \left(\dfrac{0.7T}{T}\right)^2\times11}{15+11}} = 6.2 \unitp{kW}\]
\[P_{mo} = \dfrac{P_w}{\eta_{sys}} = \dfrac{6.2}{0.87} = 7.14 \unitp{kW}x\]

\textbf{Calculate $ n_{mo} $} The purpose is to Using Table 2.4 \cite{tk1}:
\begin{itemize}
	\item 2-level transmission speed reducer, spur gear type. $ u_{h} = 11 $
	\item 1 chain drive, roller type. $ u_{ch} = 2$
\end{itemize}
Multiplying all transmission ratio yields:\\
$ u_{sys} = u_{h}u_{ch} = 11 \times 2 = 22 $\\
$ n_{mo} = u_{sys}n = 22 \times 65 = 1430 \unitp{rpm} $

\textbf{Choose motor}
To work normally, the maximum operating power of the chosen motor must be no smaller than $ P_{mo} $. In similar fashion, its rotational speed must also be no smaller than  $ n_{mo} $. Thus, from Table P1.3 \cite{tk1}, we choose motor 4A132S4Y3 which operates at $ 7.5 \unit{kW} $ maximum and $ 1455 \unit{rpm} $, which makes $ n_{mo} = 1455\unit{rpm}$.

Recalculating $ u_{sys} $ with the new $ n_{mo} $ derived from the chosen motor:\\
$ u_{sys} = {n_{mo}}/{n} = {1455}/{65} = 22.38 $

Retaining the transmission ratio of the speed reducer (i.e. let $ u_{h} = const = 11 $), the new transmission ratio of the chain drive is then:\\
$ u_{ch} = {u_{sys}}/{u_{h}} = {22.38}/{11} = 2.03 $

\section{Calculate power, rotational speed and torque}
Let $ P_{sh1} $, $ n_{sh1} $ and $ T_{sh1} $ be the transmitted power, rotational speed and torque onto shaft 1, respectively. Similarly, $ P_{sh2} $, $ n_{sh2} $ and $ T_{sh2} $ are the transmitted parameters onto shaft 2 and $ P_{sh3} $, $ n_{sh3} $ and $ T_{sh3} $ are used for shaft 3. Unless otherwise stated, these notations will be used throughout the next chapters.

\textbf{Power} The entire system is described followed by calculation as follows:\\
Chain drive power is affected by the bearings on the shaft of the mixing tank.\\
$ P_{ch} = \dfrac{P_w}{\eta_b} = \dfrac{6.2}{0.99} = 6.26 \unitp{kW}$\\
Shaft 3 power is affected by the chain drive.\\
$ P_{sh3} = \dfrac{P_{ch}}{\eta_{ch}} = \dfrac{6.26}{0.96} = 6.52 \unitp{kW}$\\
Shaft 2 power is affected by the bearings and gear drives on shaft 3.\\
$ P_{sh2} = \dfrac{P_{sh3}}{\eta_b\eta_{hg}} = \dfrac{6.52}{0.99 \times 0.97} = 6.79 \unitp{kW}$\\
Shaft 1 power is affected by the bearings and gear drives on shaft 2.\\
$ P_{sh1} = \dfrac{P_{sh2}}{\eta_b\eta_{hg}} = \dfrac{6.79}{0.99 \times 0.97} = 7.07 \unitp{kW}$

\textbf{Rotational speed} The design goal of the speed reducer is to lubricate both driven gears equally, which has a size disadvantage. Therefore, the transmission ratio of each pair of gears is calculated using Equation 3.12 \cite{tk1}:
\[u_1=u_2=\sqrt{u_h}=\sqrt{11}=3.32\]

Then,\\
from motor to shaft 1: $ n_{sh1} = n_{mo} = 1455 \unitp{rpm}$ \\
from shaft 1 to shaft 2: $ n_{sh2} = {n_{sh1}}/{u_{1}} = 1455/3.32 = 438.70 \unitp{rpm}$\\
from shaft 2 to shaft 3: $ n_{sh3} = {n_{sh2}}/{u_{2}} = 438.70/3.32 = 132.27 \unitp{rpm}$

\textbf{Torque} Subsequently, the torque is calculated as follows:\\
$ T_{mo}  = 9.55\times10^6 \times P_{mo}/n_{mo} = 9.55\times10^6 \times 7.14/1455 = 46892.66 \unit{(N\cdot mm)}$\\
$ T_{sh1} = 9.55\times10^6 \times {P_{sh1}}/{n_{sh1}} = 9.55\times10^6 \times 7.07/1455 = 46423.73 \unit{(N\cdot mm)}$\\
$ T_{sh2} = 9.55\times10^6 \times {P_{sh2}}/{n_{sh2}} = 9.55\times10^6 \times 6.79/438.70 = 147857.49 \unit{(N\cdot mm)}$\\
$ T_{sh3} = 9.55\times10^6 \times {P_{sh3}}/{n_{sh3}} = 9.55\times10^6 \times 6.52/132.27 = 470919.44 \unit{(N\cdot mm)}$

In summary, we obtain the following table:
\begin{table}[ht]
	\centering
	\begin{tabular}{lllll}\toprule
		&Motor    & Shaft 1  & Shaft 2  & Shaft 3   \\\midrule
		$ P \unitp{kW}$ & 7.14  & 7.07   & 6.79   & 6.52   \\
		$ u $ &       -   &1    &  3.32  & 3.32                 \\
		$ n \unitp{rpm}$ & 1455 & 1455  & 438.70 & 132.27 \\
		$ T \unitp{N\cdot mm}$ & 46892.66 & 46423.73 & 147857.49 & 470919.44\\
		\bottomrule
	\end{tabular}
	\caption{Output specifications}
	\label{tab:my-table}
\end{table}