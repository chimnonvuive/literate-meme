\chapter{Gear Design (shaft 2-3)}
\section*{Nomenclature}
\begin{tabular}[t]{p{0.1\linewidth}p{0.35\linewidth}}
	$ [\sigma] $ & permissible stress, $ \unit{MPa} $\\
	$ [\sigma]_{max} $ & permissible stress due to overloading, $ \unit{MPa} $\\
	$ \text{AG} $ & accuracy grade of gear\\
	$ a $ & center distance, $ \unit{mm} $\\
	$ b $ & face width, $ \unit{mm} $\\
	$ c $ & gear meshing rate\\
	$ d $ & pitch circle, $ \unit{mm} $\\
	$ d_a $ & addendum diameter, $ \unit{mm} $\\
	$ d_b $ & base diameter, $ \unit{mm} $\\
	$ d_f $ & deddendum diameter, $ \unit{mm} $\\
	$ H $ & surface roughness, HB\\
	$ K_d $ & coefficient of gear material\\	
	$ K $ & overall load factor\\
	$ K_{C} $ & load placement factor\\
	$ K_{L} $ & aging factor\\
	$ K_{v} $ & dynamic load factor at meshing area\\
	$ K_{\alpha} $ & load distribution factor on gear teeth\\
	$ K_{\beta} $ & load distribution factor on top land\\
	$ K_{qt} $ & overloading factor\\
	$ m $ & root of fatigue curve in stress test\\
	$ m_t $ & traverse module, $ \unit{mm} $\\
	
	%	$ F_a $ & axial force, $ \unit{N} $\\
	%	$ F_r $ & radial force, $ \unit{N} $\\
	%	$ F_t $ & tangential force, $ \unit{N} $\\
\end{tabular}
\begin{tabular}[t]{p{0.05\linewidth}p{0.4\linewidth}}
	$ m_n $ & normal module, $ \unit{mm} $\\
	$ N_{E} $ & working cycle of equivalent tensile stress\\
	$ N_{O} $ & working cycle of bearing stress\\
	$ S $ & safety factor\\	
	$ v $ & rotational velocity, $ \unit{m/s} $\\
	$ Y_F $ & tooth shape factor\\
	$ Y_R $ & surface roughness factor\\
	$ Y_s $ & stress concentration factor\\
	$ Y_\beta $ & helix angle factor\\
	$ Y_\varepsilon $ & contact ratio factor\\
	$ Z_R $ & surface roughness factor of the working's area\\
	$ Z_v $ & speed factor\\
	$ z_H $ & contact surface's shape factor\\
	$ z_M $ & material's mechanical properties factor\\
	$ z_v $ & virtual number of teeth\\
	$ z_\varepsilon $ & meshing condition factor\\
	$ \alpha $ & normal pressure angle. In TCVN 1065-71, $ \alpha = 20^\circ $\\
	$ \alpha_t $ & traverse pressure angle, $ ^\circ $\\
	$ \varepsilon_\alpha $ & traverse contact ratio\\
	$ \varepsilon_\beta $ & face contact ratio\\
	$ \beta $ & helix angle, $ ^\circ $\\
\end{tabular}\newpage
\noindent\begin{tabular}[t]{p{0.05\linewidth}p{0.4\linewidth}}
	$ \beta_b $ & base circle helix angle, $ ^\circ $\\
	$ \psi_{ba} $ & width to shaft distance ratio\\
	$ \psi_{bd} $ & face width factor \\
	$ \sigma_b $ & ultimate strength, $ \unit{MPa} $\\
	$ \sigma_{ch} $ & yield limit, $ \unit{MPa} $\\
	$ \sigma_{lim}^o $ & permissible stress corresponding to working cycle, $ \unit{MPa} $\\	
\end{tabular}
\begin{tabular}[t]{p{0.05\linewidth}p{0.4\linewidth}}
	$ \sigma_{max} $ & stress due to overloading, $ \unit{MPa} $ \\
	$ _{1} $ & subscript for the pinion \\
	$ _{2} $ & subscript for the driven gear\\
	$ _F $ & subscript relating to bending stress\\
	$ _H $ & subscript relating to contact stress\\
\end{tabular}

\section{Choose material}
For small scale machines, type I is sufficient. Materials of this type  are often quenched and have low hardness value ($ \text{HB}\leq 350 $), which make it easy to manufacture precisely.  However, their permissible stresses are low. In this design problem, 45 steel  is used to manufacture both gears.  The specifications of the material are $ \text{HB} 205 $, $ \sigma_b = 600\unit{MPa} $, $ \sigma_{ch} = 340 \unit{MPa}$, Table 6.1  \cite{tk1}.

\section{Estimate the permissible contact stress and bending stress}
The permissible stresses are calculated as follows, Equation 6.1 and 6.2 \cite{tk1}:
\[[\sigma_H]=\dfrac{\sigma_{Hlim}^o}{S_H}Z_RZ_vK_{xH}K_{HL}\]
\[[\sigma_{F}]=\dfrac{\sigma_{Flim}^o}{S_F}Y_RY_sK_{xF}K_{FC}K_{FL}\]

Initial estimation of these values assumes $ Z_RZ_vK_{xH} = 1 $ and $ Y_RY_sK_{xF} = 1 $. The factors will be accounted for in the next sections.

\textbf{Find $ \sigma_{lim}^o $} The value depends on hardness. It is also recommended to have the hardness of the pinion slightly higher than the driving gear from $ 10 $ to $ 15 $ units. Using Table 6.2 \cite{tk1}, the formulas below are used for quenched 45 steel:\\
$ \sigma_{Hlim}^o = 2\text{HB} + 70$\\
$ \sigma_{Flim}^o = 1.8\text{HB} $

For the pinion with $ H_1=\text{HB}205$:\\
$ \sigma_{Hlim1}^o = 2 \times 205 + 70 = 480 \unitp{MPa}$\\
$ \sigma_{Flim1}^o = 1.8\times 205 = 369 \unitp{MPa}$.

For the driven gear with $ H_2=H_1-15=205-15=\text{HB}190 $:\\
$ \sigma_{Hlim2}^o = 2 \times 190 + 70 = 450 \unitp{MPa}$\\
$ \sigma_{Flim2}^o = 1.8\times 190 = 342 \unitp{MPa}$.

\textbf{Find $ S $} The safety factors are found using Table 6.2 \cite{tk1}. $ S_H=1.1 $, $ S_F=1.75 $

\textbf{Find $ K_{FC} $} The motor works in one direction and the load is placed in one way. $ K_{FC} = 1$

\textbf{Find $ K_L $} The aging factor is determined by Equation 6.3 and 6.4 \cite{tk1} for contact stress and bending stress, respectively. The surface hardness of the gears is smaller than $ 350 $, which makes $ m_H=6 $, $ m_F=6 $. Also, both pairs of gears are meshed indefinitely which makes $ c=1 $. From working condition, the service life in hours are found:
\[L_h = 8\unit{\left( \dfrac{hours}{shift}\right)}\times Ca  K_{ng} L= 8 \times 1\times 260\times 8 = 16640\unitp{hours}\]

$ N_{HE} $ and $ N_{FE} $ are found using Equation 6.7 and 6.8 \cite{tk1}:
\begin{multline*}
N_{HE1} = 60n_{sh1}cL_h\left[ \left( \dfrac{T_1}{T}\right)^3\dfrac{t_1}{t_1+t_2} + \left( \dfrac{T_2}{T}\right)^3\dfrac{t_2}{t_1+t_2}\right] \\
= 60 \times 1455 \times 1 \times 16640 \left[ \left( \dfrac{T}{T}\right)^3\dfrac{15}{15+11} + \left( \dfrac{0.7T}{T}\right)^3\dfrac{11}{15+11}\right] = 1.05 \times10^9\unitp{cycles}
\end{multline*}
\begin{multline*}
N_{HE2} = 60n_{sh2}cL_h\left[ \left( \dfrac{T_1}{T}\right)^3\dfrac{t_1}{t_1+t_2} + \left( \dfrac{T_2}{T}\right)^3\dfrac{t_2}{t_1+t_2}\right]\\
= 60 \times 438.70 \times 1 \times 16640 \left[ \left( \dfrac{T}{T}\right)^6\dfrac{15}{15+11} + \left( \dfrac{0.7T}{T}\right)^6\dfrac{11}{15+11}\right] = 0.32 \times10^9\unitp{cycles}
\end{multline*}
\begin{multline*}
N_{FE1} = 60n_{sh1}cL_h\left[ \left( \dfrac{T_1}{T}\right)^{m_F}\dfrac{t_1}{t_1+t_2} + \left( \dfrac{T_2}{T}\right)^{m_F}\dfrac{t_2}{t_1+t_2}\right] \\
= 60 \times 1455 \times 1 \times 16640 \left[ \left( \dfrac{T}{T}\right)^6\dfrac{15}{15+11} + \left( \dfrac{0.7T}{T}\right)^6\dfrac{11}{15+11}\right] = 0.91 \times10^9\unitp{cycles}
\end{multline*}
\begin{multline*} 
N_{FE2} = 60n_{sh2}cL_h\left[ \left( \dfrac{T_1}{T}\right)^{m_F}\dfrac{t_1}{t_1+t_2} + \left( \dfrac{T_2}{T}\right)^{m_F}\dfrac{t_2}{t_1+t_2}\right]\\
= 60 \times 438.70 \times 1 \times 16640 \left[ \left( \dfrac{T}{T}\right)^6\dfrac{15}{15+11} + \left( \dfrac{0.7T}{T}\right)^6\dfrac{11}{15+11}\right] = 0.27 \times10^9\unitp{cycles}
\end{multline*}

The working cycle of bearing stress is found using the formula below, Equation 6.5 \cite{tk1}:\\
$ N_{HO1} = 30H_1^{2.4} = 30\times 205^{2.4} = 10600601.81\unit{(cycles)}$.\\
$ N_{HO2} = 30H_2^{2.4} = 30\times 190^{2.4} = 8833440.68\unit{(cycles)}$.

For steel, $ N_{FO1}= N_{FO2}= 4000000\unitp{MPa}$. The factors $ K_L $ are rounded up to 1 if it is less than 1, p.94 \cite{tk1}. Apply Equation 6.3 and 6.4 \cite{tk1}:\\
$ K_{HL1} = \sqrt[m_H]{N_{HO1}/N_{HE1}} = \sqrt[6]{10600601.81/1.05 \times10^9} = 0.46 < 1 \Rightarrow K_{HL1} = 1$\\
$ K_{HL2} = \sqrt[m_H]{N_{HO2}/N_{HE2}} = \sqrt[6]{8833440.68/0.32 \times10^9} = 0.55 < 1 \Rightarrow K_{HL2} = 1$\\
$ K_{FL1} = \sqrt[m_F]{N_{FO1}/N_{FE1}} = \sqrt[6]{4000000/0.91 \times10^9} = 0.40 < 1 \Rightarrow K_{FL1} = 1$\\
$ K_{FL2} = \sqrt[m_F]{N_{FO2}/N_{FE2}} = \sqrt[6]{4000000/0.27 \times10^9} = 0.49 < 1 \Rightarrow K_{FL2} = 1$

\textbf{Calculate $ [\sigma_H] $, $ [\sigma_{F1}] $, $ [\sigma_{F2}] $} Replacing all the values:\\
$ [\sigma_{H1}] = \sigma_{Hlim1}^oK_{HL1}/S_H = 480\times 1/1.1 = 436.36 \unitp{MPa}$\\
$ [\sigma_{H2}] = \sigma_{Hlim2}^oK_{HL2}/S_H = 450\times 1/1.1 = 409.09 \unitp{MPa}$\\
$ [\sigma_{F1}] = \sigma_{Flim1}^oK_{FC}K_{FL1}/S_{F} = 369\times 1\times 1/1.75  = 210.86\unitp{MPa}$\\
$ [\sigma_{F2}] = \sigma_{Flim2}^oK_{FC}K_{FL2}/S_{F} = 342\times 1\times 1/1.75  = 195.43 \unitp{MPa}$

The mean permissible contact stress must be lower than 1.25 times of either $ [\sigma_{H1}] $ or $ [\sigma_{H2}] $, whichever is smaller. In this case, it is $ 1.25[\sigma_{H2}]$. Replacing all the variables gives:
\begin{multline*}
[\sigma_H] =\dfrac{[\sigma_{H1}]+[\sigma_{H2}]}{2}= \dfrac{436.36+409.09}{2} = 422.73 \unitp{MPa}\\
\leq 1.25[\sigma_{H2}] = 1.25\times 409.09= 511.36\unitp{MPa}
\end{multline*}
which satisfy the condition.

In case of overloading, the permissible contact and bending stresses are calculated as follows:\\
For quenched material: $ [\sigma_H]_{max} = 2.8\times 600 = 952\unitp{MPa} $\\
For material with hardness smaller than $ \text{HB}350 $: $ [\sigma_F]_{max} = 0.8\sigma_{ch} = 0.8\times 340 = 272\unitp{MPa} $

\section{Determine basic specifications of the transmission system}

\subsection{Determine basic parameters}
Figure \ref{fig:problem} shows both pairs are helical, which  gives $ K_a = 43 $, Table 6.5 \cite{tk1}. Also, the pinion is symmetrical about the bearings and both gears surface have hardness smaller than $ \text{HB}350 $, resulting in $ \psi_{ba} = 0.3$, Table 6.6 \cite{tk1}. Then, $ \psi_{bd} $ is calculated using Equation 6.16 \cite{tk1}:\\
$ \psi_{bd} = 0.53\psi_{ba}(u+1) = 0.53\times 0.3\times(3.32+1) = 0.69 $, which is smaller than the permissible ratio $ \psi_{bdmax}=1.2 $, Table 6.6 \cite{tk1}.

The gear placement in the speed reducer is similar to diagram 5 of Table 6.7 \cite{tk1}. $ K_{H\beta} = 1.04 $, $ K_{F\beta} = 1.1 $

\textbf{Find $ a $} Since the gear drive only has involute gears, $ a $ is estimated using Equation 6.15a \cite{tk1}. It is also rounded up to the nearest multiple of 5 (small production type, p.99 \cite{tk1}):
\[a= K_a(u+1)\sqrt[3]{\dfrac{T_{sh1}K_{H\beta}}{[\sigma_H]^2u\psi_{ba}}}	=  43\times(3.32+1)\times\sqrt[3]{\dfrac{46423.73\times 1.04}{422.73^2\times 3.32\times 0.3}}
= 120.14 \unitp{mm}\]
$ \Rightarrow a= 180 \unit{mm} $

Due to complexity and maintenance, the undercutting process will be omitted throughout the calculations, that is, it is not factored in any variables of the gear drives.

\textbf{Find $ m $} Using Equation 6.17 \cite{tk1} and Table 6.8 \cite{tk1}, we determine $ m $ for each pair of gears:\\
$ m_t = (0.01\div0.02)a = (0.01\div0.02)\times 180 =  1.8 \div 3.6 \unitp{mm} \Rightarrow m_t=2\unit{mm}$

\textbf{Find $ z_1 $, $ z_2 $, $ b $} Arbitrarily choose $ \beta = 20^\circ $ (permissible value is $ 8 \div 20^\circ $). Combining Equation 6.18 and 6.20 \cite{tk1} to calculate $ z_1 $:
\[z_1 = \dfrac{2a\cos\beta}{m_t(u+1)}= \dfrac{2\times 180\cos 20^\circ}{2\times(3.32+1)} = 39.18 \Rightarrow z_1 = 40\]

Then, find $ z_2 $ and $ b $:\\
$ z_2 = uz_1 = 3.32\times 40 = 132.66 \Rightarrow z_2 = 133 $\\
$ b = \psi_{ba}a = 0.3\times 180= 54\unitp{mm}$

\textbf{Correct $ \beta $} The helix angles are corrected to compensate for rounding center distances and number of teeth, Equation 6.32 \cite{tk1}:
\[\beta = \arccos\dfrac{m_t(z_2+z_1)}{2a} = \arccos\dfrac{2\times(133+40)}{2\times 180} = 16.03 ^\circ\]

%$ y = \dfrac{a}{m_t} - 0.5(z_1+z_1) = 1.69$\\
%$ k_{yq} = \dfrac{1000y}{z_1+z_1} = 15.41 $\\
%$ y_s = \dfrac{a_{ws}}{m_{ts}} - 0.5(z_{1s}+z_{2s}) =0.59 $\\
%$ k_{ys} = \dfrac{1000y_s}{z_{1s}+z_{2s}} = 4.96 $\\
%From $ k_y $, we inspect Table 6.10a \cite{tk1} and obtain $ k_x $. The results are $ k_{xq}=1.63 $, $ k_{xs}=0.19 $.
%
%The only thing left to do  is to find $ x_1 $, $ x_2 $ and $ \alpha_{tw} $.\\
%$ \Delta_{yq}=\dfrac{k_{xq}(z_1+z_1)}{1000}= 0.18$\\
%$ x_1=0.5(y+\Delta_{yq}-\dfrac{(z_1-z_1)y}{z_1+z_1}) =0.49$\\
%$ x_{2q}=y+\Delta_{yq}-x_1 =1.38$\\
%$ \alpha_{twq}=\arccos\left[\dfrac{(z_1+z_1)m_t\cos\alpha}{2a}\right]= 24.27^\circ$\\
%$ \Delta_{ys}=\dfrac{k_{xs}(z_{1s}+z_{2s})}{1000}= 0.02$\\
%$ x_{1s}=0.5(y_s+\Delta_{ys}-\dfrac{(z_{2s}-z_{1s})y}{z_{1s}+z_{2s}}) =0.16$\\
%$ x_{2s}=y_s+\Delta_{ys}-x_{1s} =0.46$\\
%$ \alpha_{tws}=\arccos\left[\dfrac{(z_{1s}+z_{2s})m_{ts}\cos\alpha}{2a_{ws}}\right]= 21.49^\circ$

%\paragraph{Find $ x_1 $, $ x_2 $} To find $ x_1 $ and $ x_2 $, we will follow the calculation scheme provided in p.103. Since $ \beta \approx 13.59^\circ \in (10,15]$, $ z_{min} = 11$, which leads to $ z_1 $ satisfying condition $ z_1 \geq z_{min} + 2 > 10 $, according to table (6.9). Combined with $ u_{hg} = 5 \geq 3.5 $, we obtain $ x_1 = 0.3 $, $ x_2 = -0.3 $, disregarding the calculation of $ y $.

\subsection{Other parameters}
$ d_1 = {m_tz_1}/{\cos\beta} = {2\times 40}/{\cos16.03^\circ} = 83.24\unitp{mm} $\\
$ d_2 = {m_tz_2}/{\cos\beta} = {2\times 133}/{\cos16.03^\circ} = 276.76\unitp{mm} $\\
$ d_{a1} = d_1 + 2m_t = 83.24 + 2\times 2 = 87.24\unitp{mm}$\\
$ d_{a2} = d_2 + 2m_t = 276.76 + 2\times 2 = 280.76\unitp{mm}$\\
$ d_{f1} = d_1 - 2.5m_t = 83.24 - 2.5\times 2 = 78.24\unitp{mm}$\\
$ d_{f2} = d_2 - 2.5m_t = 276.76 - 2.5\times 2 = 271.76\unitp{mm}$\\
$ d_{b1} = d_1\cos\alpha = 83.24\times \cos 20^\circ = 78.22\unitp{mm}$\\
$ d_{b2} = d_2\cos\alpha = 276.76\times \cos 20^\circ = 260.07 \unitp{mm}$\\
$ \alpha_t = \arctan\left({\tan\alpha}/{\cos\beta}\right) = \arctan\left({\tan 20^\circ}/{\cos 16.03^\circ}\right) = 20.74^\circ $\\
$ v_1 = {\pi d_1n_{sh1}}/{60000} = {\pi \times 83.24 \times 1455}/{60000} = 6.34\unitp{m/s}$

\section{Stress analysis}
\subsection{Correct $ [\sigma_H] $, $ [\sigma_{F1}] $ and $ [\sigma_{F2}] $}
In reality, the factors $ Z_RZ_vK_{xH} $ and $ Y_RY_sK_{xF} $ do not equal to 1. This part will correct them to obtain the correct permissible stresses.

\begin{itemize}
	\item The roughness deviation is less than $ 1.25\mu m $. $ Z_R = 1 $
	\item The surface hardness of the gear drive is less than $ \text{HB}350 $. $ Z_v = 0.85v_1^{0.1} = 0.85\times 6.34^{0.1} = 1.02$
	\item The gears have $ d_{a1}, d_{a2} \leq 700\unit{mm} $. $ K_{xH} = 1$
	\item The gears are not polished. $ Y_R=1 $
	\item $ Y_s = 1.08-0.0695\ln(m_t) = 1.03 $
	\item $ d_{a1}, d_{a2} \leq 400 \unit{mm} $. $ K_{xF}=1 $
\end{itemize}
Replacing all the variables gives:\\
$ [\sigma_H] = 422.73\times 1\times 1.02\times 1 = 432.21\unit{MPa}$ \\
$ [\sigma_{F1}] = 210.86\times 1\times 1.03\times 1 = 217.57\unit{MPa}$\\
$ [\sigma_{F2}] = 195.43\times 1\times 1.03\times 1 = 201.65\unit{MPa}$

\subsection{Contact stress analysis}
The contact stress applied on a gear surface must satisfy Equation 6.33 \cite{tk1}:
\[
\sigma_H = z_Mz_Hz_\varepsilon\sqrt{2T_{sh1}K_H\dfrac{u+1}{bud_1^2}} \leq [\sigma_H]
\]

\textbf{Find $ z_M $} According to Table 6.5 \cite{tk1}, $ z_M = 274 $.

\textbf{Find $ z_H $} Applying Equation 6.34 \cite{tk1} and 6.35 \cite{tk1}:\\
$ \beta_b = \arctan\left( \cos\alpha_t\tan\beta\right) = \arctan\left( \cos 20.47^\circ\tan 16.03^\circ\right) = 15.04^\circ$\\
$ z_H = \sqrt{\dfrac{2\cos\beta_b}{\sin(2\alpha_t)}} = \sqrt{\dfrac{2\times\cos 15.04^\circ}{\sin(2\times 20.74^\circ)}} = 1.71$

\textbf{Find $ z_\varepsilon $} Obtaining $ z_\varepsilon $ through calculations:
\begin{multline*}
\varepsilon_\alpha = \dfrac{\sqrt{d_{a1}^2-d_{b1}^2}+\sqrt{d_{a2}^2-d_{b2}^2}-2a\sin\alpha_t}{2\pi m_t{\cos\alpha_t}/{\cos\beta}}\\
= \dfrac{\sqrt{87.24^2-78.22^2}+\sqrt{280.76^2-260.07^2}-2\times 180\times\sin 20.74^\circ}{2\pi \times 2\times{\cos 20.74^\circ}/{\cos 16.03^\circ}} = 1.38
\end{multline*}
$ \varepsilon_\beta = b\dfrac{\sin\beta}{m_t\pi} = 54\times\dfrac{\sin 16.03^\circ}{2\times\pi}=2.37 >1 \Rightarrow z_\varepsilon = \varepsilon_\alpha^{-0.5} = 1.38^{-0.5} = 0.85 $

\textbf{Find $ K_H $ and $ K_F $} We find $ K_H $, $ K_F $ using Equation 6.39 and 6.45 \cite{tk1}.

The gear drive velocity determines the accuracy grade, Table 6.13 \cite{tk1}. In case of helical gear, it is:\\
$ v_1=6.34 \unit{m/s} \leq 10 \unit{m/s}\Rightarrow \text{AG} = 8 $

From Table P2.3 \cite{tk1}, using interpolation, we approximate:\\
$ K_{Hv} = 1.04$, $ K_{Fv} = 1.18$

From Table 6.14 \cite{tk1}, using interpolation, we approximate:\\
$ K_{H\alpha} = 1.06$, $ K_{F\alpha} = 1.3 $

Knowing $ K_{H\beta} $ and $ K_{H\beta} $ from previous section, multiply all the values to obtain $ K_H $ and $ K_F $:\\
$ K_H = K_{H\beta}K_{Hv}K_{H\alpha} = 1.04\times 1.1 \times 1.06  = 1.21 $\\
$ K_F = K_{F\beta}K_{Fv}K_{F\alpha} = 1.1\times 1.18 \times 1.3 = 1.68 $

\textbf{Find $ \sigma_H $} Replacing all the variables gives:
\begin{multline*}
\sigma_H = 274\times 1.71 \times 1.55 \times \sqrt{2\times 46423.73 \times 1.21 \times\dfrac{3.32+1}{54 \times 3.32 \times 83.24^2}}\\
= 249.04 \unitp{MPa} \leq 432.21 \unit{MPa}
\end{multline*}
which satisfies the condition.
\subsection{Bending stress analysis}
For safety reasons, Equation 6.43 and 6.44 \cite{tk1} must be met for both pairs of gears:
\[
\begin{array}{l@{{} \leq {}}l}
\sigma_{F1} = \dfrac{2T_{sh1}K_FY_\varepsilon Y_\beta Y_{F1}}{bd_1m_t\cos\beta} & [\sigma_{F1}]\\ 
\sigma_{F2} = \sigma_{F1}Y_{F2}/Y_{F1} & [\sigma_{F2}]
\end{array}
\]

\textbf{Find $ Y_\varepsilon $} Using $ \varepsilon_\alpha $ calculated in the previous section, 
$ Y_\varepsilon= \varepsilon_\alpha^{-1} = 1.38^{-1} = 0.72 $

\textbf{Find $ Y_\beta $} The value of $ Y_\beta $ is calculated using the equation on p.108 \cite{tk1}:\\
$ Y_\beta = 1-{\beta}/{140^\circ}=  1-{16.03^\circ}/{140^\circ}= 0.89$

\textbf{Find $ Y_F $} Using the formula $ z_v = z\cos^{-3}(\beta) $ and Table 6.18 \cite{tk1}:\\
$ z_{v1} = z_1\cos^{-3}(\beta) = 40\times\cos^{-3}(16.03^\circ) = 45.05\Rightarrow Y_{F1} = 3.67 $\\
$ z_{v2} = z_2\cos^{-3}(\beta) = 133\times\cos^{-3}(16.03^\circ) = 149.81\Rightarrow Y_{F2} = 3.60 $

\textbf{Find $ K_F $} The value of $ K_F $ has already been found in the previous section. $ K_F = 1.68 $

\textbf{Find $ \sigma_F $} The module in Equation 6.43 \cite{tk1} is $ m_n = m_t\cos\beta $. Replacing all the values gives:
\[
\begin{array}{l@{{} \leq {}}l}
\sigma_{F1} = \dfrac{2\times 46423.73 \times 1.68 \times 0.72 \times 0.89 \times 3.67}{54 \times 83.24 \times 2 \times\cos16.03^\circ} = 42.44 \unitp{MPa} & 217.57 \unit{MPa}\\ 
\sigma_{F2} = {42.44 \times 3.60}/{3.66} = 41.57 \unitp{MPa}& 201.65 \unit{MPa}
\end{array}
\]
which satisfies the conditions.
\subsection{Overloading analysis}
From the load diagram, we determine the overloading factor:
\[
K_{qt}=\sqrt{\dfrac{\left(\dfrac{T_1}{T}\right)^2t_1 + \left(\dfrac{T_2}{T}\right)^2t_2}{t_1+t_2}}= \sqrt{\dfrac{\left(\dfrac{T}{T}\right)^2\times15 + \left(\dfrac{0.7T}{T}\right)^2\times11}{15+11}} = 1.13
\]
Using the values of $ [\sigma_H]_{max} $ and $ [\sigma_F]_{max} $ calculated in previous section combined with Equation 6.48 and 6.49 \cite{tk1}, we are able to verify the stresses are below overloading limits. Replacing all the variables gives
\[
\begin{array}{l}
\sigma_{Hmax}=\sigma_H\sqrt{K_{qt}} = 249.04\times\sqrt{1.13} = 264.64 \unitp{MPa} \leq 952.00 \unit{MPa}\\
\sigma_{F1max}=\sigma_{F1s}K_{qt} =42.44\times 1.13 = 47.92 \unitp{MPa} \leq 440 \unit{MPa}\\
\sigma_{F2max}=\sigma_{F2}K_{qt} =41.57\times 1.13 = 46.95 \unitp{MPa} \leq 272.00 \unit{MPa}\\
\end{array}
\]
which satisfy the conditions.

In summary, we have the following table:
%\subsection{Force on shafts}
%$ F_t = \dfrac{2T_{sh1}}{d_{w1}} \approx 2402.28 \unit{(N)}$\\
%$ F_r = F_t\tan\alpha_{tw} \approx 899.55 \unit{(N)}$\\
%$ F_a = F_t\tan\beta\approx 580.75 \unit{(N)}$\\

\begin{table}[ht]
	\centering
	\begin{tabular}{lp{0.2\linewidth}p{0.2\linewidth}p{0.2\linewidth}}\toprule
		& Gear drive & Pinion & Driven gear \\ \midrule
		%		$ [P] \unitp{kW} $	&	19.3	&	-		&	-		\\
		$ a\unitp{mm}    $	&	180	&	-		&	-		\\
		$ d\unitp{mm}    $	&	-		&	83.24	&	276.76	\\
		$ d_a\unitp{mm}  $	&	-		&	87.24	&	280.76	\\
		$ d_b\unitp{mm}  $	&	-		&	78.22	&	260.07	\\
		$ d_f\unitp{mm}  $	&	-		&	78.24	&	271.76	\\
		$ m\unitp{mm}    $	&	2	&	-		&	-		\\
		$ u			     $	&	3.32	&	-		&	-		\\
		$ \alpha_t\unitp{^\circ}    $	&	20.74	&	-		&	-		\\
		$ \beta\unitp{^\circ}    $	&	16.03	&	-		&	-		\\
		\bottomrule
	\end{tabular}
	\caption{Gear drive specifications. Material of choice is 45 steel}
\end{table}